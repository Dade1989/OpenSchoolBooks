
\begin{itemize}
  \item introduzione intelligenza artificiale
  \item machine learning:
  \begin{itemize}
    \item supervised learning: regressione, classificazione
    \item unsupervised learning: clustering, riduzione della dimensionalità
    \item reinforcement learning
  \end{itemize}
  \item deep learning: neural networks
\end{itemize}

%-------------------------------------------------------------------------------
\section{Introduzione}
\paragraph{Cosa si intende per intelligenza artificiale.}
\paragraph{Nessun pasto è gratis -- compromesso bias-varianza.}
\href{https://it.wikipedia.org/wiki/Compromesso_bias-varianza}{https://it.wikipedia.org/wiki/Compromesso\_bias-varianza}
\paragraph{La maledizione della dimensionalità.}
\paragraph{Deep learning.}

%-------------------------------------------------------------------------------
\section{Machine learning}\index{Machine learning}


\subsection{Supervised learning}\index{Supervised learning}
Il supervised learning (o apprendimento supervisionato) è un paradigma che permette di allenare un modello.

\subsubsection{Regressione}
La regressione consiste nell'approssimazione di funzioni continue.
\paragraph{Regressione lineare.}
\paragraph{Regressione lineare generalizzata.}

\subsubsection{Classificazione}
La classificazione consiste nell'identificazione di una categoria alla quale appartiene un oggetto.


\subsection{Unsupervised learning}\index{Unsupervised learning}

\subsubsection{Dimensionality reduction}\index{Dimensionality reduction}
La riduzione delle dimensioni di un problema consente di:
\begin{itemize}
    \item ridurre la complessità del problema
    \item mantenendo solo le informazioni principali
\end{itemize}

\begin{example}[Compressione immagini]
\end{example}

\subsubsection{Clustering}\index{Clustering}
Il clustering è il raggruppamento di oggetti in insiemi che dimostrano caratteristiche simili.


\subsection{Reinforcement learning}\index{Reinforcement learning}

%-------------------------------------------------------------------------------
\section{Deep learning}\index{Deep learning}
\dots


