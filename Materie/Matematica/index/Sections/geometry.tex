
Introduzione storica:
\begin{itemize}
  \item Euclide
  \item Cartesio
  \item Riemann
\end{itemize}

\chapter{Geometria nel piano}
\section{Geometria euclidea}
\subsection{Introduzione}
\subsection{Rette e angoli}
\subsection{Triangoli}
\subsection{Circonferenza}

\section{Geometria cartesiana}
\subsection{Coordinate cartesiane}
\subsection{Punto, distanze, retta}
\begin{definition}[Punto]
Dato un sistema di coordinate cartesiane, un punto $P$ nel piano è individuato dalle sue due coordinate $(x,y)$.
\end{definition}
\begin{definition}[Distanza tra due punti]
La distanza tra due punti nel piano viene calcolata usando il teorema di Pitagora
\begin{equation}
    d_{PQ}^2 = (x_Q - x_P)^2 + (y_Q - y_P)^2 \ .
\end{equation}
\end{definition}
{\color{red} Perchè la distanza è data come una definizione? Geometria di Riemann: la distanza definisce tutte le proprietà di una geometria}

\begin{definition}[Retta]
    La retta può essere definita come l'insieme di punti $(x,y)$ equidistanti da due punti dati $P_1(x_1,y_1)$, $P_2(x_2,y_2)$.
\end{definition}
Partendo dalla definizione
\begin{equation}
\begin{aligned}
    d_1^2 & = d_2^2 \\
    (x - x_1)^2 + (y-y_1)^2 & = (x - x_2)^2 + (y-y_2)^2 \\
    x^2 - 2 x x_1 + x_1^2 + y^2 - 2 y y_1 + y_1^2 & = x^2 - 2 x x_2 + x_2^2 + y^2 - 2 y y_2 + y_2^2 \\
\end{aligned}
\end{equation}
\begin{equation}
  \qquad \rightarrow \qquad 2(x_2 - x_1) x + 2(y_2 - y_1) y + x_1^2 + y_1^2 - x_2^2 - y_2^2 = 0 \ .
\end{equation}
Quindi l'equazione generale della retta può essere riscritta nella forma
\begin{equation}
    A x + B y + C = 0 \ .
\end{equation}

\subsection{Trasformazioni di coordinate cartesiane e trasformazioni di curve}
\subsubsection{Traslazione dell'origine}
Sistema di coordinate $O'x'y'$ con assi paralleli al sistema di coordinate $Oxy$ e coordinate dell'origine $x_{O'}$, $y_{O'}$
\begin{equation}
    \begin{cases}
        x' = x - x_{O'} \\
        y' = y - y_{O'} \\
    \end{cases}
    \qquad \qquad
    \begin{cases}
        x  = x'+ x_{O'} \\
        y  = y'+ y_{O'} \\
    \end{cases}
\end{equation}
\subsubsection{Rotazione attorno all'origine}
Sistema di coordinate $O'x'y'$ origine coincidente con quella del sistema di coordinate $Oxy$ e assi rotati di un angolo $\theta$
\begin{equation}
    \begin{cases}
        x' =  x \cos \theta + y \sin \theta \\
        y' = -x \sin \theta + y \cos \theta \\
    \end{cases}
    \qquad \qquad
    \begin{cases}
        x  =  x'\cos \theta - y'\sin \theta \\
        y  =  x'\sin \theta + y'\cos \theta \\
    \end{cases}
\end{equation}

\subsection{Coniche}
\subsubsection{Parabola}
\begin{definition}[Parabola] Insieme dei punti $P$ del piano equidistanti da un punto $F$, chiamato \textbf{fuoco}, e una retta $r$ chiamata \textbf{direttrice}.
\begin{equation}
    \text{dist}(P,F) = \text{dist}(P,r) 
\end{equation}
\end{definition}
Scegliendo il fuoco $F(0, d)$ e la retta $r: y=-d$, si ricava l'equazione della parabola con vertice nell'origine e asse coincidente con l'asse $y$ degli assi cartesiani.
\begin{equation}
\begin{aligned}
    d^2_{PF} & = d^2_{Pr} \\
    (x - x_F)^2 + (y - y_F)^2 & = (y-y_r)^2 \\
    x^2 + (y - d)^2 & = (y+d)^2 \\
    x^2 + y^2 - 2dy + d^2 & = y^2 + 2dy + d^2 
\end{aligned}
\end{equation}
\begin{equation}
    \rightarrow \qquad 4d y = x^2 \qquad \rightarrow \qquad y = \dfrac{1}{4d} x^2 \ .
\end{equation}

\subsubsection{Ellisse}
\begin{definition}[Ellisse] Insieme dei punti $P$ del piano la cui somma delle distanze da due punti $F_1$, $F_2$, chiamati \textbf{fuochi} dell'ellisse, è costante.
    \begin{equation}
        \text{dist}(P, F_1) + \text{dist}(P, F_2) = 2a
    \end{equation}
\end{definition}
Scegliendo i fuochi $F_1(-c,0)$, $F_2(c,0)$
\begin{equation}
\begin{aligned}
    & \sqrt{( x - x_1 )^2 + (y-y_1)^2} + \sqrt{(x-x_2)^2 + (y-y_2)^2} = 2a \\
    & \sqrt{( x + c )^2 + y^2} = 2a - \sqrt{(x-c)^2 + y^2} \\
    & x^2 + 2 c x + c^2 + y^2  = 4a^2 - 4 a \sqrt{(x-c)^2 + y^2} +
    x^2 - 2 c x + c^2 + y^2 \\
    & 4c x = 4a^2 - 4a  \sqrt{(x-c)^2 + y^2} \\
    & (cx - a^2)^2 = ( - a  \sqrt{(x-c)^2 + y^2})^2 \\
    & c^2x^2 - 2a^2cx + a^4 = a^2(x-c)^2 + a^2 y^2 \\
    & (a^2-c^2) x^2 + a^2y^2 = a^2 (a^2-c^2) 
\end{aligned}
\end{equation}
Definendo $b^2 := a^2 - c^2 > 0$, si può riscrivere l'equazione dell'ellisse con il centro nell'origine e gli assi coincidenti con gli assi cartesiani come
\begin{equation}
  \dfrac{x^2}{a^2} + \dfrac{y^2}{b^2} = 1 \ .
\end{equation}


\subsubsection{Iperbole}
\begin{definition}[Iperbole] Insieme dei punti $P$ del piano la cui differenza delle distanze da due punti $F_1$, $F_2$, chiamati \textbf{fuochi} dell'ellisse, è costante in valore assoluto.
    \begin{equation}
       | \text{dist}(P, F_1) - \text{dist}(P, F_2) | = 2a
    \end{equation}
\end{definition}
Scegliendo i fuochi $F_1(-c,0)$, $F_2(c,0)$
\begin{equation}
\begin{aligned}
    & \sqrt{( x - x_1 )^2 + (y-y_1)^2} - \sqrt{(x-x_2)^2 + (y-y_2)^2} = \mp 2a \\
    & \sqrt{( x + c )^2 + y^2} = \mp 2a + \sqrt{(x-c)^2 + y^2} \\
    & x^2 + 2 c x + c^2 + y^2  = 4a^2 \mp 4 a \sqrt{(x-c)^2 + y^2} +
    x^2 - 2 c x + c^2 + y^2 \\
    & 4c x = 4a^2 \mp 4a  \sqrt{(x-c)^2 + y^2} \\
    & (cx - a^2)^2 = ( \mp a  \sqrt{(x-c)^2 + y^2})^2 \\
    & c^2x^2 - 2a^2cx + a^4 = a^2(x-c)^2 + a^2 y^2 \\
    & (c^2-a^2) x^2 - a^2y^2 = a^2 (c^2-a^2) 
\end{aligned}
\end{equation}
Definendo $b^2 := c^2 - a^2 > 0$, si può riscrivere l'equazione dell'ellisse con il centro nell'origine e gli assi coincidenti con gli assi cartesiani come
\begin{equation}
  \dfrac{x^2}{a^2} - \dfrac{y^2}{b^2} = 1 \ .
\end{equation}


\chapter{Geometria nello spazio}
\section{Geometria euclidea}
\section{Geometria cartesiana}

