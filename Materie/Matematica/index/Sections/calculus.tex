

% ---------------------------------------------------------------------------------------
\chapter{Limiti}

\section{Infiniti e infinitesimi}

% ---------------------------------------------------------------------------------------
\chapter{Derivate}

\section{Definizioni}
\begin{definition}[Derivata]
\begin{equation}
  f'(x) = \dfrac{d }{dx} f(x) := \lim_{\Delta x \rightarrow 0} \dfrac{f(x+\Delta x) - f(x)}{\Delta x}
\end{equation}
\end{definition}

\paragraph{Interpretazione geometrica}

\section{Regole di derivazione}
\subsection{Regole}
\paragraph{Derivata della somma di due funzioni e il prodotto per uno scalare}
\begin{equation}
\begin{aligned}
    (f(x) + g(x))' & = f'(x) + g'(x) \\
    (a f(x))' & = a f'(x) \\
\end{aligned}
\end{equation}
\begin{property}[Operatore lineare] La derivata è un operatore lineare.
\end{property}

\paragraph{Derivata del prodotto di due funzioni}
\paragraph{Derivata del rapporto di due funzioni}
\paragraph{Derivata di una funzione composta}

\subsection{Dimostrazioni}

\paragraph{Derivata della somma di due funzioni}
\paragraph{Derivata del prodotto di due funzioni}
\begin{equation}
\begin{aligned}
\dfrac{d}{dx} \left(f(x) g(x) \right) & =
     \lim_{\Delta x \rightarrow 0} \dfrac{f(x+\Delta x) g(x+\Delta x) - f(x)g(x) }{\Delta x} =  \\
 & = \lim_{\Delta x \rightarrow 0} \dfrac{f(x+\Delta x) g(x+\Delta x) - f(x+\Delta x) g(x) + f(x+\Delta x) g(x) - f(x)g(x) }{\Delta x} =  \\
 & = \lim_{\Delta x \rightarrow 0} \dfrac{f(x+\Delta x) g(x+\Delta x) - f(x+\Delta x) g(x) + f(x+\Delta x) g(x) - f(x)g(x) }{\Delta x} =  \\
    & = \lim_{\Delta x \rightarrow 0} f(x+\Delta x) \dfrac{ g(x+\Delta x) - g(x)}{\Delta x} + \lim_{\Delta x \rightarrow 0} \dfrac{ f(x+\Delta x)  - f(x) }{\Delta x} g(x) =  f'(x)g(x) + f(x)g'(x) \ .
\end{aligned}
\end{equation}
\paragraph{Derivata del rapporto di due funzioni}
\begin{equation}
\begin{aligned}
    \dfrac{d}{dx} \left(\dfrac{f(x)}{ g(x)} \right) & =
     \lim_{\Delta x \rightarrow 0} \dfrac{1}{\Delta x} \left[ \dfrac{f(x+\Delta x)}{ g(x+\Delta x)} - \dfrac{f(x)}{g(x)} \right] =  \\
    & = \lim_{\Delta x \rightarrow 0} \dfrac{1}{\Delta x} \dfrac{f(x+\Delta x) g(x) - f(x)g(x+\Delta x)}{g(x+\Delta x)g(x)}  =  \\
    & = \lim_{\Delta x \rightarrow 0} \dfrac{1}{g(x+\Delta x)g(x)} \dfrac{f(x+\Delta x) g(x) - f(x) g(x) + f(x)g(x)  - f(x)g(x+\Delta x)}{\Delta x}  =  \\
    & = \lim_{\Delta x \rightarrow 0} \dfrac{1}{g(x+\Delta x)g(x)} \dfrac{f(x+\Delta x) - f(x)}{\Delta x} g(x) - \lim_{\Delta x \rightarrow 0} \dfrac{1}{g(x+\Delta x)g(x)} \dfrac{g(x+\Delta x) - g(x)}{\Delta x} f(x) =  \\
    & = \dfrac{f'(x) g(x)}{g^2(x)} - \dfrac{f(x)g'(x)}{g^2(x)} = \dfrac{f'(x)g(x) - f(x) g'(x)}{g^2(x)}
\end{aligned}
\end{equation}
\paragraph{Derivata di una funzione composta}
\begin{equation}
\begin{aligned}
    \dfrac{d}{dx} f(g(x)) & =
     \lim_{\Delta x \rightarrow 0} \dfrac{1}{\Delta x} \left[ f(g(x+\Delta x)) - f(g(x)) \right] =  \\
    & = \lim_{\Delta x \rightarrow 0} \dfrac{f(g(x+\Delta x)) - f(g(x))}{g(x+\Delta x) - g(x)} \dfrac{g(x+\Delta x) - g(x)}{\Delta x} =  \\
    & = \dots \\
    & = f'(g(x)) \, g'(x) \ .
\end{aligned}
\end{equation}

\section{Teoremi}
\subsection{Teorema di de l'Hopital}

\section{Tabella di derivate}

\section{Espansioni in serie}
\begin{definition}[Serie di Taylor] La serie di Taylor di una funzione $f(x)$ centrata in $x=x_0$ è la serie polinomiale
\begin{equation}
   T[f(x_0)](x) := \sum_{n=0}^{\infty} \dfrac{f^{(n)}(x_0)}{n!} (x-x_0)^n \ .
\end{equation}
\end{definition}
%
\begin{theorem}
La serie di Taylor troncata alla $n$-esima potenza,
\begin{equation}
  T^n[f(x_0)](x) = \sum_{i=0}^{n} \dfrac{f^{(i)}(x_0)}{i!} (x-x_0)^i \ ,
\end{equation}
è un'approssimazione dell'$n$-esimo ordine della funzione $f(x)$, i.e.
\begin{equation}
  f(x) - T^n[f(x_0)](x) \sim o(|x-x_0|^{n})
\end{equation}
\end{theorem}

\begin{definition}[Serie di MacLaurin] La serie di MacLaurin di una funzione $f(x)$ è definita come la sua serie di Taylor centrata in $x=0$.
\end{definition}

\section{Applicazioni}
\subsection{Studio funzione}

\subsection{Approssimazione locale di }

% ---------------------------------------------------------------------------------------
\chapter{Integrali}

\section{Definizioni}
\begin{definition}[Somma di Riemann] Data una funzione continua $f:[a,b] \rightarrow \mathbb{R}$, e una partizione $\mathcal{P} = \left\{x_0, x_1, \dots, x_n | a = x_0 < x_1 < \dots x_n = b \right\}$ dell'intervallo $[a,b]$, la somma di Riemann viene definita come
    \begin{equation}
        \sigma_n = \sum_{i=1}^{n} f(\xi_i) \ (x_{i} - x_{i-1})
    \end{equation}
con $\xi_i \in [x_{i-1}, x_i]$.
\end{definition}

\begin{definition}[Integrale di Riemann] Definendo $\Delta x := \max_i(x_i - x_{i-1}) $, l'integrale di Riemann viene definito come il limite della somma di Riemann per $\Delta x  \rightarrow 0$ (e di conseguenza il numero di intervalli della partizione $n \rightarrow \infty$), e viene indicato come
    \begin{equation}
        \int_{x=a}^b f(x) dx = \lim_{\Delta x \rightarrow 0} \sigma_n
    \end{equation}
\end{definition}

\begin{definition}[Integrale definito]
\end{definition}

\paragraph{Interpretazione geometrica}

\begin{definition}[Integrale indefinito]
\end{definition}

\section{Proprietà}

\section{Teoremi}
\begin{theorem}[Teorema della media]
\end{theorem}

\begin{theorem}[Teorema fondmaentale del calcolo infinitesimale]
    \begin{equation}
        \dfrac{d}{dx} \int_{t=a}^{x} f(t) dt = f(x)
    \end{equation}
\end{theorem}
\begin{equation}
    \begin{aligned}
        \dfrac{d}{dx} \int_{t=a}^{x} f(t) dt & = 
        \lim_{\Delta x \rightarrow 0} \dfrac{1}{\Delta x} \left[ \int_{t=a}^{x+\Delta x} f(t) dt - \int_{t=a}^{x} f(t) dt\right] = \\
        & = \lim_{\Delta x \rightarrow 0} \dfrac{1}{\Delta x} \int_{t=x}^{x+\Delta x} f(t) dt = \\
        & = \lim_{\Delta x \rightarrow 0} \dfrac{1}{\Delta x} \Delta x \, f(\xi) = \qquad \qquad \qquad \text{(con $\xi \in [x, x+\Delta x]$)} \\
        & = f(x)  \  .
    \end{aligned}
\end{equation}

\section{Integrali fondamentali}

\section{Regole di integrazione}
\subsection{Integrazione per parti}
\begin{itemize}
 \item Definendo $F(x)$, $G(x)$ le primitive delle funzioni $f(x)$, e $g(x)$
 \item Integrando in $x$ dalla regola di \textbf{derivazione del prodotto} $(F(x)G(x))' = F'(x)G(x) + F(x)G'(x)$, riscritta isolando il termine $F'(x)G(x) = (F(x)G(x))' - F(x)G'(x)$
\end{itemize}
si ottiene
\begin{equation}
\begin{aligned}
    \int f(x) G(x) dx & = \int (F(x) G(x))' dx - \int F(x) G'(x) dx = \\
    &= F(x)G(x) - \int F(x) G'(x) dx 
\end{aligned}
\end{equation}

\subsection{Integrazione con sostituzione}
\begin{itemize}
    \item Definendo la funzione composta $\overline{F}(x) = F(y(x))$ e le derivate
        \begin{equation}
            \overline{f}(x) = \dfrac{d}{dx} \overline{F}(x) \qquad , \qquad f(y) = \dfrac{d}{dy}F(y)
        \end{equation}
    \item Partendo dalla regola di \textbf{derivazione della funzione composta}, $\overline{F}(x) = F(y(x))$
        \begin{equation}
            \overline{f}(x) = \dfrac{d}{dx} \overline{F}(x) = \dfrac{d}{dx} F(y(x)) = \dfrac{d F}{dy}(y(x)) \dfrac{d y}{d x}(x) = f(y(x)) y'(x)
        \end{equation}
\end{itemize}
Usando il teorema fondamentale del calcolo infinitesimale
\begin{equation}
\begin{aligned}
    F(y) & = \int f(y) dy \\
    \overline{F}(x) & = \int \overline{f}(x) dx = \int f(y(x)) \ y'(x) dx
\end{aligned}
\end{equation}
Se si introduce la dipendenza $y(x)$ nella prima equazione, si ottiene l'uguaglianza tra le ultime due espressioni, $F(y(x)) = \overline{F}(x)$, e quindi
\begin{equation}
  \int f(y) dy = \int f(y(x)) \ y'(x) dx \ .
\end{equation}
