

% ---------------------------------------------------------------------------------------
\chapter{Limiti}

\section{Infiniti e infinitesimi}

% ---------------------------------------------------------------------------------------
\chapter{Derivate}

\section{Definizioni}
\begin{definition}[Derivata]
\begin{equation}
  f'(x) = \dfrac{d }{dx} f(x) := \lim_{\Delta x \rightarrow 0} \dfrac{f(x+\Delta x) - f(x)}{\Delta x}
\end{equation}
\end{definition}

\section{Regole di derivazione}
\subsection{Regole}
\paragraph{Derivata della somma di due funzioni}
\begin{property}[Operatore lineare] La derivata è un operatore lineare.
\end{property}

\paragraph{Derivata del prodotto di due funzioni}
\paragraph{Derivata del rapporto di due funzioni}
\paragraph{Derivata di una funzione composta}

\subsection{Dimostrazioni}

\paragraph{Derivata della somma di due funzioni}
\paragraph{Derivata del prodotto di due funzioni}
\begin{equation}
\begin{aligned}
\dfrac{d}{dx} \left(f(x) g(x) \right) & =
     \lim_{\Delta x \rightarrow 0} \dfrac{f(x+\Delta x) g(x+\Delta x) - f(x)g(x) }{\Delta x} =  \\
 & = \lim_{\Delta x \rightarrow 0} \dfrac{f(x+\Delta x) g(x+\Delta x) - f(x+\Delta x) g(x) + f(x+\Delta x) g(x) - f(x)g(x) }{\Delta x} =  \\
 & = \lim_{\Delta x \rightarrow 0} \dfrac{f(x+\Delta x) g(x+\Delta x) - f(x+\Delta x) g(x) + f(x+\Delta x) g(x) - f(x)g(x) }{\Delta x} =  \\
\end{aligned}
\end{equation}
\paragraph{Derivata del rapporto di due funzioni}
\paragraph{Derivata di una funzione composta}

\section{Teoremi}
\subsection{Teorema di de l'Hopital}

\section{Tabella di derivate}

\section{Espansioni in serie}
\begin{definition}[Serie di Taylor] La serie di Taylor di una funzione $f(x)$ centrata in $x=x_0$ è la serie polinomiale
\begin{equation}
   T[f(x_0)](x) := \sum_{n=0}^{\infty} \dfrac{f^{(n)}(x_0)}{n!} (x-x_0)^n \ .
\end{equation}
\end{definition}
%
\begin{theorem}
La serie di Taylor troncata alla $n$-esima potenza,
\begin{equation}
  T^n[f(x_0)](x) = \sum_{i=0}^{n} \dfrac{f^{(i)}(x_0)}{i!} (x-x_0)^i \ ,
\end{equation}
è un'approssimazione dell'$n$-esimo ordine della funzione $f(x)$, i.e.
\begin{equation}
  f(x) - T^n[f(x_0)](x) \sim o(|x-x_0|^{n})
\end{equation}
\end{theorem}

\begin{definition}[Serie di MacLaurin] La serie di MacLaurin di una funzione $f(x)$ è definita come la sua serie di Taylor centrata in $x=0$.
\end{definition}

\section{Applicazioni}
\subsection{Studio funzione}

\subsection{Approssimazione locale di }

% ---------------------------------------------------------------------------------------
\chapter{Integrali}

\begin{theorem}[Teorema fondmaentale del calcolo infinitesimale]
\end{theorem}
