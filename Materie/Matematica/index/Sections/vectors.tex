% Introduzione: motivazioni
Motivazione:
\begin{itemize}
  \item non tutti gli oggetti di interesse della Matematica, della Fisica o delle Scienze in generale possono essere adeguatamente rappresentati da un singolo numero
  \item esempi: posizione, velocità, forza,\dots
\end{itemize}

Storia:
\begin{itemize}
  \item \dots
  \item da vettori nello spazio fisico a struttura astratta matematica
\end{itemize}


\chapter{Algebra vettoriale}

\section{Introduzione}
Nello studio della Fisica e delle scienze in generale, si incontrano alcune grandezze che non possono essere rappresentate adeguatamente con un numero, opportunamente accompagnato dalle unità di misura se necessario. Alcuni esempi sono la posizione, la velocità o l'accelerazione di un punto nello spazio, o una forza; {\color{red} Esempi \dots Esempi di tensori: rotazioni, inerzia \dots}

\section{Definizioni}
\begin{definition}[Spazio vettoriale] Uno spazio vettoriale è una struttura matematica composta da:
    \begin{itemize}
        \item un insieme $V$, i cui elementi $\mathbf{v} \in V$ sono chiamati \textbf{vettori}
        \item un campo $F$, i cui elementi $a \in F$ sono chiamati \textbf{scalari}
        \item due operazioni chiuse rispetto a $V$, cioè il cui risultato è un elemento che appartiene a $V$, che soddisfano determinate proprietà
        \begin{itemize}
            \item \textbf{somma vettoriale} di due vettori $\mathbf{u}$, $\mathbf{v}$ $\in V$: 
                \begin{equation}
                    \mathbf{u} + \mathbf{v} = \mathbf{w} \in V
                \end{equation}
            \item \textbf{moltiplicazione per uno scalare} di un vettore $\mathbf{u} \in V$ e uno scalare $a \in F$:
                \begin{equation}
                    a \mathbf{v} = \mathbf{w} \in V
                \end{equation}
        \end{itemize}
    \end{itemize}
\end{definition}

\begin{property}[Proprietà delle operazioni]
\end{property}

\begin{definition}[Combinazione lineare di vettori] Una combinazione lineare dei vettori $\{ \mathbf{v}_k \}_{k=1:K}$ è un vettore la cui espressione che può essere scritto come
    \begin{equation}
        a_1 \mathbf{v}_1 + a_2 \mathbf{v}_2 + \dots + a_K \mathbf{v}_K \ , 
    \end{equation}
    dove i coefficienti $a_k$, $k=1:K$, sono degli scalari appartenenti al campo $F$.
\end{definition}

\begin{definition}[Vettori linearmente indipendenti] Un insieme di vettori $\{ \mathbf{v}_k \}_{k = 1:K}$ è un insieme di vettori linearmente indipendenti, se una loro combinazione lineare ha come risultato il vettore nullo solo se tutti i coefficienti della combinazione sono nulli, cioé
    \begin{equation}
        \mathbf{0} = a_1 \mathbf{v}_1 + \dots + a_K \mathbf{v}_K \qquad \rightarrow \qquad a_k = 0 \ , \quad \forall k = 1:K \ . 
    \end{equation}
\end{definition}
Dalla definizione, segue immediatamente che nessun vettore dell'insieme può essere scritto come una combinazione lineare degli altri vettori. {\color{red} Se così non fosse,\dots}

\begin{definition}[Base e dimensione di uno spazio] Una base di uno spazio lineare è un insieme massimo di vettori linearmente indipendenti, $\mathcal{B} = \{ \mathbf{b}_k \}_{k=1:N}$. La dimensione dello spazio lineare è definita come il numero degli elementi della base.
\end{definition}

\begin{definition}[Componenti di un vettore rispetto a una base] Le componenti $\{ v^k \}_{k = 1:N}$ di un vettore $\mathbf{v}$ nella base $\mathcal{B} = \{ \mathbf{b}_k \}_{k=1:N}$ vengono definite come i coefficienti della combinazione lineare
    \begin{equation}
        \mathbf{v} = v_1 \mathbf{b}_1 + \dots + v_N \mathbf{b}_N = \sum_{k=1}^N v^k \mathbf{b}_k \ .
    \end{equation}
\end{definition}

{\color{red} Un vettore è \textbf{invariante} rispetto alla base usata per descriverlo: se si cambia la base, le componenti cambiano di conseguenza.}

\section{Spazi vettoriali con prodotto interno}
\begin{definition}[Prodotto interno] Il prodotto interno $\cdot: V \times V \rightarrow F$ è un'operazione lineare tra due elementi dello spazio vettoriale, che restituisce uno scalare non negativo, con le seguenti proprietà
    \begin{itemize}
        \item simmetria: $\mathbf{v} \cdot \mathbf{w} = \mathbf{w} \cdot \mathbf{v}$ 
        \item linearità: $( a \mathbf{u} + b \mathbf{v} ) \cdot \mathbf{w} = a \mathbf{u} \cdot \mathbf{w} + b \mathbf{v} \cdot \mathbf{w}$
        \item non-negatività: $\mathbf{v} \cdot \mathbf{v} \ge 0$, con l'uguaglianza che vale solo se $\mathbf{v} = \mathbf{0}$
    \end{itemize}
\end{definition}

\begin{definition}[Norma indotta dal prodotto interno] La norma di un vettore $\mathbf{v}$ indotta da un prodotto interno è definita come
    \begin{equation}
        \| \mathbf{v} \|^2 = \mathbf{v} \cdot \mathbf{v} \ .
    \end{equation}
\end{definition}

\begin{definition}[Base ortonormale] Una base ortonormale $\{ \mathbf{\hat{e}}_k \}_{k=1:N}$, è una base i cui vettori sono legati dalla relazione
    \begin{equation}
        \mathbf{\hat{e}}_i \cdot \mathbf{\hat{e}}_k = \delta_{ik} =
        \begin{cases} 1 \ , \qquad i = k \\ 0 \ , \qquad i \ne k \end{cases}
    \end{equation}
\end{definition}

\section{Spazi vettoriali bi- e tri-dimenisonali}
\subsection{Spazio vettoriale bidimensionale}
\paragraph{Prodotto interno.}
{\color{red} \dots}
\begin{equation}
    \mathbf{u} \cdot \mathbf{v} =
    ( u^1 \mathbf{\hat{e}}_1 + u^2 \mathbf{\hat{e}}_2) \cdot ( v^1 \mathbf{\hat{e}}_1 + v^2 \mathbf{\hat{e}}_2) = u^1 v^1 + u^2 v^2 \ . 
\end{equation}
Usando una base ortogonale $\{ \mathbf{\hat{E}}_i \}_{i=1:2}$ che ha il primo vettore orientato come $\mathbf{u}$, si può scrivere
\begin{equation}
    \mathbf{u} \cdot \mathbf{v} =
      U^1 \mathbf{\hat{E}}_1 \cdot ( V^1 \mathbf{\hat{E}}_1 + V^2 \mathbf{\hat{E}}_2) = U^1 V^1 = U V \cos \theta_{\mathbf{u} \mathbf{v}} \ . 
\end{equation}
\paragraph{Prodotto vettoriale.}

\subsection{Spazio vettoriale tridimensionale}

\paragraph{Prodotto interno.}
\begin{equation}
    \mathbf{u} \cdot \mathbf{v} =
    ( u^1 \mathbf{\hat{e}}_1 + u^2 \mathbf{\hat{e}}_2 + u^3 \mathbf{\hat{e}}_3) \cdot ( v^1 \mathbf{\hat{e}}_1 + v^2 \mathbf{\hat{e}}_2 + v^3 \mathbf{\hat{e}}_3) = u^1 v^1 + u^2 v^2 + u^3 v^3 \ . 
\end{equation}
Usando una base ortogonale $\{ \mathbf{\hat{E}}_i \}_{i=1:3}$ che ha:
\begin{itemize}
    \item il primo vettore orientato come $\mathbf{u}$, tale che il vettore $\mathbf{u}$ può essere scritto come $\mathbf{u} = U^1 \mathbf{\hat{E}}_1$
    \item il secondo vettore $\mathbf{\hat{E}}_2$ tale che il vettore $\mathbf{v}$ può essere scritto come $\mathbf{v} = V^1 \mathbf{\hat{E}}_1 + V^2 \mathbf{\hat{E}}_2$
    \item il terzo vettore $\mathbf{\hat{E}}_3$ orientato di conseguenza, ortogonale ai due vettori $\mathbf{u}$, $\mathbf{v}$
\end{itemize}
Risulta ancora una volta dimostrata la relazione
\begin{equation}
    \mathbf{u} \cdot \mathbf{v} =
      U^1 \mathbf{\hat{E}}_1 \cdot ( V^1 \mathbf{\hat{E}}_1 + V^2 \mathbf{\hat{E}}_2) = U^1 V^1 = U V \cos \theta_{\mathbf{u} \mathbf{v}} \ . 
\end{equation}

\paragraph{Prodotto vettoriale.}
{\color{red} \dots}

\section{Applicazioni}
\subsection{Geometria}
\subsection{Fisica}

% ==============================================================================
\chapter{Coordinate in spazi euclidei e cenni di calcolo vettoriale}

\section{Introduzione}
\begin{definition}[Vettore posizione]
\end{definition}

\begin{definition}[Coordinate] Il vettore posizione in uno spazio euclideo $N$-dimensionale può essere espresso come una funzione di $N$ variabili indipendenti $\{ q^i \}_{i=1:N}$, dette coordinate,
    \begin{equation}
        \mathbf{r}(q^i) \ .
    \end{equation}
    Se ogni combinazione di coordinate identifica un punto nello spazio, le coordinate vengono definite \textbf{regolari}.
\end{definition}
\begin{example}[Coordinate cartesiane]
\end{example}
\begin{example}[Coordinate polari nel piano]
\end{example}
\begin{example}[Coordinate sferiche]
\end{example}

\section{Funzioni di più variabili - campi}
\begin{equation}
    f(\mathbf{r}) = f(\mathbf{r}(q^i)) = F(q^i)
\end{equation}

\subsection{Limiti e funzioni continue}

\subsection{Derivate}
\begin{definition}[Funzione derivabile]
\end{definition}


\begin{definition}[Derivate parziali] La derivata parziale di una funzione viene definita come la derivata della funzione rispetto a una delle variabili indipendenti, mantenendo costanti le altre variabili,
    \begin{equation}
        \dfrac{\partial f}{\partial q^i} = \lim_{\varepsilon \rightarrow 0} \dfrac{f(q^1, \dots, q^i+\varepsilon, \dots, q^N) - f(q^1, \dots, q^i, \dots, q^N) }{\varepsilon}
    \end{equation}
\end{definition}


\begin{definition}[Derivata direzionale] La derivata direzionale di un campo $f(\mathbf{r})$ nel punto $\mathbf{r}_0$ nella direzione identificata dal versore $\mathbf{\hat{t}}$ è definita come,
    \begin{equation}
        \nabla_{\mathbf{\hat{t}}} f(\mathbf{r}_0) = \lim_{\varepsilon \rightarrow 0} \dfrac{f(\mathbf{r}_0+\varepsilon \mathbf{\hat{t}})-f(\mathbf{r}_0)}{\varepsilon} \ ,
    \end{equation}
    cioè come il limite del rapporto incrementale del valore della funzione $f(\mathbf{r})$, muovendosi dal punto $\mathbf{r}_0$ al punto $\mathbf{r}_0 + \varepsilon \mathbf{\hat{t}}$.
\end{definition}

\subsection{Operatori differenziali}
\begin{definition}[Gradiente]
    \begin{equation}
        \nabla_{\mathbf{\hat{t}}} f =: \mathbf{\hat{t}} \cdot \mathbf{\nabla} f
    \end{equation}
\end{definition}
\begin{property}[Operatore nabla, $\nabla$ -- vettore formale]
\end{property}
\begin{property}[Coordinate cartesiane]
\end{property}
\begin{property}[Direzione di massima crescita]
\end{property}

\begin{definition}[Divergenza] La divergenza viene definita come l'operatore differenziale del primo ordine che prende un campo vettoriale $\mathbf{F}(\mathbf{r})$ e restituisce un campo scalare, che può essere rappresentato con il prodotto scalare tra il vettore formale nabla $\nabla$ e il campo vettoriale $\mathbf{F}(\mathbf{r})$
    \begin{equation}
        \nabla \cdot \mathbf{F}(\mathbf{r}) \ ,
    \end{equation}
\end{definition}

\begin{property}[Coordinate cartesiane]
L'espressione in coordinate cartesiane della divergenza di un campo vettoriale $\mathbf{F}(\mathbf{r})$ è
    \begin{equation}
        \begin{aligned}
            \nabla \cdot \mathbf{F}(\mathbf{r}) & = \left( \mathbf{\hat{x}} \dfrac{\partial}{\partial x} + \mathbf{\hat{y}} \dfrac{\partial}{\partial y} + \mathbf{\hat{z}} \dfrac{\partial}{\partial z} \right) \cdot (F_x \mathbf{\hat{x}} + F_y \mathbf{\hat{y}} + F_z \mathbf{\hat{z}} ) = \\
        & = \dfrac{\partial F_x}{\partial x} + \dfrac{\partial F_y}{\partial y} + \dfrac{\partial F_z}{\partial z} \ .
        \end{aligned}
    \end{equation}
\end{property}

\begin{property}[Flusso elementare]
\end{property}

\begin{definition}[Rotore] Il rotore viene definito come l'operatore differenziale del primo ordine che prende un campo vettoriale $\mathbf{F}(\mathbf{r})$ e restituisce un campo vettoriale, che può essere rappresentato con il prodotto vettore tra il vettore formale nabla $\nabla$ e il campo vettoriale $\mathbf{F}(\mathbf{r})$
    \begin{equation}
        \nabla \times \mathbf{F}(\mathbf{r}) \ ,
    \end{equation}
\end{definition}
\begin{property}[Coordinate cartesiane]
L'espressione in coordinate cartesiane del rotore di un campo vettoriale $\mathbf{F}(\mathbf{r})$ è
    \begin{equation}
        \begin{aligned}
            \nabla \times \mathbf{F}(\mathbf{r}) & = \left( \mathbf{\hat{x}} \dfrac{\partial}{\partial x} + \mathbf{\hat{y}} \dfrac{\partial}{\partial y} + \mathbf{\hat{z}} \dfrac{\partial}{\partial z} \right) \times (F_x \mathbf{\hat{x}} + F_y \mathbf{\hat{y}} + F_z \mathbf{\hat{z}} ) = \\
            & = \begin{vmatrix} \mathbf{\hat{x}} & \mathbf{\hat{y}} & \mathbf{\hat{z}} \\
            \frac{\partial}{\partial x} & \frac{\partial}{\partial y} & \frac{\partial}{\partial z} \\
            F_x & F_y & F_z \end{vmatrix} = \\
            & = \mathbf{\hat{x}} \left( \frac{\partial F_z}{\partial y} - \frac{\partial F_y}{\partial z} \right) +  
                \mathbf{\hat{y}} \left( \frac{\partial F_x}{\partial z} - \frac{\partial F_z}{\partial x} \right) +  
                \mathbf{\hat{z}} \left( \frac{\partial F_y}{\partial x} - \frac{\partial F_x}{\partial y} \right)    
        \end{aligned}
    \end{equation}
\end{property}
\begin{property}[Circuitazione elementare]
\end{property}

\subsection{Integrali}
\subsubsection{Integrali di linea}
\begin{definition}[Massa] L'integrale ``della massa'' $m$ di una curva $\gamma$, viene definito come l'integrale sulla curva di una funzione scalare $f(\mathbf{r})$, definita densità lineare,
    \begin{equation}
        m = \int_{\gamma} f(\mathbf{r})
    \end{equation}
\end{definition}
\begin{definition}[Lavoro] L'integrale ``del lavoro'' del campo vettoriale ``di forza'' $\mathbf{F}(\mathbf{r})$ lungo la linea $\gamma$ è definito come
    \begin{equation}
        L = \int_{\gamma} \mathbf{F}(\mathbf{r}) \cdot d \mathbf{r} = \int_{\gamma} \mathbf{F}(\mathbf{r}) \cdot \mathbf{\hat{t}} \ . 
    \end{equation}
\end{definition}
\begin{definition}[Circuitazione] La circuitazione di un campo vettoriale $\mathbf{F}(\mathbf{r})$ è definito come il suo integrale del lavoro lungo una linea chiusa $\gamma$,
    \begin{equation}
        \Gamma_{\gamma}(\mathbf{F}) := \oint_{\gamma} \mathbf{F}(\mathbf{r}) \cdot \mathbf{\hat{t}} \ .
    \end{equation}
\end{definition}
\subsubsection{Integrali di superficie}
\begin{definition}[Massa] L'integrale ``della massa'' $m$ di una superficie $S$, viene definito come l'integrale sulla superficie di una funzione scalare $f(\mathbf{r})$, definita densità superficiale,
    \begin{equation}
        m = \int_{S} f(\mathbf{r})
    \end{equation}
\end{definition}
\begin{definition}[Flusso]
    \begin{equation}
        \Phi_S(\mathbf{F}) = \int_S \mathbf{F}(\mathbf{r}) \cdot \mathbf{\hat{n}} \ .
    \end{equation}
\end{definition}
\subsubsection{Integrali di volume}
\begin{definition}[Massa] L'integrale ``della massa'' $m$ di una superficie $V$, viene definito come l'integrale sul volume di una funzione scalare $f(\mathbf{r})$, definita densità volumetrica,
    \begin{equation}
        m = \int_{V} f(\mathbf{r})
    \end{equation}
\end{definition}


\subsection{Teoremi}
\begin{theorem}[Teorema del gradiente]
    \begin{equation}
        \oint_{\partial V} f(\mathbf{r}) \mathbf{\hat{n}}(\mathbf{r}) = \int_V \mathbf{\nabla} f(\mathbf{r})
    \end{equation}
\end{theorem}

\begin{theorem}[Teorema della divergenza]
    \begin{equation}
        \Phi_{\partial V}(\mathbf{F}) = \oint_{\partial V} \mathbf{F}(\mathbf{r}) \cdot \mathbf{\hat{n}}(\mathbf{r}) = \int_V \mathbf{\nabla} \cdot \mathbf{F}(\mathbf{r})
    \end{equation}
\end{theorem}

\begin{theorem}[Teorema del rotore]
    \begin{equation}
        \Gamma_{\partial S}(\mathbf{F}) = \oint_{\partial S} \mathbf{F}(\mathbf{r}) \cdot \mathbf{\hat{t}}(\mathbf{r}) = \int_S \mathbf{\nabla} \times \mathbf{F}(\mathbf{r})
    \end{equation}
\end{theorem}

