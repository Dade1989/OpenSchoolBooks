% Introduzione: motivazioni
Motivazione:
\begin{itemize}
  \item non tutti gli oggetti di interesse della Matematica, della Fisica o delle Scienze in generale possono essere adeguatamente rappresentati da un singolo numero
  \item esempi: posizione, velocità, forza,\dots
\end{itemize}

Storia:
\begin{itemize}
  \item \dots
  \item da vettori nello spazio fisico a struttura astratta matematica
\end{itemize}


\chapter{Algebra vettoriale}

\section{Definizioni}
\begin{definition}[Spazio vettoriale] Uno spazio vettoriale è una struttura matematica composta da:
    \begin{itemize}
        \item un insieme $V$, i cui elementi $\mathbf{v} \in V$ sono chiamati \textbf{vettori}
        \item un campo $F$, i cui elementi $a \in F$ sono chiamati \textbf{scalari}
        \item due operazioni chiuse rispetto a $V$, cioè il cui risultato è un elemento che appartiene a $V$, che soddisfano determinate proprietà
        \begin{itemize}
            \item \textbf{somma vettoriale} di due vettori $\mathbf{u}$, $\mathbf{v}$ $\in V$: 
                \begin{equation}
                    \mathbf{u} + \mathbf{v} = \mathbf{w} \in V
                \end{equation}
            \item \textbf{moltiplicazione per uno scalare} di un vettore $\mathbf{u} \in V$ e uno scalare $a \in F$:
                \begin{equation}
                    a \mathbf{v} = \mathbf{w} \in V
                \end{equation}
        \end{itemize}
    \end{itemize}
\end{definition}

\begin{property}[Proprietà delle operazioni]
\end{property}

\begin{definition}[Base e dimensione di uno spazio]
\end{definition}


\subsection{Spazi vettoriali con prodotto interno}


\section{Applicazioni}
\subsection{Geometria}
\subsection{Fisica}


\chapter{Coordinate in spazi euclidei e cenni di calcolo vettoriale}
\begin{definition}[Vettore posizione]
\end{definition}

\begin{definition}[Coordinate]
\end{definition}
\begin{example}[Coordinate cartesiane]
\end{example}
\begin{example}[Coordinate polari]
\end{example}
\begin{example}[Coordinate sferiche e superficie terrestre]
\end{example}

