
\chapter{Algebra simbolica - Calcolo letterale}

\section{Monomi}
\begin{definition}[Monomio] Un monomio è un'espressione matematica costituita dal prodotto di un coefficiente esplicitamente numerico e una parte letterale, nella quale compaiono unicamente moltiplicazioni e potenze intere.
\end{definition}
\begin{remark} Viene richiesto che le potenze della parte letterale siano intere, per evitare di porre delle condizioni sulle basi delle potenze, essendo le potenze non intere definite solo per numeri reali positivi.
\end{remark}

\begin{definition}[Monomi simili] I monomi simili sono i monomi che hanno la stessa parte letterale.
\end{definition}

\subsection{Somma e differenza}
\subsection{Prodotto e divisione}
\subsection{Potenze e radici}
\begin{definition}[Potenze e radici intere] La potenza intera di ordine $n \in \mathbb{N}$ di un monomio $x$ è definita come il prodotto di $x$ per se stesso $n$ volte,
\begin{equation}
  p_n(x) := x^n = \underbrace{x \cdot x \cdot \dots \cdot x}_{\text{$n$ volte}} \ .
\end{equation}
L'operazione inversa, quando possibile, è definita come radice di ordine $n$,
\begin{equation}
  x := \sqrt[n]{p_n(x)} = p_n(x)^{\frac{1}{n}} \ .
\end{equation}
\end{definition}
\begin{remark} Per una comprensione più completa, bisogna rifarsi all'algebra dei numeri complessi \ref{book:complex_algebra}.
\end{remark}

\begin{definition}[Potenze e radici non intere]
\end{definition}

\subsection{Esponenziali e logaritmi}
\subsection{Funzioni armoniche}
\subsection{Funzioni iperboliche}

\section{Polinomi}
\begin{definition}[Polinomio]
Un polinomio reale di grado $n$ viene definito come una combinazione lineare dei monomi di grado $\le n$,
\begin{equation}
 p(x) = a_0 + a_1 x + \dots + a_n x^n = \sum_{i=0}^{n} a_i x^i
\end{equation}
\end{definition}

\section{Potenze e radici}
\begin{definition}[Potenze e radici intere]
\end{definition}


\chapter{Equazioni}

\chapter{Disequazioni}

\chapter{Sistemi di equazioni e di disequazioni}
