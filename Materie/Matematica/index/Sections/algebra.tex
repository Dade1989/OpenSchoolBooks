
\chapter{Algebra simbolica - Calcolo letterale}

% ------------------------------------------------------------------------------
\section{Monomi}
\begin{definition}[Monomio] Un monomio è un'espressione matematica costituita dal prodotto di un coefficiente esplicitamente numerico e una parte letterale, nella quale compaiono unicamente moltiplicazioni e potenze intere.
\end{definition}
\begin{remark} Viene richiesto che le potenze della parte letterale siano intere, per evitare di porre delle condizioni sulle basi delle potenze, essendo le potenze non intere definite solo per numeri reali positivi.
\end{remark}

\begin{definition}[Monomi simili] I monomi simili sono i monomi che hanno la stessa parte letterale.
\end{definition}

\subsection{Somma e differenza}
\subsection{Prodotto e divisione}
\subsection{Potenze e radici}
\begin{definition}[Potenze e radici intere] La potenza intera di ordine $n \in \mathbb{N}$ di un monomio $x$ è definita come il prodotto di $x$ per se stesso $n$ volte,
\begin{equation}
  p_n(x) := x^n = \underbrace{x \cdot x \cdot \dots \cdot x}_{\text{$n$ volte}} \ .
\end{equation}
L'operazione inversa, quando possibile, è definita come radice di ordine $n$,
\begin{equation}
  x := \sqrt[n]{p_n(x)} = p_n(x)^{\frac{1}{n}} \ .
\end{equation}
\end{definition}
\begin{remark} Per una comprensione più completa, bisogna rifarsi all'algebra dei numeri complessi \ref{book:complex_algebra}.
\end{remark}

\begin{definition}[Potenze e radici non intere]
\end{definition}

% ------------------------------------------------------------------------------
\section{Polinomi}
\begin{definition}[Polinomio]
Un polinomio reale di grado $n$ viene definito come una combinazione lineare dei monomi di grado $\le n$,
\begin{equation}
 p(x) = a_0 + a_1 x + \dots + a_n x^n = \sum_{i=0}^{n} a_i x^i
\end{equation}
\end{definition}

\begin{definition}[Zero di un polinomio] Viene definito zero -- o radice -- (reale) di un polinomio un numero $\overline{x}$ ($\in \mathbb{R}$) tale che $p(\overline{x}) = 0$.
\end{definition}

\begin{definition}[Scomposizione in fattori]
    Ogni polinomio di variabile reale e coefficienti reali può essere scomposto in fattori polinomiali di primo e secondo grado. Ad esempio un polinomio di grado $n$, può essere scomposto nel prodotto di $p_1$ polinomi di primo grado e $p_2$ polinomi di secondo grado,
    \begin{equation}
        \begin{aligned}
            P(x) & = a_0 + a_1 x + \dots + a_n x^n = \\
            & = (A_1 x+B_1) \dots (A_{p_1}x+B_{p_1}) (C_1 x^2+D_1 x+E_1) \dots (C_{p_2} x^2 + D_{p_2} x + E_{p_2} ) = \\
            & = K (x+\tilde{B}_1) \dots (x+\tilde{B}_{p_1}) (x^2 + \tilde{D}_1 x + \tilde{E}_1) \dots (x^2 + \tilde{D}_{p_2} x + \tilde{E}_{p_2})
        \end{aligned}
    \end{equation}
    con $p_1 + 2 p_2 = n$.
\end{definition}

\subsection{Scomposizioni notevoli}
La dimostrazione di queste identità viene ricavata facilmente tramite calcolo diretto.
\paragraph{Quadrato della somma}
\begin{equation}
    (a+b)^2 = a^2 + 2 ab + b^2
\end{equation}
\paragraph{Cubo della somma}
\begin{equation}
    (a+b)^3 = a^3 + 3 a^2 b + 3 a b^2 + b^3
\end{equation}
\paragraph{Potenza di un binomio}
\begin{equation}
    (a+b)^n = \sum_{k=0}^{n} \binom{n}{k} a^{n-k} b^{k}
\end{equation}
\paragraph{Differenza di quadrati}
\begin{equation}
    a^2 - b^2 = (a+b)(a-b)
\end{equation}
\paragraph{Somma e differenza di cubi}
\begin{equation}
    a^3 \mp b^3 = (a \mp b)(a^2 \pm ab + b^2)
\end{equation}
\paragraph{Differenza di potenze di grado qualsiasi}
\begin{equation}
\begin{aligned}
    a^n - b^n & =
        (a-b) \sum_{k=0}^{n-1} a^{n-1-k} b^k = \\
    & = (a-b)(a^{n-1} + a^{n-2}b + a^{n-3}b^2 + \dots + a b^{n-2} + b^{n-1})
\end{aligned}
\end{equation}
\paragraph{Somma di potenze di grado dispari}
Per ogni numero dispari $n = 2m + 1$, $m \in \mathbb{N}$, si può dimostrare che
\begin{equation}
\begin{aligned}
    a^{2m+1} + b^{2m+1} & =
        (a + b) \sum_{k=0}^{2m} (-1)^k a^{n-1-k} b^k = \\
    & = (a + b)(a^{n-1} - a^{n-2}b + a^{n-3}b^2 + \dots - a b^{n-2} + b^{n-1})
\end{aligned}
\end{equation}

\subsection{Divisione tra polinomi e resto}
\begin{definition}[Divisione tra polinomi]
    Dati due polinomi $P(x)$, $D(x)$ esistono due polinomi $Q(x)$, $R(x)$ tali che
    \begin{equation}
        P(x) = D(x) Q(x) + R(x) \ ,
    \end{equation}
    con il grado di $R(x)$ minore del grado di $Q(x)$. I polinomi $Q(x)$ e $R(x)$ vengono definiti rispettivamente \textbf{quoziente} e \textbf{resto} della divisione.
\end{definition}

\subsubsection{Divisione esatta e scomposizone di polinomi: regola di Ruffini}

\subsection{Formule di Viete e Newton}
{\color{red} Relazioni tra i coefficienti di un polinomio e le sue radici}


% ------------------------------------------------------------------------------
\section{Potenze e radici}
\begin{definition}[Potenze e radici intere]
\end{definition}

% ------------------------------------------------------------------------------
\section{Esponenziali e logaritmi}
\subsection{Esponenziale}
\begin{definition}[Esponenziale] L'elevamento a potenza di un numero $a$
\begin{equation}
   y = a^x
\end{equation}
è un operazione che coinvolge due numeri, $a$ detto base e $x$ detto esponente.
\end{definition}
\subsubsection{Potenze non intere, valori ammissibili}
\subsubsection{Proprietà}
\paragraph{Prodotto di potenze con la stessa base}
\begin{equation} a^m a^n = a^{m+n} \end{equation}
\paragraph{Potenza di potenza}
\begin{equation} (a^m)^n = a^{mn} \end{equation}
\paragraph{Prodotto di potenze con lo stesso esponente}
\begin{equation} a^m b^m = (ab)^{m} \end{equation}

% ------------------------------------------------------------------------------
\subsection{Logaritmo}
\begin{definition}[Logaritmo] Il logaritmo è l'operazione inversa
\begin{equation}
    x = \log_a y \qquad \text{se $y = a^x$}
\end{equation}
\end{definition}
\subsubsection{Potenze non intere, valori ammissibili}
\subsubsection{Proprietà}
\paragraph{Somma di logaritmi con la stessa base}
\begin{equation}
  \log_a m + \log_a n = \log_a (mn)
\end{equation}
Dimostrazione
\begin{equation}
\begin{cases}
  m = a^{\log_a m} \\
  n = a^{\log_a n} \\
  mn = a^{\log_a mn}
\end{cases}
\rightarrow
\begin{aligned}
    mn & = m \cdot n \\
    a^{\log_a mn} & = a^{\log_a m} a^{\log_a n} = a^{\log_a m + \log_a n}
\end{aligned}
\end{equation}
\begin{equation}
  \rightarrow \log_a mn = \log_a m + \log_a n
\end{equation}
\paragraph{Prodotto di un logaritmo per uno scalare}
\begin{equation}
  b \log_a m = \log_a m^{b}
\end{equation}
\paragraph{Cambio di base di un logaritmo}
\begin{equation}
     \log_b m = \log_b a \log_a m 
\end{equation}
Dimostrazione
\begin{equation}
\begin{cases}
    m = b^{\log_b m} \\
    a = b^{\log_b a} \\
    m = a^{\log_a m} = \left( b^{\log_b a} \right)^{\log_a m} = b^{\log_b a \, \log_a m}
\end{cases}
\end{equation}
e confrontando le due espressioni per $m$ si ottiene
\begin{equation}
    \rightarrow \log_b m = \log_b a \, \log_a m
\end{equation}

% ------------------------------------------------------------------------------
\section{Funzioni armoniche}
\subsection{La circonferenza e la definizione delle funzioni seno e coseno}

\subsection{La definizione delle funzioni tangente, cotangente, secante e cosecante}

\subsection{Formule del seno e coseno di somme e differenze}
\begin{equation}
\begin{aligned}
    \cos(\alpha \pm \beta) & = \cos(\alpha) \cos(\beta) \mp \sin(\alpha) \sin(\beta) \\
    \sin(\alpha \pm \beta) & = \sin(\alpha) \cos(\beta) \pm \cos(\alpha) \sin(\beta) \\
\end{aligned}
\end{equation}
\begin{equation}
\begin{aligned}
    \cos(\alpha) \cos(\beta) & = \dfrac{1}{2} \left[ \cos(\alpha + \beta) + \cos(\alpha-\beta) \right] \\
    \sin(\alpha) \sin(\beta) & = \dfrac{1}{2} \left[ \cos(\alpha + \beta) - \cos(\alpha-\beta) \right] \\
    \sin(\alpha) \cos(\beta) & = \dfrac{1}{2} \left[ \sin(\alpha + \beta) - \sin(\alpha-\beta) \right] \\
\end{aligned}
\end{equation}

% ------------------------------------------------------------------------------
\section{Funzioni iperboliche}


% ==============================================================================
\chapter{Equazioni}
\section{Equazioni algebriche}
\begin{definition}[Equazioni algebriche] \dots
\end{definition}

\subsection{Equazioni polinomiali}
\begin{definition}[Equazione polinomiale] Un'equazione polinomiale ha la forma
    \begin{equation} p(x) = 0 \end{equation}
dove $p(x)$ è un polinomio.
Il \textbf{grado} dell'equazione corrisponde al grado del polinomio $p(x)$, cioé alla potenza massima dei monomi. In maniera più esplicita, quindi, si può scrivere un'equazione polinomiale di grado $n$ come
    \begin{equation}
        a_n x^n + a_{n-1} x^{n-1} + \dots + a_1 x + a_0 = 0 \ , \qquad \text{con $a_n \ne 0$}
    \end{equation}
\end{definition}

\paragraph{Esistenza e numero delle soluzioni.} Un'equazione polinomiale di grado $n$ ha \textbf{al massimo} $n$ soluzioni reali. L'esistenza di soluzioni reali non è in generale garantita, mentre il \textbf{teorema fondamentale dell'algebra} assicura che esistano esattamente $n$ soluzioni complesse di un'equazione polinomiale con coefficienti complessi.

\subsubsection{Equazioni di primo grado}
La forma generale delle equazioni di primo grado è
\begin{equation}
    a x + b = 0 \ , \qquad \text{con $a \ne 0$}
\end{equation}
e la soluzione è
\begin{equation}
  x = -\dfrac{a_0}{a_1} \ .
\end{equation}
\subsubsection{Equazioni di secondo grado}
La forma generale delle equazioni di secondo grado è
\begin{equation}
    a x^2 + b x + c = 0 \ , \qquad \text{con $a \ne 0$}
\end{equation}
%
Un'equazione di secondo grado può ammettere nel campo dei numeri reali 2 soluzioni (distinte o coincidenti) o nessuna soluzione, a seconda del valore dell'espressione definita come \textbf{discriminante}, $\Delta := b^2 - 4 a c$:
\begin{itemize}
    \item $\Delta > 0$: due soluzioni reali distinte
    \item $\Delta = 0$: due soluzioni reali coincidenti
    \item $\Delta < 0$: nessuna soluzione reale
\end{itemize}
Quando il discriminante è non negativo, le soluzioni dell'equazione sono date dall'espressione
\begin{equation}
    x_{1,2} = \dfrac{-b \mp \sqrt{\Delta}}{2a} \ .
\end{equation}
%
\paragraph{Formula risolutiva dell'equazione di secondo grado.}
Una dimostrazione della formula risolutiva viene ricavata con la regola di completamento del quadrato
\begin{equation}
    \begin{aligned}
        0 & = a x^2 + b x + c = \\
          & = a x^2 + b x + \dfrac{b^2}{4a} - \dfrac{b^2}{4a} + c = \\
          & = a \left(x + \dfrac{b}{2a} \right)^2 - \dfrac{b^2}{4a} + c \\
    \end{aligned}
\end{equation}
\begin{equation}
    \rightarrow \left( x + \dfrac{b}{2a} \right)^2 = \dfrac{b^2}{4 a^2} - \dfrac{c}{a} = \dfrac{b^2 - 4 a c}{4a^2} = \dfrac{\Delta}{4a^2}
\end{equation}
\'E ora facile notare come questa equazione ha soluzioni solo quando il discriminante è non negativo. Quando il discriminante è non negativo, è possibile estrarre la radice quadra dell'espressione
\begin{equation}
    x_{1,2} + \dfrac{b}{2a} = \mp \dfrac{\sqrt{\Delta}}{2a} \qquad \rightarrow \qquad 
    x_{1,2} = \dfrac{-b  \mp \sqrt{\Delta}}{2a} \ . 
\end{equation}

\subsection{Equazioni algebriche razionali}
\begin{definition}[Equazioni algebriche razionali] Le equazioni algebriche razionali sono equazioni che contengono polinomi, loro rapporti e potenze intere.
\end{definition}
\begin{example}[Esempi di equazioni algebriche razionali]
\begin{equation}
    \dfrac{(x + 3)^2}{(x-1)} = 4 x 
\end{equation}
\begin{equation}
    \dfrac{2x}{x^2+1} = -\dfrac{1}{x}
\end{equation}
\end{example}

\subsubsection{Metodo di soluzione}
\begin{enumerate}
    \item Per prima cosa è necessario determinare le \textbf{condizioni di esistenza} di una soluzione. Poiché nelle equazioni può comparire la \textbf{divisione} tra polinomi, bisogna richiedere che questa e tutte le operazioni scritte nel problema abbiano senso: ad esempio, nelle condizioni di esistenza bisogna richiedere che non avvengano divisioni per zero.
    \item Successivamente, è possibile procedere con le semplificazioni per la ricerca della soluzione.
\end{enumerate}

\paragraph{Esempi}
Seguendo questo metodo di soluzione, procediamo a risolvere le equazioni dell'esempio {\color{red} cit}.
\begin{equation}
    \dfrac{(x + 3)^2}{(x-1)} = 4 x  \qquad \qquad \text{C.E.: } x \ne 1 \quad \rightarrow \quad x \in \mathbb{R}\backslash \{1\}
\end{equation}
\begin{equation}
    \begin{aligned}
        & (x + 3)^2 = 4x(x-1) \\
        & x^2+6x+9 = 4x^2 -4x \\
        & 3x^2-10x-9 = 0 \\
        & \rightarrow \quad x_{1,2} = \dfrac{5 \mp \sqrt{25+3\cdot9}}{6} = \dfrac{5 \mp \sqrt{52}}{3}
    \end{aligned}
\end{equation}
\begin{equation}
    \dfrac{2x}{x^2+1} = -\dfrac{1}{x} \qquad \qquad \text{C.E.: } x \ne 0 \quad \rightarrow \quad x \in \mathbb{R}\backslash \{0\}
\end{equation}
\begin{equation}
\begin{aligned}
    & 2x^2= - x^2 - 1 \\
    & 3x^2= - 1 \quad \rightarrow \quad \nexists x \in \mathbb{R}
\end{aligned}
\end{equation}

\subsection{Equazioni algebriche irrazionali}
\begin{definition}[Equazioni algebriche irrazionali] Le equazioni algebriche razionali sono equazioni che contengono polinomi, loro rapporti e potenze intere e non.
\end{definition}
\begin{example}[Esempi di equazioni algebriche irrazionali]
\begin{equation}
    \sqrt[3]{x-3} = 2 \qquad , \qquad \dfrac{2 x}{(x-1)^{\frac{1}{2}}} = - 2
\end{equation}
\end{example}
\subsubsection{Metodo di soluzione}
\begin{enumerate}
    \item Per prima cosa è necessario determinare le \textbf{condizioni di esistenza} di una soluzione. Bisogna richiedere che questa e tutte le operazioni scritte nel problema abbiano senso: bisogna richiedere che
        \begin{itemize}
            \item che non avvengano \textbf{divisioni} per zero;
            \item che siano non negativi i radicandi di eventuali \textbf{radici} con indice intero pari o non intero.
        \end{itemize}
    \item Successivamente, è possibile procedere con le semplificazioni per la ricerca della soluzione.
\end{enumerate}

\paragraph{Esempi}
\begin{equation}
    \sqrt[3]{x-3} = 2 \qquad \qquad \text{C.E.: } x \in \mathbb{R}
\end{equation}
\begin{equation}
    x-3 = 8 \qquad \rightarrow \qquad x = 11
\end{equation}
\begin{equation}
    \dfrac{2 x}{(x-1)^{\frac{1}{2}}} = - 2 \qquad \qquad \text{C.E.: } x-1>0 \quad \rightarrow \quad x \in (1,+\infty)
\end{equation}
\begin{equation}
\begin{aligned}
    &  x = -  (x-1)^{\frac{1}{2}} \\
    &  x^2 =   x-1  \\
    & x^2 - x + 1 = 0 \\
    & \Delta = (-1)^2 - 4 \cdot 1 \cdot 1 = -3 < 0 \qquad \rightarrow \qquad \nexists x \in \mathbb{R}
\end{aligned}
\end{equation}

\section{Equazioni non algebriche o trascendenti}
\subsection{Equazioni con i valori assoluti}
\subsection{Equazioni con esponenti e logaritmi}
\subsection{Equazioni con le funzioni armoniche}

\section{Metodi di soluzione approssimati}
\subsection{Metodo grafico}
\subsection{Metodi numerici}
{\color{red} Riferimento al capitolo dei metodi numerici}

% ==============================================================================
\chapter{Disequazioni}
\section{Disequazioni algebriche}
\section{Disequazioni non algebriche}

% ==============================================================================
\chapter{Sistemi di equazioni e di disequazioni}


