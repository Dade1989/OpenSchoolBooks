
% ==============================================================================
\chapter{Algebra complessa}\label{ch:complex-algebra}
\section{Definizione dei numeri complessi}
\begin{definition}[Unità immaginaria] 
 \begin{equation}
   i := \sqrt{-1}
 \end{equation}
\end{definition}

\begin{definition}[Numero complesso]
\begin{equation}
  z = x + i y \qquad , \qquad x,y \in \mathbb{R}
\end{equation}
\end{definition}


\subsection{Rappresentazione del piano complesso (di Argand--Gauss)}
Si può definire una relazione biunivoca tra l'insieme dei numeri complessi $\mathbb{C}$ e il piano $\mathbb{R}^2$.

\subsubsection{Rappresentazione cartesiana.}

\subsubsection{Rappresentazione polare.}
Trasformazione tra coordinate cartesiane e polari
\begin{equation}
\begin{cases}
  x = r \cos \theta \\
  y = r \sin \theta
\end{cases} \qquad , \qquad
\begin{cases}
  r = \sqrt{x^2 + y^2} \\
  \theta = \text{atan2}(x,y)
\end{cases}
\end{equation}
e quindi
\begin{equation}
  z = x + i y = r \left( \cos \theta + i \sin \theta \right)
\end{equation}

\paragraph{La relazione di Eulero e la rappresentazione polare dei numeri complessi.} Usando le espansioni in serie di Taylor delle funzioni $e^{i\theta}$, $\cos \theta$ e $\sin \theta$, Eulero ricavò la formula che da lui prende il nome
\begin{equation}
  e^{i \theta} = \cos \theta + i \sin \theta \ .
\end{equation}
\begin{proof}[Dimostrazione con le serie di Taylor]
\begin{equation}
\begin{array}{rll}
   \cos \theta  & = \sum_{n=0}^{\infty} (-1)^n \dfrac{\theta^{2n}}{(2n)!} & = 1 - \dfrac{\theta^2}{2} + \dfrac{\theta^4}{4!} + \dots \\
   \sin \theta  & = \sum_{n=0}^{\infty} (-1)^n \dfrac{\theta^{2n+1}}{(2n+1)!} & = \theta - \dfrac{\theta^3}{3!} + \dfrac{\theta^5}{5!} + \dots \\
   e^{i \theta} & = \sum_{n=0}^{\infty} \dfrac{(i\theta)^n}{n!} & = 1 + i\theta - \dfrac{\theta^2}{2} - i \dfrac{\theta^3}{3!} + \dfrac{\theta^4}{4!} + i \dfrac{\theta^5}{5!} +  \dots = \\
  & & = \left[  1 - \dfrac{\theta^2}{2} + \dfrac{\theta^4}{4!} + \dots \right] + i \left[ \theta - \dfrac{\theta^3}{3!} + \dfrac{\theta^5}{5!} + \dots \right]  = \\
  & & = \cos \theta + i \sin \theta \ .
\end{array}
\end{equation}
\end{proof}
%
\begin{proof}[Dimostrazione con la derivata] Si definisce la funzione con variabile reale $\theta \in \mathbb{R}$ e valori complessi $f: \mathbb{R} \rightarrow \mathbb{C}$, $f(\theta) = \cos \theta + i \sin \theta$, e si calcola la derivata rispetto alla variabile indipedente $\theta$
\begin{equation}
    \dfrac{d}{d \theta} \left( \cos \theta + i \sin \theta \right) = -\sin \theta + i \cos \theta = i \left( \cos \theta + i \sin \theta \right)
\end{equation}
    La funzione $f(\theta) = \cos \theta + i \sin \theta$ soddisfa quindi l'equazione differenziale
\begin{equation}
    \dfrac{d}{d \theta} f(\theta) = i f(\theta) \ ,
\end{equation}
    con condizione $f(0) = 1$. Essendo questo un problema lineare, se esiste, esiste una sola soluazione. Si può dimostrare che questa equazione è soddisfatta anche dalla funzione a variabile reale e valori complessi $g: \mathbb{R} \rightarrow \mathbb{C}$, $g(\theta) = e^{i\theta}$, la cui derivata è $g'(\theta) = i e^{i\theta} = i g(\theta)$.

    \noindent
    Possiamo quindi concludere l'uguaglianza tra le due funzioni $g(\theta) = f(\theta)$, $e^{i \theta} = \cos \theta + i \sin \theta$.
\end{proof}

% ------------------------------------------------------------------------------
\section{Operazioni con i numeri complessi}
In questa sezione vengono presentate le operazioni algebriche con i numeri complessi, osservando come convenga la forma cartesiana per somma e differenza, mentre la forma polare per prodotti, divisioni e soprattutto potenze e radici.

\subsection{Somma e differenza}
\begin{equation}
   z_1 + z_2 = x_1 + x_2 + i (y_1 + y_2) \ .
\end{equation}

\subsection{Prodotto e divisione}
\begin{equation}
   z_1 z_2 = r_1 r_2 e^{i(\theta_1+\theta_2)} \ .
\end{equation}

\subsection{Potenze e radici}
\begin{equation}
  z^n = \left( r e^{i\theta} \right)^n = r^n e^{i n\theta}
\end{equation}
\begin{equation}
  z^{\frac{1}{n}} = \left( r e^{i(\theta + 2\pi m)} \right)^{\frac{1}{n}} = r^{\frac{1}{n}} e^{i\left(\frac{\theta}{n} + \frac{m}{n} 2 \pi \right)}
\end{equation}

\subsection{Esponenziali e logaritmi}
\begin{equation}
\begin{aligned}
    e^z
    & = e^{x+i y} = e^x \, e^{i y} = e^x \left( \cos y + i \sin y \right) \\
    & = e^{r e^{i\theta}} = e^{r \left( \cos \theta + i \sin \theta \right)} = e^{r \cos \theta} e^{i \, r \, \sin \theta} = e^{r \cos \theta} \left[ \cos(r \sin \theta) + i \sin (r \sin \theta) \right]
\end{aligned}
\end{equation}
%
\begin{equation}
    \ln z = \ln \left( r e^{i \theta} \right) = \ln r + \ln e^{i \theta} = \ln r + i \, \theta
\end{equation}

% ==============================================================================
\chapter{Calcolo complesso}\label{ch:complex-calculus}
