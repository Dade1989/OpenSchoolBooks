
% ==============================================================================
\chapter{Variabili casuali}
% ------------------------------------------------------------------------------
\begin{definition}[$\sigma$-algebra, $\mathcal{F}$]
\end{definition}
\begin{definition}[Misura di probabilità, $\nu$]
\end{definition}
\begin{definition}[Spazio misurabile, $(\Omega, \mathcal{F})$]
\end{definition}
\begin{definition}[Spazio di probabilità, $(\Omega, \mathcal{F}, \nu)$]
\end{definition}
\begin{definition}[Variabile casuale, $X:(\Omega, \mathcal{F}, \nu) \rightarrow (E, \mathcal{E})$]
    Una variabile casuale può essere definita come una funzione $X: \Omega \rightarrow E$, che ha come:
    \begin{itemize}
        \item dominio l'insieme degli eventi (o spazio campionario), $\Omega$,
        \item codominio lo spazio dei risultati o delle osservazioni, $E$
    \end{itemize}
\end{definition}
% ------------------------------------------------------------------------------
\section{Statistica univariata}
% ..............................................................................
\subsection{Variabili casuali, discrete e continue}
% ------------------------------------------------------------------------------
\begin{definition}[Variabile casuale discreta] Una variabile aleatoria è discreta, se l'insieme dei suoi possibili valori (discreti) è finito o numerabile, cioé può essere messo in corrispondenza biunivoca con i numeri naturali.
\end{definition}
\begin{example} La variabile casuale che descrive il lancio di un dado a sei facce può i valori $\{1,2,3,4,5,6\}$ e quindi è una variabile casuale discreta.
\end{example}
\begin{example} Una variabile casuale che ha come possibili valori i numeri interi $\mathbb{Z} = \{\dots,-2,-1,0,1,2,\dots\}$ è una variabile discreta, poiché l'insieme dei suoi valori è numerabile.
\end{example}
\begin{definition}[Variabile casuale continua] Una variabile aleatoria è continua, se l'insieme dei suoi possibili valori ha la potenza del continuo, come i numeri reali.
\end{definition}
\begin{example} Una variabile casuale che rappresenta la distanza di un lancio di freccia dal centro di un bersaglio è una variabile casuale continua, poiché può assumere tutti i valori reali non nulli, $X(\omega) \in \mathbb{R}^+$.
\end{example}
\begin{example} Una variabile casuale che ha come possibili valori i numeri reali dell'intervallo limitato $E = [0, 1] \subset \mathbb{R}$ è una variabile continua.
\end{example}
% ------------------------------------------------------------------------------
\subsection{Funzioni di probabilità}
% ------------------------------------------------------------------------------
\begin{definition}[Funzione cumulativa di probabilità] La funzione cumulativa di probabilità valuta la probabilità totale che un evento $\omega$ abbia un valore osservato $X(\omega)$ appartenente a un sottoinsieme del codominio, $A \subseteq E$
\begin{equation}
    P_X(A) = \nu \{\omega \in \Omega: X(\omega) \in A \} = P(X \in A)
\end{equation}
\end{definition}
\begin{definition}[Distribuzione di probabilità -- variabile discreta] Per una variabile discreta $X$, la funzione distribuzione di probabilità rappresenta la probabilità dei singoli valori osservabili,
    \begin{equation}
        p(x) = P(X = x)
    \end{equation}
\end{definition}
\begin{definition}[Distribuzione di probabilità -- variabile continua] Per una variabile continua $X$, la funzione cumulativa $P(X \in A)$ può essere rappresentata come l'integrale sull'insieme $A$ della funzione distribuzione di probabilità $p(x)$,
    \begin{equation}
       P(X\in A) = \int_{A} p(x)
    \end{equation}
\end{definition}

\paragraph{Proprietà delle funzioni di probabilità.}
\begin{itemize}
    \item \textbf{Non-negatività}: la probabilità di un evento è non negativa. Da questo segue che la distribuzione di probabilità è non negativa,
        \begin{equation}
            p(x) \ge 0 \ .
        \end{equation}
    \item \textbf{Unitarietà}: la probabilità cumulativa su tutti i possibili valori che può assumere la variabile casuale è unitaria, 100\%. Questo risultato implica che non si possono verificare eventi che implichino un valore della variabile casuale al di fuori del suo codominio;
        \begin{itemize}
            \item per una variabile discreta:
                \begin{equation}
                    \sum_{k} p(x_k) = 1
                \end{equation}
            \item per una variabile continua:
                \begin{equation}
                    \int_E p(x) dx = 1
                \end{equation}
        \end{itemize}
\end{itemize}
% ..............................................................................
\subsection{Indicatori statistici}
% ..............................................................................

% ------------------------------------------------------------------------------
\section{Statistica multivariata}
% ------------------------------------------------------------------------------
\subsection{Variabili casuali discrete e continue}
% ..............................................................................

% ..............................................................................
\subsection{Funzioni di probabilità}
% ..............................................................................
\subsection{Teorema di Bayes}\label{ch:bayes-thm}
% ..............................................................................
\subsection{Momenti di una distribuzione}
% ..............................................................................
Dopo aver raccolto le variabili casuali scalari $X_i$ in una variabile casuale vettoriale $\mathbf{X}$, usando il formalismo matriciale
\begin{equation}
    \mathbf{X} = \begin{bmatrix}  X_1 \\ \dots  \\ X_n \end{bmatrix}
\end{equation}
possiamo definire alcuni indicatori sintetici. 

% \subsubsection{Valore atteso (volgarmente chiamato media)}
\begin{definition}[Valore atteso (volgarmente chiamato media)]
Il valore atteso di una variabile casuale multivariata (o multidimensionale) viene definita come la media pesata di tutti i possibili valori $\mathbf{x}_I$ della variabile casuale $\mathbf{X}_I$, pesati per il valore corrispondente della densità di probabilità
\begin{equation}
    \mathbb{E}[\mathbf{x}] := \overline{\mathbf{X}} := \mathbf{\mu}_X = \sum_{I} f(\mathbf{x}_I) \mathbf{x}_I \ . 
\end{equation}
\end{definition}

%\subsubsection{Covarianza}
\begin{definition}[Covarianza]
La covarianza viene definita come il valore atteso del ``prodotto tensoriale'' della deviazione della media con sé stesso, cioé
\begin{equation}
    \mathbf{C}_{XX} := \mathbb{E}[(\mathbf{X} - \overline{\mathbf{X}})(\mathbf{X} - \overline{\mathbf{X}})^T] \  .
\end{equation}
\end{definition}

Usando le proprietà della media, si può riscrivere la covarianza come
\begin{equation}
\begin{aligned}
    \mathbf{C}_{XX} & = \mathbb{E}[(\mathbf{X} - \overline{\mathbf{X}})(\mathbf{X} - \overline{\mathbf{X}})^T] = \\
    & = \mathbb{E}[\mathbf{X} \mathbf{X}^T] - \mathbf{E}[\mathbf{X}\overline{\mathbf{X}}^T] - \mathbf{E}[\overline{\mathbf{X}}\mathbf{X}^T] + \overline{\mathbf{X}} \, \overline{\mathbf{X}}^T = \\
    & = \mathbb{E}[\mathbf{X} \mathbf{X}^T] - \mathbf{E}[\mathbf{X}]\overline{\mathbf{X}}^T - \overline{\mathbf{X}}\mathbf{E}[\mathbf{X}^T] + \overline{\mathbf{X}} \, \overline{\mathbf{X}}^T = \\
    & = \mathbb{E}[\mathbf{X} \mathbf{X}^T] - \mathbf{E}[\mathbf{X}]\overline{\mathbf{X}}^T - \overline{\mathbf{X}}\mathbf{E}[\mathbf{X}^T] + \overline{\mathbf{X}} \, \overline{\mathbf{X}}^T = \\
    & = \mathbb{E}[\mathbf{X} \mathbf{X}^T] - \overline{\mathbf{X}} \, \overline{\mathbf{X}}^T 
\end{aligned}
\end{equation}



% ==============================================================================
\chapter{Processi casuali}
% ------------------------------------------------------------------------------


% ==============================================================================
\chapter{Approcci alla statistica}
% ------------------------------------------------------------------------------
\section{Statistica descrittiva}
% ------------------------------------------------------------------------------
\begin{definition}[Statistica descrittiva]
La statistica descrittiva si occupa principalmente di una \textbf{rappresentazione riassuntiva}, tramite indicatori statistici o grafici.
La statistica descrittiva non si preoccupa di costruire un modello del fenomeno che si sta osservando, ma piuttosto di rappresentare i dati in forma non parametrica.
\end{definition}

Di solito, un approccio descrittivo al problema costituisce una fase preliminare a un approccio inferenziale: prima di scegliere un tipo di modello adatto a rappresentare il problema, è meglio dare un'occhiata ai dati disponibili. \vspace{10pt}

\paragraph{Variabili aleatorie univariate.} La rappresentazione descrittiva delle osservazioni di una variabile casuale prevede:
\begin{itemize}
  \item la rappresentazione grafica della distribuzione della probabilità (istogrammi,...)
  \item indicatori statistici sintetici di:
  \begin{itemize}
    \item di tendenza: media, mediana, moda
    \item di dispersione: varianza, deviazione standard, intervalli e quartili
    \item di forma: simmetria (skewness), curtosi (kurtosis)
  \end{itemize}
\end{itemize}
\paragraph{Variabili aleatorie multivariate.} La rappresentazione simultanea di più variabili casuali prevede anch'essa
\begin{itemize}
  \item la rappresentazione grafica delle probabilità congiunte, condizionali o marginali, in forma tabulare o grafica
  \item indicatori statistici sintetici
  \begin{itemize}
    \item indicatori di tendenza, dispersione e forma
    \item indicatori di relazione tra le variabili: dipendenza, correlazione e covarianza
  \end{itemize}
\end{itemize}

% ------------------------------------------------------------------------------
\section{Statistica inferenziale}
% ------------------------------------------------------------------------------
\begin{definition}[Statistica inferenziale] L'approccio inferenziale alla statistica prevede l'uso dei dati per la \textbf{costruzione di un modello} del fenomeno osservato, che permetta di formulare delle proposizioni sul fenomeno osservato, come:
\begin{itemize}
    \item la stima di valori, con un certo intervallo di confidenza, ad esempio tramite regressione
    \item la classificazione, o il raggruppamento di osservazioni in gruppi
    \item la conferma o la confutazione di un'ipotesi
\end{itemize}
\end{definition}

\subsection{Modelli statistici}
Un modello statistico del fenomeno osservato spesso si riduce a un modello matematico della distribuzione di probabilità, e il problema di allenamento/taratura del modello si riduce a un problema di approssimazione di una funzione. A seconda del numero di parametri liberi (gradi di libertà) del modello da tarare, e dalla rigidità delle ipotesi sul modello si possono distinguere:
\begin{itemize}
    \item \textbf{modelli parametrici}: modelli costruiti su ipotesi che possono essere stringenti, e che hanno un numero finito di parametri liberi di cui calcolare il valore. Ad esempio, si può assumere che la funzione densità di probabilità possa essere stimata con una combinazione di funzioni appartenenti a una famiglia di funzioni. Il valore dei parametri del modello viene calcolato risolvendo il problema di approssimazione di una funzione, che meglio rappresenti i dati disponibili. Esempi molto comuni di questo tipo di modelli sono:
    \begin{itemize}
        \item i modelli lineari generalizzati
        \item le reti neurali, che permettono una combinazione non lineare di funzioni
    \end{itemize}
    \item \textbf{modelli non parametrici}: modelli costruiti senza alcuna ipotesi particolare, che spesso si basano su stimatori non parametrici dei valori di tendenza e dispersione
    \item \textbf{modelli semi-parametrici}: una via di mezzo tra i due
\end{itemize}

{\color{red}
\vspace{10pt}
\noindent
Quando si progetta e si calcolano i parametri di un modello, bisogna prestare attenzione a:
\begin{itemize}
    \item modello sufficientemente generale da poter rappresentare il fenomeno
    \item modello che permetta degli adeguati livelli di \textbf{accuratezza} e \textbf{generalizzazione}: \textbf{tradeoff bias-variance}
    \item valutazione della bontà del modello
\end{itemize}
}


% ==============================================================================
\chapter{Esempi di applicazioni}
% ------------------------------------------------------------------------------

% ==============================================================================
\chapter{Introduzione all'intelligenza artificiale}
% ------------------------------------------------------------------------------


