

% ==============================================================================
\chapter{Successioni e serie di numeri reali}\label{ch:series}
% ------------------------------------------------------------------------------
\section{Successioni}
\begin{definition}[Serie di numeri reali]
\end{definition}

% ------------------------------------------------------------------------------
\section{Serie}
\begin{definition}[Serie di numeri reali]
\end{definition}
\begin{definition}[Serie infinite di numeri reali]
\end{definition}
\begin{definition}[Carattere di una serie] Una serie può essere
\begin{itemize}
    \item convergente
    \item divergente
    \item indeterminata
\end{itemize}
\end{definition}

\subsection{Serie notevoli}
\subsubsection{Serie armonica}
\begin{definition}[Serie armonica]
\begin{equation}
    \sum_{k=0}^{+\infty} \dfrac{1}{k}
\end{equation}
\end{definition}
\subsubsection{Serie geometrica}
\begin{definition}[Serie geometrica]
\begin{equation}
    \sum_{k=0}^{+\infty} \dfrac{1}{a^k}
\end{equation}
\end{definition}
\subsubsection{La costante di Napier}\label{sec:napier}
\begin{definition}[La costante di Napier, o il numero di Euler -- $e$]
La successione di numeri $\{ e_n \}$,
\begin{equation}
    e_n = \sum_{k=0}^{n} \frac{1}{k!} \ ,
\end{equation}
è una serie monotona crescente e superiormente limitata, e quindi esiste il limite finito il cui valore viene definito costante di Napier, o numero di Eulero
\begin{equation}
   e := \lim_{n \rightarrow +\infty} e_n = \sum_{k=0}^{+\infty} \frac{1}{k!} \ . 
\end{equation}
\end{definition}
\begin{proof}
\'E immediato riconoscere che la successione $e_n$ è crescente, essendo definito come una somma di numeri positivi,
\begin{equation}
    e_{n+1} = \sum_{k=0}^{n+1} \dfrac{1}{k!} = \sum_{k=0}^{n} \dfrac{1}{k!} + \dfrac{1}{(n+1)!} = e_n + \dfrac{1}{(n+1)!} > e_n \ .
\end{equation}
Per dimostrare che è limitata superiormente, si può osservare che ogni termine della serie è minore del termine rispettivo della serie armonica di base $\frac{1}{2}$
\begin{equation}
\begin{aligned}
    e_{n}
    & = 1 + 1 + \dfrac{1}{2} + \dfrac{1}{3!} + \dfrac{1}{4!}  + \dfrac{1}{5!}  + \dots < \\
    & < 1 + 1 + \dfrac{1}{2} + \dfrac{1}{2^2}+ \dfrac{1}{2^3} + \dfrac{1}{2^4} + \dots = \\
    & = 1 + \sum_{k=0}^{n} \dfrac{1}{2^k} = 1 + S_n\left(\frac{1}{2}\right) 
      = 1 + \dfrac{1 - \left(\frac{1}{2}\right)^{n+1}}{1 - \frac{1}{2}} \ ,
\end{aligned}
\end{equation}
e per il limite $n \rightarrow +\infty$, si ottiene
\begin{equation}
    e_{\infty} = \lim_{n \rightarrow +\infty} e_n = \sum_{k=0}^{+\infty} \dfrac{1}{k!} < 1 + S\left(\dfrac{1}{2}\right) = 1 + 2 = 3 \ .
\end{equation}
\end{proof}
\begin{definition}[Definizioni equivalenti di $e$]
\begin{equation}
    e = 
    \begin{cases}
        \displaystyle\sum_{k=0}^{+\infty} \dfrac{1}{k!} \\
        \displaystyle\lim_{n \rightarrow +\infty} \left( 1 + \dfrac{1}{n} \right)^n \\ 
    \end{cases}
\end{equation}
\end{definition}

Si vuole dimostrare che le due definizioni sono tra di loro equivalenti, confrontando i valori della serie troncata e della funzione $\left(1 + \frac{1}{n}\right)^n$
\begin{equation}
\begin{aligned}
    s_n & := \sum_{k=0}^{n} \dfrac{1}{k!} \\
    t_n & := \left( 1+ \frac{1}{n}  \right)^n
\end{aligned}
\end{equation}
\begin{proof}[$t_n \le s_n$] Usando il teorema binomiale
\begin{equation}
\begin{aligned}
    t_n = \left( 1 + \dfrac{1}{n} \right)^n 
    & = \sum_{k=0}^{n} \binom{n}{k} 1^{n-k} \left(\dfrac{1}{n} \right)^{k} = \\
    & = 1 + n \, \dfrac{1}{n} + \dfrac{n(n-1)}{2} \, \dfrac{1}{n^2} +  \dfrac{n(n-1)(n-2)}{2 \cdot 3} \, \dfrac{1}{n^3} + \dots = \\
    & = 1 + 1 + \frac{1}{2} \left( 1 - \frac{1}{n} \right) +  \dfrac{1}{3!} \left( 1-\frac{1}{n} \right) \left( 1-\frac{2}{n} \right) + \dots = \\
    & < 1 + 1 + \dfrac{1}{2} + \dfrac{1}{3!} + \dots = \sum_{k=0}^{n} \frac{1}{k!} = s_n
\end{aligned}
\end{equation}
\end{proof}
\begin{proof}[$t_n \ge s_n$]
Si prende un numero $m < n$ fissato, e si definisce
{\small
\begin{equation}
\begin{aligned}
    t_{n,m} & := \left( 1 + \dfrac{1}{n} \right)^m \\
    & = \sum_{k=0}^{m} \binom{m}{k} 1^{n-k} \left(\dfrac{1}{n} \right)^{k} = \\
    & = 1 + 1 + \frac{1}{2} \left( 1 - \frac{1}{n} \right) +  \dfrac{1}{3!} \left( 1-\frac{1}{n} \right) \left( 1-\frac{2}{n} \right) + \dots \\
    & \quad \dots + \dfrac{1}{m!}\left(1-\frac{1}{n}\right) \dots \left(1-\frac{m-1}{n}\right) = \\
    & < t_n \\
    s_m & = 1 + 1 + \dfrac{1}{2} + \dots + \dfrac{1}{m!}
\end{aligned}
\end{equation}
}
Fissando $m$ e facendo tendere all'infinito $n$, $n \rightarrow \infty$
\begin{equation}
    \lim_{n \rightarrow \infty} t_n > \lim_{n \rightarrow \infty} t_{n,m} = 1 + 1 + \dfrac{1}{2} + \dots \dfrac{1}{m!} = \sum_{k=0}^{m} \dfrac{1}{k!} = s_m
\end{equation}
\end{proof}

\begin{proof}[Conclusioni]
Sapendo che
    \begin{equation}
        \begin{cases}
            s_n > t_n \\
            s_n < \lim_{m \rightarrow \infty} t_m
        \end{cases} \qquad \rightarrow \qquad
        \begin{cases}
            \text{limsup}_{n\rightarrow\infty} s_n \ge \text{limsup}_{n\rightarrow\infty} t_n \\
            s_n \le \text{liminf}_{n\rightarrow\infty} t_n \\
        \end{cases}
    \end{equation}
e ricordando che $\limsup \ge \liminf$ si può scrivere
\begin{equation}
    \limsup_{n\rightarrow\infty} t_n \ge \liminf_{n\rightarrow\infty} t_n \ge \limsup_{n\rightarrow\infty} s_n \ge \limsup_{n\rightarrow\infty} t_n \ .
\end{equation}
\end{proof}

% ==============================================================================
\chapter{Serie di funzioni}
% ------------------------------------------------------------------------------
\section{Serie di potenze}
\subsection{Serie di Taylor}

% ------------------------------------------------------------------------------
\section{Serie di Fourier}\label{ch:fourier-series}

% ==============================================================================
