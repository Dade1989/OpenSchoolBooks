
\chapter{Strumenti e discipline in matematica}

{\color{red}
\section{Quantità}
\begin{itemize}
    \item numeri - scalari ($\mathbb{N}$, \dots, $\mathbb{R}$, \dots)
    \item vettori
    \item tensori
    \item quaternioni, e altri oggetti matematici esotici
\end{itemize}
}

Sviluppo della matematica:
\begin{itemize}
    \item saper contare: aritmetica
    \item saper descrivere lo spazio: geometria
\end{itemize}


\section{Classificazione}
Nel corso della storia, la matematica è diventata una materia estremamente diversificata. 

Una classificazione delle molte aree di interesse della matematica può risultare utile ad avere una visione di insieme della materia. 

La matematica è divisa tradizionalmente in \textbf{matematica pura}, che studia gli strumenti propri della matematica e le loro proprietà, e \textbf{matematica applicata}, che applica gli strumenti della matematica a problemi di interesse tipici di altre discipline.

Una classificazione più dettagliata delle discipline della matematica è resa difficile dalle relazioni esistenti tra di esse. Viene qui presentata una classificazione parziale, ispirata alla \textit{Classificazione delle ricerche matematiche (MSC)} usata dall'\textit{American Mathematical Society} per la classificazione delle pubblicazioni nei database.

\subsection{Matematica pura}
\subsubsection{Matematica generale e fondamenti}
\begin{itemize}
    \item storia e filosofia
    \item logica
\end{itemize}
\subsubsection{Algebra e matematica discreta}
\begin{itemize}
    \item teoria dei numeri
    \item teoria dei campi
    \item algebra lineare e multilineare; matrici
\end{itemize}

\subsubsection{Analisi}
\begin{itemize}
    \item funzioni reali e complesse
    \item misura e integrazione
    \item equazioni differenziali, ed equazioni alle differenze
    \item sistemi dinamici
    \item successioni, serie e approssimazioni
    \item trasformate: analisi di Fourier
    \item calcolo delle variazioni, ottimizzazione e controllo
\end{itemize}

\subsubsection{Geometria e topologia}
\begin{itemize}
    \item geometria
    \item geometria differenziale
    \item topologia
\end{itemize}

\subsection{Matematica applicata}
\subsubsection{Statistica}
\subsubsection{Fisica}
\subsubsection{Altre scienze naturali}
\subsubsection{Scienze sociali e del comportamento}
\subsubsection{Matematica numerica e informatica}
\subsubsection{Ottimizzazione}
L'ottimizzazione si occupa della ricerca dei punti di massimo o di minimo di una funzione $f:A \rightarrow \mathbb{R}$.
\begin{equation}
    \text{Determinare il punto $\mathbf{x}_0 \in A$ tale che $f(\mathbf{x_0}) \ge f(\mathbf{x}) \ \forall \mathbf{x} \in A$}
\end{equation}
\subsubsection{Teoria dei sistemi e controllo}
La teoria dei sistemi rappresenta un qualunque sistema che evolve nel tempo $t$ con:
\begin{itemize}
    \item gli ingressi $\mathbf{u}(t)$, che possono agire sul sistema
    \item le uscite $\mathbf{y}(t)$, che possono essere lette dal sistema
    \item le variabili di stato, $\mathbf{x}(t)$ che determinano lo stato del sistema
    \item le equazioni di stato del sistema che ne governano la dinamica, legando lo stato e la sua evoluzione agli ingressi
        \begin{equation}
            \mathbf{f}(\dot{\mathbf{x}}, \mathbf{x}, \mathbf{u}) = \mathbf{0}
        \end{equation}
    \item le equazioni di uscita che permettono di ricavare le variabili di uscita in funzione dello stato e degli ingressi
        \begin{equation}
            \mathbf{y} = \mathbf{g}(\mathbf{x}, \mathbf{u})
        \end{equation}
\end{itemize}
Il problema del controllo di un sistema può essere formulato come un problema di ottimizzazione vincolata, volendone ottimizzare le prestazioni (o minimizzare le perdite), soggette ai vincoli definiti dalle equazioni che governano il sistema.
\subsubsection{Didattica}



\chapter{Approccio alla matematica}
\section{Formulazione e soluzione di un problema}
Una volta formulato un problema, ci si chiede:
\begin{itemize}
    \item il problema ammette soluzione?
    \item se il problema ammette soluzione, la soluzione è unica?
    \item se la soluzione non è unica, quante soluzioni esistono?
\end{itemize}

\paragraph{Esempi}


\chapter{Breve storia della matematica}
\paragraph{Preistoria}
\begin{itemize}
  \item \dots
\end{itemize}
\paragraph{Antichità e Medioevo}
\begin{itemize}
  \item \textbf{Egitto}
  \item \textbf{Babilonia}
  \item \textbf{Grecia}
  \item \textbf{India}
      \begin{itemize}
          \item invenzione del \textbf{sistema numerico posizionale decimale}: tra le dieci cifre, c'è lo \textbf{zero}
          \item vengono introdotte le \textbf{funzioni trigonometriche} e applicate nell'ambito dell'astronomia: nel calcolo delle traiettorie introducono concetti simili a quelli di derivata, senza però mai riuscire a sviluppare una teoria del calcolo infinitesimale
      \end{itemize}
  \item \textbf{Cina}
  \item \textbf{Persia e Islam}
      \begin{itemize}
          \item mondo islamico ponte tra la cultura ellenistica e quella indiana
          \item nel IX secolo, il matematico persiano \textbf{al-Khwarizmi} usa il sistema di numerazione posizionale decimale appreso dagli indiani per descrivere metodi grafici e analitici per la risoluzione delle equazioni di secondo grado. Dal suo nome deriva la parola \textbf{algoritmo}, e dalla sua opera principale la parola \textbf{algebra}
          \item sviluppi in trigonometria
          \item nel XIV secolo, il matematico persiano \textbf{Al-Kashi} inventa il modo per calcolare una radice di un polinomio di grado $n$, oggi conosciuta come regola di Ruffini, italiano che la riscoprì nel XVIII secolo; fornisce la prima \textbf{dimostrazione per induzione} nota a oggi, con la quale dimostra il \textbf{teorema binomiale}
          \item lo sviluppo della matematica si interrompe nel XIV secolo, in un periodo di instabilità politica e religiosa
      \end{itemize}
  \item \textbf{Europa}
      \begin{itemize}
          \item verso il XII secolo inizia un nuovo periodo di sviluppo in Europa; ritorna interesse sulla matematica, stimolato da problemi pratici e commerciali
          \item studio delle \textbf{serie infinite}, e introduzione alcuni concetti tra i concetti tra i quali il \textbf{grafico di una funzione} (Fibonacci, Oresme)
          \item studio della prospettiva e geometria, stimolati da motivi artistici
      \end{itemize}
\end{itemize}
\paragraph{XVI secolo}
\begin{itemize}
    \item interesse per l'algebra e le soluzioni delle equazioni polinomiali di terzo (disputa da Tartaglia e Cardano, che pubblica l'\textit{Ars Magna}) e quarto grado (Ferrari)
    \item Viète produce contributi alla trigonometria e all'algebra, trovando le \textbf{formule di Viète} che legano i coefficienti alle radici di un'equazione
    \item Nepier (Nepero) introduce i \textbf{logaritmi}
    \item evoluzione della notazione matematica: vengono introdotti i segni ancora utilizzati per le operazioni (+, -, $\times$), il segno di uguale (=), e maggiore e minore (>, <); Viète introduce il \textbf{calcolo letterale}, usando delle lettere per indicare i coefficienti delle equazioni
\end{itemize}
\paragraph{XVII secolo}
\begin{itemize}
  \item in Europa vengono fondate accademie e associazioni scientifiche, come l'Accademia in Francia e la Royal Society in Inghilterra; nelle università vengono istituite le prime cattedre di matematica
  \item Fermat e Descartes (Cartesio), nella sua \textit{La Gèometrie}, formulano e illustrano i fondamenti della \textbf{geometria analitica}, Cap.\S\ref{ch:geometry-analytical};
  \item Fermat porta contributi alla \textbf{teoria dei numeri}, formulando una gran quantità di congetture, in gran parte dimostrate da Euler pochi decenni dopo. L'ultima congettura, nota come \textit{ultimo teorema di Fermat}, venne dimostrata solo nel 1995: di questa congettura, Fermat sostenne di conoscere la dimostrazione, ma di non aver avuto sufficiente spazio sul foglio per scriverla.
  \item Fermat e Pascal portano sviluppi anche al calcolo delle probabilità e delle combinazioni.
  \item i contributi di Wallis, Mengoli e Mercator su \textbf{serie e prodotti infiniti}, Cap.\S\ref{ch:series}, introducono tecniche simili a quelle che verranno sviluppate nel calcolo infinitesimale 
  \item nell'ambito dello studio della \textbf{dinamica} dei corpi, Newton e Leibniz sviluppano contemporaneamente i fondamenti del \textbf{calcolo infinitesimale}, Parte \S\ref{book:calculus}, introducendo i concetti di \textbf{derivata}, Cap.\S\ref{ch:derivatives}, e \textbf{integrale}, Cap.\S\ref{ch:integrals}, e dimostrando il \textbf{il teorema fondamentale del calcolo infinitesimale} che lega le due operazioni 
  \item i fratelli Johann e Jacob Bernoulli contribuiscono allo sviluppo della statistica, formulando la \textbf{legge dei grandi numeri} e allo sviluppo del calcolo infinitesimale, dedicandosi allo studio e alla soluzione delle equazioni differenziali nell'ambito di problemi di meccanica; assunto da dal marchese de l'Hopital, Johann scopre la regola che prende il nome di \textbf{regola di de l'Hopital}
  \item vengono sviluppate le \textbf{serie di Taylor}, Sez.\ref{ch:taylor}
\end{itemize}
\paragraph{XVIII secolo}
\begin{itemize}
    \item Euler (Eulero): analisi matematica; soluzione \textbf{equazioni differenziali}; teoria dei numeri; \textbf{analisi complessa} (estensione di funzioni reali in campo complesso, identità di Eulero,\dots); topologia e teoria dei grafi (problema dei 7 ponti di Konigsberg)
    \item {\color{red} Daniel Bernoulli,}
    \item {\color{red} d'Alembert si dedica allo studio del moto dei corpi e alla meccanica razionale}
    \item sviluppi nel calcolo delle probabilità: vengono formulati il \textbf{teorema di Bayes} e il \textbf{metodo Monte Carlo}
    \item nella seconda metà del secolo, Parigi diventa il più importante centro matematico
        \begin{itemize}
            \item viene sviluppato il \textbf{calcolo delle variazioni} e utilizzato da Laplace (\textit{Méchanique Céleste}) e Lagrange (\textit{Méchanique analytique}) nella riformulazione della meccanica e nello studio del moto dei corpi celesti
            \item per la soluzione di problemi di meccanica, vengono introdotte le \textbf{trasformate} di Laplace e Legendre
            \item viene sviluppato il metodo dei \textbf{moltiplicatori di Lagrange}
            \item nell'ambito del calcolo differenziale, viene introdotto il concetto di potenziale e l'equazione differenziale di Laplace
        \end{itemize}
\end{itemize}
\paragraph{XIX secolo}
A Parigi si aggiungono i centri tedeschi di Berlino, Konigsberg e Gottinga
\begin{itemize}
    \item Jacobi e Cauchy ottengono notevoli risultati in \textbf{algebra lineare}, chiarendo il concetto di \textbf{determinante} di una matrice e introducendo il concetto di \textbf{jacobiano} di una funzione
    \item Cauchy si occupa di una formulazione e dimostrazione rigorosa di alcuni risultati di analisi infinitesimale, dando un impulso decisivo allo studio dell'\textbf{analisi complessa}, Cap.\S\ref{ch:complex-calculus}, estendendo i concetti dell'analisi infinitesimale alle funzioni di variabile complessa; Cauchy contribuisce allo sviluppo della statistica e allo studio delle successioni; fondamentali sono anche i suoi contributi alla \textbf{meccanica dei mezzi continui}, introducendo il concetto di sforzo e contribuendo ai primi risultati in teoria dell'elasticità
    \item \textbf{Fourier} introduce la \textbf{serie}, Sez.\S\ref{ch:fourier-series}, e \textbf{la trasformata} che prendono il suo nome, nello studio dei moti ondulatori e della trasmissione del calore
    \item Gauss, forse il più grande matematico della modernità insieme ad Euler, studia e insegna a Gottinga. Dopo aver proposto strumenti, risultati e congetture in teoria dei numeri, tra i quali l'\textbf{aritmetica modulare} e la \textbf{congettura dei numeri primi}, nella sua tesi di dottorato dimostra il \textbf{teorema fondamentale dell'algebra}; successivamente introduce il concetto di \textbf{numeri complessi}, Cap.\S\ref{ch:complex-algebra}, e la loro rappresentazione grafica nel piano complesso, di \textit{Argand e Gauss}; in piena maturità fu coinvolto in rilevazioni geografiche del Regno di Hannover, attività che gli permise di formulare dei risultati in stato embrionale di \textbf{geometria differenziale} delle superfici e delle curve e intuire l'esistenza di \textbf{geometrie non euclidee}: in questo ambito raggiunge un risultato fondamentale noto come \textit{theorema egregium}, che mette in realzione la \textbf{curvatura delle superfici} con la misura di distanze e angoli sulla superficie, informazioni riassumibili nella \textbf{metrica}; i lavori pratici e sperimentali che riguardavano le misure gli permisero di (ri)formulare il \textbf{metodo dei minimi quadrati} per la regressione di dati sperimentali e di sviluppare la \textbf{distribuzione di probabilità gaussiana}; negli ultimi anni di vita, da una collaboarzione con il fisico Weber, si dedica agli studi primordiali sull'elettromagnetismo, e formula il \textbf{teorema (di Gauss) del flusso del campo elettrico}
  \item Dirichlet succede a Gauss alla cattedra di Gottinga, e produce risultati in teoria dei numeri (alcuni dei quali furono pubblicati postumi da Dedekind), nello studio dei sistemi dinamici e della loro stabilità
  \item Riemann studia a Gottinga e succede alla cattedra di Dirichlet. A Riemann sono dovuti progressi in analisi reale e complessa, una rivoluzione e uno sviluppo della \textbf{geometria differenziale} che avrebbe fornito tutti gli strumenti necessari ad Einstein per formulare la teoria della relatività generale, e un'unica ma rivoluzionaria opera in teoria dei numeri in cui formula un'ipotesi che una volta dimostrata permetterebbe di determinare la distribuzione dei \textbf{numeri primi}, nota oggi come \textbf{ipotesi di Riemann}
  \item Hamilton riformula la meccanica lagrangiana in quella oggi nota come meccanica hamiltoniana, e introduce dei nuovi oggetti matematici, i \textbf{quaternioni}, estensione dei numeri complessi in 4 dimensioni utili per rappresentare le rotazioni nello spazio tridimensionale; ottiene risultati in algebra lineare (teorema di Cayley-Hamilton)
  \item Wierestrass: definizione rigorosa dei fondamenti dell'analisi (teorema di Weierstrass su esisteanza di minimi e massimi di funzioni a variabile reale)
  \item Boole definisce le operazioni di algebra sugli insiemi, l'\textbf{algebra booleana}, dando un contributo alla logica e alla teoria dell'informazione
  \item Peano: tentativo di definizione assiomatica della matematica
  \item Dedekind e Cantor si dedicano allo studio degli insiemi infiniti e alla loro dimensione, o cardinalità
\end{itemize}
\paragraph{XX secolo}
\begin{itemize}
  \item la matematica della probabilità e della meccanica quantistica: Lebesgue, Hilbert, von Neumann, Kolmogorov
  \item geometria: simmetrie, tassellature e impacchettamenti; i frattali (Mandelbrot)
  \item la nascita dell'informatica: Turing, Von Neumann
  \item la teoria dell'informazione: Shannon
  \item la teoria dei giochi: von Neumann, Morgestern e Nash
  \item l'incompletezza della matematica: Godel
\end{itemize}

