
\chapter{Strumenti e discipline in matematica}

{\color{red}
\section{Quantità}
\begin{itemize}
    \item numeri - scalari ($\mathbb{N}$, \dots, $\mathbb{R}$, \dots)
    \item vettori
    \item tensori
    \item quaternioni, e altri oggetti matematici esotici
\end{itemize}
}

\section{Discipline}
Nel corso della storia, la matematica è diventata una materia estremamente diversificata. 

Una classificazione delle molte aree di interesse della matematica può risultare utile ad avere una visione di insieme della materia. 

La matematica è divisa tradizionalmente in \textbf{matematica pura}, che studia gli strumenti propri della matematica e le loro proprietà, e \textbf{matematica applicata}, che applica gli strumenti della matematica a problemi di interesse tipici di altre discipline.

Una classificazione più dettagliata delle discipline della matematica è resa difficile dalle relazioni esistenti tra di esse. Viene qui presentata una classificazione parziale, ispirata alla \textit{Classificazione delle ricerche matematiche (MSC)} usata dall'\textit{American Mathematical Society} per la classificazione delle pubblicazioni nei database.

\subsection{Matematica pura}
\subsubsection{Fondamenti}
\begin{itemize}
    \item storia e filosofia
    \item logica
\end{itemize}
\subsubsection{Algebra e matematica discreta}
\begin{itemize}
    \item teoria dei numeri
    \item teoria dei campi
    \item algebra lineare e multilineare; matrici
\end{itemize}

\subsubsection{Analisi}
\begin{itemize}
    \item funzioni reali e complesse
    \item misura e integrazione
    \item equazioni differenziali, ed equazioni alle differenze
    \item sistemi dinamici
    \item successioni, serie e approssimazioni
    \item trasformate: analisi di Fourier
    \item calcolo delle variazioni, ottimizzazione e controllo
\end{itemize}

\subsubsection{Geometria e topologia}
\begin{itemize}
    \item geometria
    \item geometria differenziale
    \item topologia
\end{itemize}

\subsection{Matematica applicata}
\subsubsection{Statistica}
\subsubsection{Fisica}
\subsubsection{Altre scienze naturali}
\subsubsection{Scienze sociali e del comportamento}
\subsubsection{Matematica numerica e informatica}
\subsubsection{Ottimizzazione}
L'ottimizzazione si occupa della ricerca dei punti di massimo o di minimo di una funzione $f:A \rightarrow \mathbb{R}$.
\begin{equation}
    \text{Determinare il punto $\mathbf{x}_0 \in A$ tale che $f(\mathbf{x_0}) \ge f(\mathbf{x}) \ \forall \mathbf{x} \in A$}
\end{equation}
\subsubsection{Teoria dei sistemi e controllo}
La teoria dei sistemi rappresenta un qualunque sistema che evolve nel tempo $t$ con:
\begin{itemize}
    \item gli ingressi $\mathbf{u}(t)$, che possono agire sul sistema
    \item le uscite $\mathbf{y}(t)$, che possono essere lette dal sistema
    \item le variabili di stato, $\mathbf{x}(t)$ che determinano lo stato del sistema
    \item le equazioni di stato del sistema che ne governano la dinamica, legando lo stato e la sua evoluzione agli ingressi
        \begin{equation}
            \mathbf{f}(\dot{\mathbf{x}}, \mathbf{x}, \mathbf{u}) = \mathbf{0}
        \end{equation}
    \item le equazioni di uscita che permettono di ricavare le variabili di uscita in funzione dello stato e degli ingressi
        \begin{equation}
            \mathbf{y} = \mathbf{g}(\mathbf{x}, \mathbf{u})
        \end{equation}
\end{itemize}
Il problema del controllo di un sistema può essere formulato come un problema di ottimizzazione vincolata, volendone ottimizzare le prestazioni (o minimizzare le perdite), soggette ai vincoli definiti dalle equazioni che governano il sistema.
\subsubsection{Didattica}



\chapter{Approccio alla matematica}
\section{Formulazione e soluzione di un problema}
Una volta formulato un problema, ci si chiede:
\begin{itemize}
    \item il problema ammette soluzione?
    \item se il problema ammette soluzione, la soluzione è unica?
    \item se la soluzione non è unica, quante soluzioni esistono?
\end{itemize}

\paragraph{Esempi}


\chapter{Breve storia della matematica}
\paragraph{\dots}
\paragraph{XVI secolo}
\begin{itemize}
  \item Nepier (Nepero) introduce i logaritmi
\end{itemize}
\paragraph{XVII secolo}
\begin{itemize}
  \item Fermat
  \item Descartes (Cartesio) illustra ne \textit{La Gèometrie} i fondamenti della \textbf{geometria analitica}
  \item Huygens, Pascal
  \item Newton e Leibniz sviluppano contemporaneamente i fondamenti del \textbf{calcolo differenziale} e \textbf{integrale}, nell'ambito dello studio della \textbf{dinamica}
  \item fratelli Johann e Jakob Bernoulli
  \item de l'Hopital, Taylor
\end{itemize}
\paragraph{XVIII secolo}
\begin{itemize}
  \item Euler (Eulero): analisi matematica; soluzione equazioni differenziali; teoria dei numeri; analisi complessa (estensione di funzioni reali in ambito complesso; identità di Eulero); topologia e teoria dei grafi (problema die 7 ponti di Konigsberg)
  \item d'Alembert si dedica allo studio del moto dei corpi e lla meccanica razionale
  \item Legendre
  \item Bayes: probabilità
  \item istituzione di scuole scientifiche, Parigi importante centro scientifico del tempo
  \item Laplace: meccanica razionale e celeste (\textit{Mécanique Céleste}); trasformata; calcolo differenziale: potenziale, laplaciano ed equazione di Laplace
  \item Lagrange: formulazione lagrangiana della meccanica (\textit{Mécanique analytique}); calcolo delle variazioni; metodo dei moltiplicatori di Lagrange; teoria dei numeri
\end{itemize}
\paragraph{XIX secolo}
\begin{itemize}
  \item Jacobi: algebra lineare (determinante di matrici)
  \item Cauchy: algebra lineare; analisi complessa; statistica; teoria dei numeri; meccanica dei solidi
  \item Fourier: studio della trasmissione del calore; serie e trasformata di Fourier
  \item Gauss: teorema fondamentale dell'algebra; teoria dei numeri
  \item Dirichlet:
  \item Riemann: teoria dei numeri; geometria
  \item Hamilton: quaternioni; algebra lineare (teorema di Cayley-Hamilton); riformulazione della meccanica lagrangiana nella meccanica hamiltoniana
  \item Wierestrass: definizione rigorosa dei fondamenti dell'analisi (teorema di Weierstrass su esisteanza di minimi e massimi di funzioni a variabile reale)
  \item Boole: algebra sugli insiemi, logica, e teoria dell'informazione
  \item Peano: tentativo di definizione assiomatica della matematica
  \item Cantor: studio degli insiemi infiniti e la loro dimensione
\end{itemize}
\paragraph{XX secolo}
\begin{itemize}
  \item la matematica della probabilità e della meccanica quantistica: Lebesgue, Hilbert, von Neumann, Kolmogorov
  \item la nascita dell'informatica: Turing, Von Neumann
  \item la teoria dell'informazione: Shannon
  \item la teoria dei giochi: von Neumann, Morgestern e Nash
  \item l'incompletezza della matematica: Godel
\end{itemize}

