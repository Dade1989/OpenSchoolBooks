
\chapter{Aspetti studiati in matematica}

\chapter{Approccio alla matematica}
\section{Formulazione e soluzione di un problema}
Una volta formulato un problema, ci si chiede:
\begin{itemize}
    \item il problema ammette soluzione?
    \item se il problema ammette soluzione, la soluzione è unica?
    \item se la soluzione non è unica, quante soluzioni esistono?
\end{itemize}

\paragraph{Esempi}


\chapter{Breve storia della matematica}
\paragraph{\dots}
\paragraph{XVI secolo}
\begin{itemize}
  \item Nepier (Nepero) introduce i logaritmi
\end{itemize}
\paragraph{XVII secolo}
\begin{itemize}
  \item Fermat
  \item Descartes (Cartesio) illustra ne \textit{La Gèometrie} i fondamenti della \textbf{geometria analitica}
  \item Huygens, Pascal
  \item Newton e Leibniz sviluppano contemporaneamente i fondamenti del \textbf{calcolo differenziale} e \textbf{integrale}, nell'ambito dello studio della \textbf{dinamica}
  \item fratelli Johann e Jakob Bernoulli
  \item de l'Hopital, Taylor
\end{itemize}
\paragraph{XVIII secolo}
\begin{itemize}
  \item Euler (Eulero): analisi matematica; soluzione equazioni differenziali; teoria dei numeri; analisi complessa (estensione di funzioni reali in ambito complesso; identità di Eulero); topologia e teoria dei grafi (problema die 7 ponti di Konigsberg)
  \item d'Alembert si dedica allo studio del moto dei corpi e lla meccanica razionale
  \item Legendre
  \item Bayes: probabilità
  \item istituzione di scuole scientifiche, Parigi importante centro scientifico del tempo
  \item Laplace: meccanica razionale e celeste (\textit{Mécanique Céleste}); trasformata; calcolo differenziale: potenziale, laplaciano ed equazione di Laplace
  \item Lagrange: formulazione lagrangiana della meccanica (\textit{Mécanique analytique}); calcolo delle variazioni; metodo dei moltiplicatori di Lagrange; teoria dei numeri
\end{itemize}
\paragraph{XIX secolo}
\begin{itemize}
  \item Jacobi: algebra lineare (determinante di matrici)
  \item Cauchy: algebra lineare; analisi complessa; statistica; teoria dei numeri; meccanica dei solidi
  \item Fourier: studio della trasmissione del calore; serie e trasformata di Fourier
  \item Gauss: teorema fondamentale dell'algebra; teoria dei numeri
  \item Dirichlet:
  \item Riemann: teoria dei numeri; geometria
  \item Hamilton: quaternioni; algebra lineare (teorema di Cayley-Hamilton); riformulazione della meccanica lagrangiana nella meccanica hamiltoniana
  \item Wierestrass: definizione rigorosa dei fondamenti dell'analisi (teorema di Weierstrass su esisteanza di minimi e massimi di funzioni a variabile reale)
  \item Boole: algebra sugli insiemi, logica, e teoria dell'informazione
  \item Peano: tentativo di definizione assiomatica della matematica
  \item Cantor: studio degli insiemi infiniti e la loro dimensione
\end{itemize}
\paragraph{XX secolo}
\begin{itemize}
  \item la matematica della probabilità e della meccanica quantistica: Lebesgue, Hilbert, von Neumann, Kolmogorov
  \item la nascita dell'informatica: Turing, Von Neumann
  \item la teoria dell'informazione: Shannon
  \item la teoria dei giochi: von Neumann, Morgestern e Nash
  \item l'incompletezza della matematica: Godel
\end{itemize}

