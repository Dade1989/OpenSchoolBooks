
\chapter{Cronologia}


\paragraph{Fluorescenza e fosforescenza.}
Sono entrambi meccanismi di rilassamento di un atomo che si trova in uno stato eccitato.
\begin{itemize}
    \item la fluorescenza si interrompe immediatamente quando si interrompe la sorgente di eccitazione; è dovuta a una conversione \textit{singoletto-singoletto}
    \item la fosforescenza continua per un determinato lasso di tempo dopo lo spegnimento della sorgente di eccitazione; è dovuta a una conversione \textit{tripletto} e successivamente al decadimento allo stato fondamentale di \textit{singoletto}
\end{itemize}

\begin{definition}[Fluorescenza]
\'E la proprietà di un materiale di riemettere le radiazioni elettromagnetiche assorbite. Di solito, vengono assorbite radiazioni nell'ultravioletto e riemesse nel visibile. Descritta nel 1845 da John Herschel, può essere compresa solo con le conoscenze della struttura atomica del XX secolo.

\noindent
    L'effetto si interrompe immediatamente quando cessa la fonte di energia.
\end{definition}

\begin{definition}[Fosforescenza]
L'effetto continua per un lasso di tempo dopo lo spegnimento della fonte di energia.
\end{definition}

\paragraph{Studi su radioattività}
\begin{itemize}
    \item nel 1985 Rontgen scopre i raggi X
    \item nel 1896 Becquerel scopre la radioattività naturale dell'uranio, durante uno studio sulla relazione tra fosforescenza e raggi X; osserva il decadimento-$\beta$
    \item nel 1898 Marie e Pierre Curie scoprono gli elementi radioattivi polonio e radio
    \item nel 1898 Rutherford osserva il decadimento-$\alpha$
    \item nel 1900 Villard osserva il decadimento-$\gamma$
    \item in seguito agli studio di Rutherford, i decadimenti nucleari vengono classificati in 3 classi principali
        \begin{itemize}
            \item decadimento-$\alpha$
            \item decadimento-$\beta$
            \item decadimento-$\gamma$
        \end{itemize}
        alle quali successivamente si aggiungono
        \begin{itemize}
            \item emissione di neutroni
            \item emissione di protoni
            \item fissione spontanea
        \end{itemize}
\end{itemize}

\paragraph{Studi su tubi a vuoto e raggi}
\begin{itemize}
    \item \textbf{William Crooks} conduce una serie di ricerche sulla conduzione elettrica nei gas a bassa pressione (nei tubi di Crooks), osservando i cosiddetti \textit{raggi catodi}, poi scoperti essere raggi di elettroni
    \item \textbf{J.J. Thomson} riconosce l'elettrone nei raggi catodici, misurandone il rapporto $\dfrac{\text{carica}}{\text{massa}}$, costante al variare del gas usato nei tubi a vuoto
    \item \textbf{Eugen Goldstein} conduce esperimenti sui raggi catodici, insieme al fisico \textbf{Wilhelm Wien}. Riconosce l'esistenza di una corrente di particelle cariche positive in direzione inversa, i raggi anodici. A differenza degli elettroni nei raggi catodici, queste correnti positive dipendono dal gas all'interno del tubo, misurandone il rapporto $\dfrac{\text{carica}}{\text{massa}}$: riconoscendo il rapporto massimo usando l'idrogeno come gas, si può considerare questa la prima osservazione dei protoni, cioè di ioni $\textbf{H}^+$. Questi lavori gettano le basi per la spettrografia di massa.
\end{itemize}

\paragraph{Studi sulla struttura della materia} Cronologia degli studi
\begin{itemize}
    \item filosofi greci: natura discreta della materia, esistenza del vuoto? 
    \item \textbf{Evangelista Torricelli}, '600, esperimenti sulla pressione nei liquidi, acqua e mercurio: barometro di Torricelli, creazione del vuoto
    \item \textbf{Robert Boyle}, ultimo degli alchimisti e primo dei chimici (``toglie la "al" davanti a "chimia"''). Descrizione precisa degli esperimenti, in modo tale da poter essere repllicati. Pubblica un trattato che può essere considerato il primo testo di chimica.
        Indagini sulla natura discreta della materia. Legge di Boyle: ``l'aria si comporta come una molla'', a $T$ costante, $\Delta P = K \Delta\left(\frac{1}{V} \right)$
    \item \textbf{Daniel Bernoulli} pubblica \textit{Hydrodynamica}, nel quale presenta un'embrionale \textbf{teoria cinetica dei gas}: immagina i gas composto da una moltitudine di particelle in moto; la pressione su una parete è generata dagli urti delle particelle su di essa.
    \item \textbf{Antoine Lavoisier} (1743-1794), il più importante chimico del secolo. Importanza della misura del peso di reagenti e prodotti. Formula la \textbf{legge di conservazione della massa}, valida in fisica classica.
    \item \textbf{Robert Brown} (1773-1858) osserva il moto dei pollini al microscopio, successivamente chiamato moto browniano.
    \item Berthollet si chiede se le sostanze chimiche hanno una composizione definita
    \item Joseph Proust analizza il carbonato basico di rame dei tetti (malachite), riprodotta in laboratorio a partire dal rame: tutti e 3 i campioni hanno la stessa composizione, rispondendo a Berthollet: le sostanza hanno una composizione definita.
        {\color{red} ma come fa ad analizzare la composizione delle sostanze? Tramite reazioni?}
    \item \textbf{John Dalton} (1766-1844). Durante i suoi studi di botanica, scopre di non percepire i colori come i suoi colleghi: si accorge quindi di quello che chiamiamo daltonismo. Riprende gli studi di Lavoisier e Proust, dai qualli parte per le sue analisi chimiche, i cui risultati sono pubblicati nei \textit{Sistemi di filosofia chimica}: scopre che c'è sempre \textbf{un rapporto molto semplice} (tra interi ``piccoli'') tra i componenti elementari delle sostanze. Questa osservazione gli permette di formulare la sua teoria atomica. Inoltre spiega come pesare gli atomi in relazione all'atomo di idrogeno, ma purtroppo ne sbaglia la formula molecolare.
    \item \textbf{Gay--Lussac}: esperimenti sui gas, legati alla sua passione per le mongolfiere. Compie esperimenti concentrandosi sulla misura del volume: ad esempio, facendo reagire $H$, $Cl$ per ottenere $H \, Cl$, serve esattamente un litro di $H$ e un litro di $Cl$ per ottenere un litro di $H \, Cl$; 2l $H_2$ + 1l $O_2$ per ottenere 2l di $H_2 O$.
        Dibattito con Dalton sulle proporzioni nella reazione dell'acqua (dovuta agli esperimenti errati di Dalton, e all'errore logico e di autorità ``sono sbagliati i tuoi esperimenti perché non sono in accordo con la mia teoria'')
    \item \textbf{Amedeo Avogadro} (1776-1856) leggendo gli studi di Gay--Lussac, deduce la legge che prende il suo nome: \textit{Volumi uguali di gas nelle stess condizioni di temperatura e pressione contengono lo stesso numero di molecole},
       trovando la corretta composizione di idrogeno, ossigeno, acqua e la relazione bilanciata
        \begin{equation}
            2 H_2 + O_2 \rightarrow 2 H_2 O \ .
        \end{equation}
       Il suo lavoro rimane sconosciuto. La confusione dovuta agli studi sbagliati di Dalton continua.
    \item Liebig e Wholer (1824) sintetizzano due sostanze con la stessa composizione
    \item Berzelius: notazione moderna degli elementi. Controlla gli esperimenti di Liebig e Wholer e conferma che si ottengono due sostanze diverse con la stessa composizione, e gli dà il nome di \textbf{isomeri}: non solo la composizione, ma anche la loro disposizione all'interno degli atomi
    \item \textbf{Stanislao Cannizzaro} (1860) incaricato di tenere un corso universitario all'università di Genova, scrive delle dispense che, riprendendo il lavoro di Avogadro, cercano di dipanare la gran confusione presente allora in chimica e le pubblica come un corso di filosofia chimica.
    \item la fisica sviluppa teorie della termodinamica, incluse le leggi della trasmissione del calore, e dell'elettromagnietismo che prevedono che la materia sia continua e infinitamente divisibile.
    \item \textbf{Rudolf Clausius} si pone la domanda: come si muovono gli atomi? Domanda che ai chimici poco interesava. Nel 1857 pubblica un lavoro sulla teoria cinetica dei gas legando la temperatura dei gas con la velocità media delle particelle, prevedendo delle velocità enormi, $\sim 500 m/s$. Le molecole si urtano tra di loro.
    \item \textbf{James Clerk Maxwell} sviluppa la teoria cinetica dei gas, prevedendo la distribuzione di velocità delle molecole in un fluido
    \item \textbf{Ludwig Boltzmann} (1844-1906) partendo dagli studi di Maxwell, con i suoi sviluppi fonda quella che oggi conosciamo come Meccanica Statistica. Diatriba con Mach sulla natura atomica della materia. Debole psiscologicamente, si toglie la vita a Trieste a inizio '900.
    \item nel 1905, Einstein pubblica il lavoro sul moto browniano per una spiegazione del moto del polline osservato da Brown: non si riesce a vedere il moto delle molecole d'acqua, ma l'effetto dei loro urti sul polline. Perrin, compie un esperimento che lega la teoria matematica di Einstein sul moto browniano e le osservazioni sul polline.
    \item \dots
    \item \textbf{J.J. Thomson} studi sui raggi catodici:
        \begin{itemize}
             \item 1897: scoperta dell'elettrone, e stima del rapporto $\frac{\text{massa}}{\text{carica}}$
             \item modello atomico ``a panettone''
             \item scperta degli isotopi
        \end{itemize}
        A Thomson sono dovute la scoperta della radioattività naturale del potassio e che latomo di idrogeno ha un solo elettrone.
    \item \textbf{Ernest Rutherforf} (1871-1937)
        \begin{itemize}
            \item lavori sulla radioattività:
                \begin{itemize}
                    \item studio sulla radioattività di torio, uranio e radio
                    \item studi sul tempo di dimezzamento (o emivita) nei processi radioattivi
                \end{itemize}
            \item lavori sulla struttura atomica 
                \begin{itemize}
                    \item esperimenti di Geiger--Marsden: bombardamento con particelle-$\alpha$ di una lamina d'oro
                    \item nel 1911 Rutherford interpreta lo scattering delle particelle-$\alpha$ su uno schermo fluorescente (solfuro di zinco), e propone il modello di atomo, con un nucleo positivo compatto
                    \item nel 1919 viene scoperto il protone
                \end{itemize}
        \end{itemize}
    \item nel 1910 \textbf{Robert Millikan} pubblica i risultati dei suoi studi che consentono la misura la carica dell'elettrone con l'esperimento della goccia d'olio
    \item \textbf{James Chadwick} scopre il neutrone nel 1932
    \item \dots
    \item 1981: microscopi a forza atomica
\end{itemize}

\subsubsection*{Riferimenti}
\begin{itemize}
    \item conferenza di Dario Bressanini, \textit{Vedere gli atomi con la mente -- La scoperta degli atomi}
    \item video dell'IBM sul film più piccolo al mondo, \textit{A boy and his atom}
\end{itemize}

\vspace{20pt}
Capitoli successivi:
\begin{itemize}
    \item Prerequisiti:
        \begin{itemize}
            \item fluorescenza
            \item fosforescenza
        \end{itemize}
    \item Esperimenti:
        \begin{itemize}
            \item
        \end{itemize}
\end{itemize}

\chapter{Radioattività}
\section{Storia}

\section{Decadimenti nucleari}
\subsection{Decadimento-$\alpha$}
Osservati da Rutherford nel 1899.
Un nucleo instabile emette una particella-$\alpha$, cioè un nucleo di elio ${}^4 \text{He}$, composto da 2 protoni e due neutroni
\begin{equation}
    A_1(Z,N) \quad \rightarrow \quad A_2(Z-2,N-2) + \alpha
\end{equation}
\'E un fenomeno che si manifesta come la tendenza di tutti i sistemi fisici a convergere verso condizioni stabili di energia minima, che si manifesta principalmente per elementi transuranici con $A > 210$, tramite l'\textit{effetto tunnel}.
\subsection{Decadimento-$\beta$}
Osservati da Becuqerel nel 1896.
La radiazione emessa è di elettroni.

\noindent
Si possono distinguere 3 tipi di decadimento-$\beta$:
\begin{itemize}
    \item decadimento positivo
        \begin{equation}
            A(Z,N) \ \rightarrow \ A(Z-1,N+1) + e^- + \overline{\nu}_e
        \end{equation}
    \item decadimento negativo
        \begin{equation}
            A(Z,N) \ \rightarrow \ A(Z-1,N+1) + e^+ + \nu_e
        \end{equation}
    \item cattura elettronica
        \begin{equation}
            A(Z,N) + e^- \ \rightarrow \ A(Z-1,N+1) + \nu_e
        \end{equation}
\end{itemize}

\noindent
Esempio
\begin{equation}
    {}^{60}_{27} \text{Co} \ \rightarrow \ {}^{60}_{27} \text{Ni}^* + e^- + \overline{\nu}_e
\end{equation}
\subsection{Decadimento-$\gamma$}
Osservati da Villard nel 1900.
La radiazione è composta da fotoni, quanti di radiazione elettromagnetica, non da materia. Non essendo composti da materia, hanno un potere ionizzante minore, e per una loro schermatura servono spessori maggiori.
I raggi-$\gamma$ sono prodotti da transizionei nucleari o subatomiche.

\noindent
Esempio
\begin{equation}
\begin{aligned}
    & {}^{60}_{27} \text{Co}   \ \rightarrow \ {}^{60}_{28} \text{Ni}^* + e^- + \overline{\nu}_e \\
    & {}^{60}_{27} \text{Ni}^* \ \rightarrow \ {}^{60}_{28} \text{Ni} + \gamma
\end{aligned}
\end{equation}

\noindent
In termini di ionizzazione, la radiazione-$\gamma$ interagisce con la materia in tre modi principali:
\begin{itemize}
    \item l'effetto fotoelettrico
    \item lo scattering di Compton
    \item la creazione di coppie
\end{itemize}

\subsection{Emissione di neutroni}
\subsection{Emissione di protoni}
\subsection{Fissione spontanea}
\'E un processo di decadimento tipico di isotopi pesanti

\chapter{Struttura della materia}

\paragraph{Esperimento di Thomson.} Possiamo riassumere l'idea dell'esperimento di Thomson analizzando la deviazione di una carica in movimento quando attraversa una regione dello spazio dove è presente un campo magnetico.

\noindent
\begin{itemize}
    \item Velocità iniziale della carica in moto: $\mathbf{v} = v \mathbf{\hat{x}}$
    \item traiettoria della carica soggetta alla forza di Lorentz in una regione di spazio con campo elettromagnetico uniforme
        \begin{equation}
            m \ddot{\mathbf{r}} = \mathbf{F}^{Lorentz} = q \left ( \mathbf{E} - \mathbf{B} \times \mathbf{v} \right) =  q \left ( \mathbf{E} - \mathbf{B} \times \dot{\mathbf{r}} \right) \ .
        \end{equation}
    \begin{itemize}
        \item solo campo elettrico trasversale al moto, $\mathbf{E} = E \mathbf{\hat{y}}$
            \begin{equation}
            \begin{cases}
                m \ddot x = 0 \\
                m \ddot y = q E
            \end{cases} \qquad \rightarrow \qquad
            \begin{cases}
                x(t) = v t \\
                y(t) = \frac{1}{2} \frac{q}{m} E t^2
            \end{cases}
            \end{equation}
            Se la lunghezza in direzione $x$ della regione in cui il campo elettrico non è trascurabile è $L$, la particella raggiunge questo punto in $t^* = \frac{L}{v}$ la velocità della particella carica in $x = L$ è 
            \begin{equation}
            \begin{cases}
                u(t^*) = v \\
                v(t^*) = \frac{1}{2} \frac{q}{m} E \left( \frac{L}{v} \right)^2
            \end{cases}
            \end{equation}
        \item solo campo magnetico trasversale al moto, $\mathbf{B} = B \mathbf{\hat{y}}$ 
            \begin{equation}
            \begin{cases}
                m \ddot x = - q B \dot{z} \\
                m \ddot y = 0 \\
                m \ddot z = q B \dot{x} \\
            \end{cases} \qquad \rightarrow \qquad
            \begin{cases}
                \dddot{x} + \left(\frac{q}{m}\right)^2 B^2 \dot{x} = 0 \\
                y(t) = 0 \\
                \dddot{z} + \left(\frac{q}{m}\right)^2 B^2 \dot{z} = 0 \\
            \end{cases}
            \end{equation}
            \begin{equation}
            \begin{cases}
                \ddot{u} + \left(\frac{q}{m}\right)^2 B^2 u = 0 \\
                y(t) = 0 \\
                \ddot{w} + \left(\frac{q}{m}\right)^2 B^2 w = 0 \\
            \end{cases} \qquad \rightarrow \qquad
            \begin{cases}
                u(t) = A_u \cos (\Omega t) + B_u \sin (\Omega t) \\
                v(t) = 0 \\
                w(t) = A_w \cos (\Omega t) + B_w \sin (\Omega t) \\
            \end{cases} 
            \end{equation}
            \begin{equation}
            \begin{cases}
                v = u(0) = A_u \\
                0 = v(0) = 0 \\
                0 = w(0) = A_w \\
            \end{cases} \qquad \rightarrow \qquad
            \begin{cases}
                x(t) = \frac{v}{\Omega} \sin (\Omega t) - \frac{B_u}{\Omega} \cos (\Omega t) + C_u \\
                y(t) = 0 \\
                z(t) = - \frac{B_w}{\Omega} \cos (\Omega t) + C_w \\
            \end{cases} 
            \end{equation}
            \begin{equation}
            \begin{cases}
                0 = x(0) = - \frac{B_u}{\Omega} + C_u \\
                0 = y(0) = 0 \\
                0 = z(0) = - \frac{B_w}{\Omega} + C_w \\
            \end{cases} \qquad \rightarrow \qquad
            \begin{cases}
                x(t) = \frac{v}{\Omega} \sin (\Omega t) + \frac{B_u}{\Omega} \left[ 1 - \cos (\Omega t) \right] \\
                y(t) = 0 \\
                z(t) = \frac{B_w}{\Omega} \left[ 1 - \cos (\Omega t) \right] \\
            \end{cases} 
            \end{equation}
            \begin{equation}
            \begin{aligned}
                0 & = \ddot{x} + \Omega \dot{z} = \\
                  & = - v \Omega \sin(\Omega t) + B_u \Omega \cos(\Omega t) + \Omega \left[ B_w \sin(\Omega t) \right]
            \end{aligned}
            \end{equation}
            \begin{equation}
               \rightarrow \quad B_w = v \quad , \qquad B_u = 0 \ .
            \end{equation}
            \begin{equation}
            \begin{cases}
                x(t) = \frac{v}{\Omega} \sin (\Omega t) \\ 
                y(t) = 0 \\
                z(t) = \frac{v}{\Omega} \left[ 1 - \cos (\Omega t) \right] \\
            \end{cases} \qquad \rightarrow \qquad
            \begin{cases}
                x(t) = \frac{v}{B} \frac{m}{q} \sin \left(\frac{q}{m} B t \right) \\ 
                y(t) = 0 \\
                z(t) = \frac{v}{B} \frac{m}{q} \left[ 1 - \cos \left(\frac{q}{m} B t \right) \right] \\
            \end{cases}
            \end{equation}
    \end{itemize}
\end{itemize}


\paragraph{Esperimento di Millikan.} Esperimento delle gocce d'olio. La resistenza aerodinamica agente su un corpo di piccole dimensioni è proporzionale alla velocità,
\begin{equation}
    \mathbf{F} = - c \mathbf{v} \ .
\end{equation}
L'equazione del moto in direzione verticale (con l'asse positivo verso il basso) di una particella d'olio in un campo elettrico $\mathbf{E} = E \mathbf{\hat{z}}$ è 
\begin{equation}
    m \ddot{z} = m g - c \dot{z} + q E \quad \rightarrow \quad
    m \dot{v} + c v = m g
\end{equation}
L'integrazione delle equazioni differenziali con velocità iniziale $v(0) = v_0$ porta all'espressione della velocità delle particelle d'olio,
\begin{equation}
    v(t) =  \frac{mg + qE}{c} + \left( v_0 - \dfrac{mg + qE}{c} \right)  e^{-\frac{c}{m} t}  \ .
\end{equation}
Le velocità limite nel caso di campo elettrico nullo o campo elettrico acceso sono,
\begin{equation}
    v_{\infty,0} = \dfrac{mg}{c} \qquad , \qquad
    v_{\infty} = \dfrac{mg + qE}{c} \ .
\end{equation}
Dal confronto delle due velocità, si può calcolare la carica delle particelle,
\begin{equation}
    \qquad \rightarrow \qquad
    \frac{q E}{c} = v_{\infty} - v_{\infty,0}
    \qquad \rightarrow \qquad
    q = \dfrac{c}{E} \left( v_{\infty} - v_{\infty,0} \right) \ .
\end{equation}
Il campo elettrico $E$ è regolabile nell'esperimento, le velocità limite $v_{\infty}$, $v_{\infty,0}$ possono essere misurate, il coefficiente di resistenza $c$ può essere stimato dalla legge di Stokes per delle sfere minuscole
\begin{equation}
    c = 6 \pi \, \mu \, r
\end{equation}

