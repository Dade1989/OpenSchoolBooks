
\begin{itemize}
    \item \textbf{Robert Boyle} (1627-1691) formula una teoria atomistica della materia. Compie studi sulle pompe ad aria insieme a \textbf{Robert Hooke}, usate negli esperimenti sulle proprietà dei gas, che gli permettono di formulare la legge che prenderà il suo nome e e che formula la relazione inversa tra pressione e volume occupato da un gas a temperatura costante.
    \item Newcomen (1664-1729) presenta la sua macchina a vapore, come prima applicazione industriale del vapore.
    \item \textbf{Daniel Bernoulli} (1700-1782) pubblica nel 1738 il lavoro \textit{Hydodynamica}, opera sullo studio dei fluidi, che contiene una prima formulazione della \textbf{teoria cinetica dei gas}
    \item \textbf{James Watt} (1736-1819) all'università di Glasgow, conduce esperimenti sul vapore, insieme a Joseph Black, che introduce i concetti di calore latente e del calore specifico. Watt migliora la macchina di Newcomen e presenta il suo modello di macchina a vapore. Con l'introduzione della distribuzione a cassetti e del meccanismo biell-manovella per creare un moto rotatorio dal moto alternativo del pistone, la macchina a vapore diventa una macchina motrice a utilizzo generalizzato in molti settori, protagonista della rivoluzione industriale inglese
    \item esperimenti sui gas permettono a \textbf{Jacques Charles}, \textbf{Alessandro Volta} di formulare indipendentemente la legge che verrà pubblicata nel 1802 dal \textbf{Joseph Louis Gay-Lussac}, sulla proporzionalità diretta tra temperatura e volume occupato da un gas in un sistema chiuso durante una trasformazione isobara. Gay-Lussac formula poi la legge sulla proporzionalità diretta tra pressione e temperatura di un gas in un sistema chiuso a volume costante.
    \item \textbf{Sadi Carnot} (1796-1832), interessato allo studio e al miglioramento dell'efficienza dei motori termici, pubblica nel 1824 \textit{Riflessioni sulla forza motrice del fuoco}, testo che getta le basi per lo studio della termodinamica. {\color{red} cosa c'è in questo testo di interessante?}
    \item \textbf{Benoit Paul Emile Clapeyron} (1799-1864) sviluppa il lavoro di Carnot, introducendo una descrizione analitica e grafica dei processi termodinamici. Indaga i cicli termodinamici nel \textbf{piano $P-V$} che prende il suo nome. Riprende e sviluppa l'idea di Carnot di \textbf{trasformazione reversibile} e riformula il \textbf{prinicipio di Carnot} sull'efficienza massima di delle macchine termiche, fornendo una prima formulazione di quello che diventerà il secondo principio della termodinamica. Lavora inoltre sulle transizioni di fase, e sulla statica delle strutture elastiche.
    \item gli studi \textbf{James Prescott Joule} (1818-1889) sulla natura del calore mettono in relazione il calore e il lavoro meccanico, 
    \item \textbf{Rudolf Clausius} (1822-1888)
        \begin{itemize} 
            \item formula i principi della termodinamica. 
                \begin{itemize}
                    \item Nel 1850 formula il \textbf{primo principio della termodinamica}, introducendo il concetto di \textbf{energia interna} come variabile di stato del sistema: il primo principio della termodinamica rappresenta un bilancio dell'energia totale e, come caso particolare, la prima formulazione della \textbf{legge della conservazione dell'energia}.
                    \item Nel 1865 riformula il principio di Carnot e introduce il concetto di \textbf{entropia} per la distinzione tra processi reversibili e irreversibili, e la formulazione del \textbf{secondo principio della termodinamica};
                \end{itemize} 
            \item è autore di sviluppi nella teoria cinetica dei gas, introducendo i gradi di libertà molecolari e il concetto di cammino libero medio di una particella.
       \end{itemize}
   \item \textbf{William Rankine} (1820-1872), a Glasgow, dopo i primi anni dedicati alla scienza delle costruzioni e alla meccanica, si concentra sulla termodinamica e lo studio del moto dei fluidi (interessi su onde d'urto e vortici):
       \begin{itemize}
           \item compie studi sul bilancio di energia, coniando il termine \textit{energia potenziale}, e dando la definizione di \textit{energia come la capacità di compiere lavoro}
           \item compie studi sull'evaporazione di un liquido, e sul vapore, fornendo l'espressione del calore latente di exaporazione
           \item applica i suoi studi ai motori termici, proponendo il \textbf{ciclo termodinamico} che prende il suo nome
       \end{itemize}
    \item \textbf{William Thomson} (1824-1907), anche noto come \textbf{Lord Kelvin}, è autore di importanti ricerche nell'ambito dell'analisi dell'elettromagnetismo e della termodinamica, oltre ad aver partecipato tra le sue tante attività alla progettazione alla posa dei cavi transatlantici per la trasmissione delle informazioni. In termodinamica, si possono ricordare:
        \begin{itemize}
            \item la \textbf{scala di temperatura assoluta}, e l'introduzione del concetto di \textbf{zero assoluto}
            \item l'enunciato della seconda legge della termodinamica
            \item la scoperta dell'effetto Joule-Thomson
        \end{itemize}
    {\color{red}
    \item \textbf{James Clerk Maxwell}
    \item \textbf{Ludwig Boltzmann}
    }
\end{itemize}


% ------------------------------------------------------------------------------
\chapter{Principi della termodinamica}\label{thermodynamics:principles}
\section{Primo principio}
Il primo principio della termodinamica è il bilancio di energia totale del sistema: la variazione di energia totale di un sistema è uguale alla somma del calore entrante nel sistema dall'ambiente e del lavoro delle forze esterne agenti sul sistema. In forma incrementale
\begin{equation}
  \Delta E^{tot} = Q^{ext} + L^{ext} \ ,
\end{equation}
\begin{equation}
  \dot{E}^{tot} = \dot{Q}^{ext} + P^{ext} \ .
\end{equation}
%
Usando il teorema dell'energia cinetica della meccanica,
\begin{equation}
  \dot{K} = P^{tot} = P^{ext} + P^{int} \ ,
\end{equation}
e definendo l'energia interna $U$ del sistema come differenza tra l'energia totale e l'energia cinetica,
\begin{equation}
  U := E^{tot} - K \ ,
\end{equation}
si può ricavare un'equazione per il bilancio dell'energia interna
\begin{equation}
  \dot{U} = \dot{Q}^{ext} - P^{int} \ .
\end{equation}

\section{Secondo principio}
Il secondo principio della termodinamica introduce
Il secondo principio della termodinamica ha diversi enunciati equivalenti, formulati da Clausis, Kelvin e Planck.
L'enunciato più generale è quello di Clasius, mentre gli enunciati di Kelvin e Planck coinvolgono macchine termiche e quindi, per questi due enunciati, si rimanda al capitolo \ref{thermodynamics:thermal_machines} sulle macchine termiche.

Si assume di poter separare il contributo della potenza delle forze interne $P^{int}$ nella somma della potenza delle forze reversibili e in quella delle forze irreversibili, definita dissipazione, $P^{int} = P^{int,rev} + D$

\begin{equation}
\begin{aligned}
  d U & = \left(\dfrac{\partial U}{\partial x}\right)_S dx + \left(\dfrac{\partial U}{\partial S}\right)_x dS \\
  d U & = \delta Q^{ext} - \delta L^{int} = \\
      & = \delta Q^{ext} - \delta^r L^{int,r} + \underbrace{\delta^+ D}_{\ge 0} = \\
      & = -\delta^r L^{int,r} + \delta Q^{ext} + \delta^+ D 
\end{aligned}
\end{equation}
\begin{equation}
  - \delta^r L^{int,r} =  \left(\dfrac{\partial U}{\partial x}\right)_S dx
\qquad , \qquad
  \delta Q^{ext} + \delta^+ D = \left(\dfrac{\partial U}{\partial S}\right)_x dS \ .
\end{equation}
Definendo la temperatura $T := \left(\dfrac{\partial U}{\partial S}\right)_x > 0$, per il terzo principio, si può riscrivere 
\begin{equation}
  T dS = \delta Q^{ext} + \delta^+ D \ge \delta Q^{ext} \qquad \rightarrow \qquad dS \ge \dfrac{\delta Q^{ext}}{T} \ .
\end{equation}

\subsection{Conseguenze del secondo principio della termodinamica}
Nello scambio di calore tra due sistemi $S_1$, $S_2$, si manifesta un flusso di calore positivo dal corpo a temperatura maggiore a quello a temperatura minore. Chiamando $Q_{21}$ il flusso di calore tra i due sistemi, scegliendo il segno positivo se fa aumentare l'energia del sistema $2$,
\begin{equation}
\begin{aligned}
    &  Q_{21} > 0 \qquad \text{se $T_1 > T_2$} \\ 
    &  Q_{21} < 0 \qquad \text{se $T_1 < T_2$} \\ 
\end{aligned}
\end{equation}
possiamo riassumere queste condizioni sulla trasmissione del calore con l'unica relazione
\begin{equation}
    Q_{21} \left( \dfrac{1}{T_2} - \dfrac{1}{T_1} \right) > 0 \ ,
\end{equation}
o, introducendo il flusso di calore ``in direzione opposta'' $Q_{12} = -Q_{21}$,
\begin{equation}
    \dfrac{Q_{21}}{T_2} + \dfrac{Q_{12}}{T_1} > 0 \ .
\end{equation}

\subsection{Integrale di Clausius}
Chiamando $\delta Q$ il flusso di calore tra l'ambiente esterno e un sistema omogeneo, positivo se ``entrante'' nel sistema,
si può scrivere
\begin{equation}
    \dfrac{\delta Q}{T} - \dfrac{\delta Q}{T^{ext}} > 0
\end{equation}
\begin{equation}
    0 = \oint dS \ge \oint \dfrac{\delta Q}{T} \ge \oint \dfrac{\delta Q}{T^{ext}} 
\end{equation}

\subsection{Secondo principio della dinamica per sistemi non omogenei}

\begin{equation}
\begin{aligned}
    dS = \sum_i dS_i
    & = \sum_i \left( \dfrac{\delta Q^{i,ext(i)}}{T_i} + \dfrac{\delta^+ D_i}{T_i} \right) = \\
    & = \sum_i \left( \dfrac{\delta Q^{i,ext} + \sum_{j \ne i} \delta Q_{ij}}{T_i} + \dfrac{\delta^+ D_i}{T_i} \right) = \\
    & = \sum_i \underbrace{ \dfrac{\delta Q^{i,ext}}{T_i} }_{\ge \frac{\delta Q^{i,ext}}{T_i^{ext}}} +  \underbrace{ \sum_i \sum_{j \ne i} \dfrac{\delta Q_{ij}}{T_i} }_{\ge 0} + \sum_i \underbrace{ \dfrac{\delta^+ D_i}{T_i}}_{\ge 0} \ge \sum_i \dfrac{\delta Q^{i,ext}}{T_i^{ext}}
\end{aligned}
\end{equation}


\section{Terzo principio}
Il terzo principio della termodinamica postula la positività della temperatura
\begin{equation}
  T := \left(\dfrac{\partial U}{\partial S}\right)_x > 0
\end{equation}

\section{Principio zero - Equilibrio termico}

% ==============================================================================
\chapter{Relazioni e coefficienti termodinamici}

% ==============================================================================
\chapter{Stati della materia e leggi costitutive}
% ------------------------------------------------------------------------------
\section{Stati della materia, variabili termodinamaiche e diagramma di stato}
% ------------------------------------------------------------------------------
\section{Gas}
\subsection{Leggi sui gas: Boyle, Gay-Lussac, Charles}
\begin{equation}
    P V = \text{cost} \qquad \text{con $T$ costante (legge di Boyle)}
\end{equation}
\begin{equation}
    \dfrac{V}{T} = \text{cost} \qquad \text{con $P$ costante (legge di Charles)}
\end{equation}
\begin{equation}
    \dfrac{P}{T} = \text{cost} \qquad \text{con $V$ costante (legge di Gay-Lussac)}
\end{equation}
\subsection{Equazione di stato dei gas perfetti}
\begin{equation}
    P V = n R T
\end{equation}
con $n$ numero di moli, $R$ \textbf{costante dei gas}.

La costante dei gas è legata alla \textbf{costante di Boltzmann} $k_B$ dal \textbf{numero di Avogadro} $N_A$,
\begin{equation}
    R = k_B N_A
\end{equation}
\begin{equation}
    P V = n R T = \underbrace{n N_A}_{= N} k_B T = N k_B T
\end{equation}
con $N$ numero di particelle (molecole) del sistema.

\subsection{Equazione di stato di Van der Waals -- equazione di stato dei gas reali}

\subsection{Teoria cinetica dei gas -- cenni}

\subsubsection{Equipartizione dell'energia}

\paragraph{Gradi di libertà delle molecole}


% ------------------------------------------------------------------------------
\section{Liquidi}
% ------------------------------------------------------------------------------
\section{Solidi}
% ------------------------------------------------------------------------------
\section{Miscele}
% ------------------------------------------------------------------------------
\section{Vapore e aria umida}

% ==============================================================================
\chapter{Macchine termiche}\label{thermodynamics:thermal_machines}
% ------------------------------------------------------------------------------
\begin{definition}[Ciclo termodinamico] Un ciclo termodinamico è un insieme di trasformazioni termodinamiche che riporta il sistema allo stato di partenza.
\end{definition}
Trascurando i termini dinamici, $E^{tot} = K + U \simeq U$, il primo principio della termodinamica
\begin{equation}
    \Delta U = L + Q
\end{equation}
tra lo stato iniziale e finale del ciclo, coincidenti, prevede $\Delta U = 0$, e quindi
\begin{equation}
    0 = L + Q \ ,
\end{equation}
o distinguendo i contributi di calore entrante $Q^{in} > 0$ e uscente $Q^{out} < 0$, si può riscrivere
\begin{equation}
    0 = L + Q^{in} + Q^{out} \ .
\end{equation}

\begin{definition}[Ciclo termodinamico diretto] In un ciclo termodinamico diretto, si fornisce calore $Q^{in}$ alla macchina per ottenere un lavoro meccanico, $L<0$.
\end{definition}
\begin{example}[Motori a combustione interna -- alternativi]
\end{example}
\begin{example}[Motori a combustione interna -- continui]
\end{example}
\begin{example}[Macchina a vapore]
\end{example}

\begin{definition}[Ciclo termodinamico inverso] In un ciclo termodinamico inverso, si fornisce lavoro (e quindi $L < 0$) a una macchina termica per sottrarre calore a un sistema.
\end{definition}
\begin{example}[Frigorifero]
\end{example}

\begin{definition}[Efficienza di un ciclo termodinamico]
    \begin{equation}
        \eta = \dfrac{\text{Effetto utile}}{\text{Ingresso del sistema}}
    \end{equation}
\end{definition}

\begin{definition}[Efficienza di un ciclo termodinamico diretto] L'efficienza $\eta$ di un ciclo termodinamico è il rapporto tra il lavoro utile ottenuto $L$ e il calore fornito $Q^{in}$ (che rappresenta il massimo lavoro ottenibile)
    \begin{equation}
        \eta = \dfrac{-L}{Q^{in}} = \dfrac{Q^{in}+Q^{out}}{Q^{in}} = 1 + \dfrac{Q^{out}}{Q^{in}} =  1 - \dfrac{|Q^{out}|}{Q^{in}} \ .
    \end{equation}
\end{definition}

\begin{definition}[Efficienza di un ciclo termodinamico inverso] $L > 0$
    \begin{equation}
        \eta = \dfrac{Q^{in}}{L}
    \end{equation}
\end{definition}

\section{Macchina ideale di Carnot}
\begin{definition}[Ciclo di Carnot]
    Il ciclo ideale di Carnot è formato da due trasformazioni isoterme e due adiabatiche.
\end{definition}
\begin{definition}[Macchina ideale di Carnot]
    La macchina ideale di Carnot è una macchina reversibile che opera tra due temperature, compiendo un ciclo di Carnot.
\end{definition}
\begin{definition}[Principio di Carnot]
    A pari temperature estreme di un ciclo termodinamico, la macchina ideale di Carnot è la macchina che ha efficienza massima,
    \begin{equation} \eta \le \eta_C \ . \end{equation}
\end{definition}

\noindent
Usando la relazione
\begin{equation}
    0 \ge \oint \dfrac{\delta Q}{T} \ge \oint \dfrac{\delta Q}{T^{ext}} \ ,
\end{equation}
si può scrivere
\begin{equation}
    0 \ge \dfrac{Q_1}{T_1} + \dfrac{Q_2}{T_2} \qquad \rightarrow \qquad \dfrac{-Q_2}{Q_1} \ge \dfrac{T_2}{T_1}
\end{equation}
\begin{equation}
    \eta = \dfrac{-L}{Q_1} = \dfrac{Q_1 + Q_2}{Q_1} = 1 + \dfrac{Q_2}{Q_1} \le 1 - \dfrac{T_2}{T_1} =: \eta_C \ .
\end{equation}

% ------------------------------------------------------------------------------
\section{Enunciati del secondo principio della termodinamica di Kelvin-Planck e Clausius}

\begin{definition}[Enunciato di Kelvin] \'E impossibile realizzare una macchina termica ciclica in grado di convertire in lavoro tutto il calore assorbito da una sorgente a temperatura costante.
\end{definition}
\begin{proof}[Equivalenza enunciati di Clausius e Kelvin] Dall'enunciato di Clausius del secondo principio della termodinamica, si può ricavare
\begin{equation}
    0 \ge \dfrac{Q_1}{T_1} + \dfrac{Q_2}{T_2} \ .
\end{equation}
Se il calore $Q_2$ scambiato con il corpo a temperatura $T_2$ è nullo, allora
\begin{equation}
     0 \ge \dfrac{Q_1}{T_1} \ ,
\end{equation}
e quindi, poiché la temperatura assoluta è una quantità non-negativa, segue che
\begin{equation}
    Q_1 = -L \le 0 \ ,
\end{equation}
ossia, una macchina che scambia calore con il solo corpo a temperatura $T_1$, assorbe lavoro e cede calore corpo $1$. \'E quindi impossibile realizzare la situazione contraria, cioè una macchina che assorbe calore unicamente dalla sorgente a temperatura costante $T_1$ e produce lavoro $-L = Q_1 > 0$.
\end{proof}

\begin{definition}[Enunciato di Planck] \'E impossibile realizzare una trasformazione il cui unico risultato sia quello di trasferire calore da un corpo a una temperatura data a un corpo a temperatura maggiore.
\end{definition}
\begin{proof}[Equivalenza enunciati di Clausius e Planck] Dall'enunciato di Clausius del secondo principio della termodinamica, si può ricavare
\begin{equation}
    0 \ge \dfrac{Q_1}{T_1} + \dfrac{Q_2}{T_2} \ .
\end{equation}
Se il lavoro fatto sul sistema è nullo, $L = 0$, allora
\begin{equation}
    0 = Q_1 + Q_2 \ ,
\end{equation}
e quindi,
\begin{equation}
    0 \ge Q_1 \left( \dfrac{1}{T_1} - \dfrac{1}{T_2} \right) \ .
\end{equation}
    Se $T_1 > T_2$, segue che $\left(\frac{1}{T_1} - \frac{1}{T_2} \right) < 0$ e quindi $Q_1 = - Q_2 > 0$, cioè il sistema trasferisce il calore dalla sorgente a temperatura maggiore a quella a temperatura minore.

\end{proof}

% ------------------------------------------------------------------------------
\section{Cicli termodinamici e macchine termiche}
\subsection{Cicli termodinamici diretti}
\subsubsection{Ciclo Otto}
\subsubsection{Ciclo Diesel}
\subsubsection{Ciclo Joule-Brayton}
\subsubsection{Ciclo Rankine}
% ------------------------------------------------------------------------------
\subsection{Cicli termodinamici inversi}

% ==============================================================================
\chapter{Trasmissione del calore}\label{ch:heat-transmission}
% ------------------------------------------------------------------------------
\section{Conduzione}
La conduzione è un meccanismo di trasmissione del calore che avviene nei solidi e nei fluidi.

\noindent
La conduzione del calore in un solido è descritto dall'\textbf{equazione di Fourier}
\begin{equation}
    \rho c \dfrac{\partial T}{\partial t} = k \nabla^2 T
\end{equation}
% ------------------------------------------------------------------------------
\section{Convezione}
La convezione è il meccanismo di trasmissione del calore principale nei fluidi.
% ------------------------------------------------------------------------------
\section{Irraggiamento}
L'irraggiamento è un meccanismo di trasmissione del calore nel vuoto.
