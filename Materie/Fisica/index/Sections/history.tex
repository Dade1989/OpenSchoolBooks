
\section{Personaggi}
\begin{itemize}
  \item Galileo
  \begin{itemize}
    \item
    \item
  \end{itemize}
  \item Newton
  \begin{itemize}
    \item
    \item
  \end{itemize}
  \item Snell
  \item Huygens
  \begin{itemize}
    \item
    \item
  \end{itemize}
  \item Hooke
  \item Lagrange: meccanica analitica
  \item Laplace: meccanica analitica
  \item Bernoulli
  \item Cauchy
  \item Navier
  \item Stokes
  \item Young
  \item Fresnel
  \item Thompson
  \item Fourier
  \item Carnot
  \item Joule
  \item Kelvin
  \item Volta
  \item Faraday
  \item Maxwell
  \item Heaviside
  \item Boltzmann
  \item Marie Sklodowska e Pierre Curie
  \item Thomson
  \item Rutherford
  \item Einstein
  \item Planck
  \item Bohr
  \item Heisenberg
  \item Schrodinger
  \item Dirac
  \item Born
  \item \dots
\end{itemize}

% ------------------------------------------------------------------------------
\chapter{Breve storia della fisica}
\section{La fisica prima di Galileo}
\subsection{\dots}
\subsection{Le leggi del moto dei piantei: Copernico e Keplero}

\section{XVI e XVII secolo: la meccanica e l'ottica}
\begin{itemize}
  \item studio della dinamica dei corpi: reso difficile sulla Terra dalla presenza dell'aria e dalla resistenza dei corpi in moto
  \begin{itemize}
    \item Galileo ricava il suo principio di inerzia paragonando la condizione di quiete con la condizione nella stiva di una nave in moto non accelerato (nella stiva non si percepisce la resistenza dell'aria)
    \item ci si dedica allo studio dei corpi celesti, che Newton suppone si muovano nel vuoto
  \end{itemize}
  \item per svolgere lo studio dei corpi celesti servono strumenti ottici, come il cannocchiale
  \begin{itemize}
    \item nel procedimento di perfezionamento del cannocchiale, gli scienziati si trovano a studiare diversi fenomeni ottici
    \item i risultati degli studi che permettono di perfezionare il cannocchiale trovano applicazione anche nel miglioramento dei microscopi
  \end{itemize}
\end{itemize}
\subsection{Il metodo scientifico: Galileo}
\begin{itemize}
  \item Il metodo scientifico
  \item Osservazioni astronomiche
  \item Il principio di inerzia
\end{itemize}
\subsection{La meccanica classica e la legge di gravitazione universale: Newton}
Isaac Newton (1642-1727) era studente al Trinity College di Cambridge, quando nel 1665 l'istituto venne chiuso per il diffondersi della peste, costringendo Newton a proseguire in autonomia i propri studi. Il 1666 viene considerato il suo \textit{annus mirabilis} nel quale approfondì i suoi studi, sviluppando il calcolo infinitesimale -- sviluppato in maniera indipendente dal tedesco Gottfried Wilhelm von Leibniz (1646-1716) --, formulando i tre principi della dinamica classica e la legge di gravitazione universale.

Al ritorno in università a Cambridge nel 1667, Newton venne nominato membro del Trinity College e professore di matematica nel 1669.

Nel 1679, dopo essersi dedicato agli studi sull'ottica, Newton ritornò agli studi sulla gravità per la determinazione delle orbite dei pianeti e la derivazione rigorosa delle leggi di Keplero. Proprio quest'ultima derivazione forniva la risposta al dubbio che si sarebbero posti nel 1684 tre membri della Royal Society: il matematico e architetto Christopher Wren (1632-1723) -- celebre per il suo ruolo nella ricostruzione di Londra dopo il grande incendio del 1666 --, il fisico Robert Hooke (1635-1703) -- curatore degli esperimenti presso la Royal Society, da considerarsi il primo sperimentatore professionista retribuito della storia, celebre per i suoi esperimenti di ottica, il perfezionamento di microscopi e telescopi e la formulazione della legge elastica --, ed Edmond Halley (1656-1742) -- professore a Oxford, famoso astronomo, matematico e scienziato della Terra, al quale fu intitolata la cometa della quale prevedette correttamente il ritorno dopo le osservazioni del 1532, del 1607 di Keplero, e del 1682.

Nel 1687 vennero dati alla stampa i \textit{Philosophiae Naturalis Principia Mathematica}, testo nel quale Newton pubblicava molti dei suoi risultati tenuti fino ad allora inediti, tra i quali i principi della dinamica e la legge di gravitazione universale, usati per svolgere alcuni problemi sul moto dei corpi celesti, spiegare le maree come effetto dell'attrazione gravitazionale della Luna e presentare una prima stima della velocità del suono nell'aria: quest'ultima stima era sbagliata, a causa dell'ipotesi sbagliata da parte di Newton sulla propagazione del suono a temperatura costante.

\subsection{L'ottica}
\begin{itemize}
  \item Newton: prisma e lenti; cannocchiale a riflessione
  \item Huygens
  \item Snell e Fresnel?
\end{itemize}

\section{XVIII e XIX secolo: la termodinamica e l'elettromagnetismo}
\subsection{La meccanica analitica}
\begin{itemize}
  \item Leonhard Euler ()
  \item Lagrange ()
  \item Laplace ()
\end{itemize}
\subsection{La meccanica dei mezzi continui: solidi e fluidi}
Solidi:
\begin{itemize}
  \item Hooke: legge costitutiva di solidi elastici
  \item Euler ()--Bernoulli (): modello di trave elastica
  \item Navier (): equazioni di governo del comportamento elastico delle strutture
\end{itemize}
Fluidi:
\begin{itemize}
  \item Torricelli e Pascal: studi sulla statica dei fluidi
  \item Newton: studi sulla viscosità
  \item Bernoulli, D'Alembert, Lagrange, Laplace, Poisson: studio dei fluidi non viscosi
  \item Hagen, Poiseuille: studio di alcune correnti di fluidi viscosi: correnti in tubi
  \item Navier, Stokes: equazioni di governo dei fluidi
  \item Prandtl, von Karman: studio dello strato limite
  \item Reynolds, Kolmogorov: studio della turbolenza
\end{itemize}
\subsection{La termodinamica}
\begin{itemize}
    \item Hooke () e Boyle () compiono studi sui gas, con l'impiego di pompe ad aria
    \item Black () e Watt () sviluppano i concetti di capacità termica e calore latente all'Università di Glasgow
    \item Sadi Carnot (1796-1832), abbandonata la carriera militare nell'esercito francese, si dedicò agli studi sull'efficienza delle macchine termiche, i cui risultati vengono pubblicati nelle \textit{Riflessioni sulla potenza motrice del fuoco}
    \item I primi due principi della termodinamica vengono formulati nei lavori del 1850 di William Rankine (1820-1872) tra le università di Glasgow ed Edinburgo, Rudolf Clausius (1822-1888) tra Berlino e Zurigo, e William Thompson (Lord Kelvin, 1824-1907) a Glasgow.
    \item Clausius introduce il concetto di entropia nel 1865.
    \item Josiah Willam Gibbs (1839-1903), professore di matematica fisica a Yale, pubblica tre articoli sull'equilibrio e l'evoluzione spontanea dei processi termodinamici, incluse le reazioni chimiche
\end{itemize}
\subsection{La meccanica statistica}
\begin{itemize}
    \item Gli studi di Daniel Bernoulli () pubblicati nel 1738 nel volume \textit{Hydrodynamica} fondano le basi della teoria cinetica dei gas: viene data una descrizione molecolare dei gas; la pressione viene messa in relazione con il numero di urti delle molecole, la temperatura con l'energia cinetica media.
    \item Studi embrionali di termodinamica statistica vengono presentati da Rudolf Clausius e James Clerk Maxwell, che propone la distribuzione delle velocità molecolari
    \item Ludwig Boltzmann () sviluppa la meccanica statistica, riuscendo a spiegare come le leggi della termodinamica classica (descrizione macroscopica del fenomeno) siano un'evidenza del comportamento microscopico di un sistema costituito da un gran numero di particelle: fornisce una definizione statistica dell'entropia, legandola al numero di microstati di un sistema, che può essere interpretata come una misura del disordine del sistema stesso. Nel 1902 J. Willard Gibbs formalizza la meccanica statistica come approccio generale a ogni sistema -- macroscopici o microscopici, gassosi o non gassosi.
\end{itemize}

\subsection{L'elettromagnetismo}

\section{XX secolo: la fisica moderna}
\subsection{La relatività di Einstein}
\paragraph{La relatività speciale o ristretta.}
\paragraph{La relatività generale: una nuova teoria della gravitazione.}

\subsection{La meccanica quantistica}
