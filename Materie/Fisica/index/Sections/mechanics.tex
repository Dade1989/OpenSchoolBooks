
% ==============================================================================
\begin{definition}[Meccanica] La meccanica è la branca della fisica che studia il moto di corpi e le sue cause.
\end{definition}

\section*{Argomenti}
\paragraph{La modellazione.} I corpi vengono rappresentati con \textbf{modelli} astratti o semplificati. I principali modelli che si usano in meccanica sono:
\begin{itemize}
    \item corpo puntiforme
    \item corpo esteso rigido
    \item corpo esteso deformabile
\end{itemize}

\noindent
Si possono riconoscere diversi aspetti della meccanica.

\vspace{8pt}
\paragraph{Cinematica.} La cinematica si occupa della descrizione del moto dei corpi, senza indagare le cause del moto. La cinematica dei sistemi meccanici è descritta completamente quando sono note alcune quantità del sistema in funzione del tempo. Per i diversi modelli di sistemi meccanici queste quantità sono:
\begin{itemize}
    \item punto: posizione
    \item corpo esteso rigido: posizione di un punto materiale e orientazione del corpo
    \item corpo esteso deformabile: posizione di tutti i punti del corpo
\end{itemize}
Calcolando la derivata rispetto al tempo di queste quantità si trovano la velocità e l'accelerazione dei punti, e la velocità e accelerazione angolare dei corpi estesi rigidi.

\vspace{8pt}
\paragraph{Proprietà inerziali e quantità dinamiche.} La massa può essere definita come quantità di materia. La distribuzione di massa nello spazio determina le proprietà inerziali di un sistema.
Le proprietà inerziali per i diversi modelli sono:
\begin{itemize}
    \item punto: massa
    \item corpi estesi (inclusi sistemi di punti): massa totale, momento statico e momento di inerzia
\end{itemize}
Una volta note la cinematica e le proprietà inerziali di un sistema, si possono definire le quantità dinamiche che svolgono un ruolo principale nelle equazioni che governano la dinamica dei corpi. Queste quantità sono:
\begin{itemize}
    \item quantità di moto, $\mathbf{Q}$
    \item momento della quantità di moto rispetto al polo $H$, $\mathbf{\Gamma}_H$
    \item energia cinetica, $K$
\end{itemize}
La loro espressione per un corpo puntiforme è:
\begin{itemize}
    \item quantità di moto, $\mathbf{Q} = m \mathbf{v}$
    \item momento della quantità di moto, $\mathbf{\Gamma}_H = ( \mathbf{r} - \mathbf{r}_H ) \times m \mathbf{v}$
    \item energia cinetica, $K = \frac{1}{2} m |\mathbf{v}|^2$
\end{itemize}
Per definizione (si veda come vengono ricavate le equazioni del moto), queste quantità dinamiche sono \textbf{additive}. Le quantità dinamiche per corpi estesi possono quindi essere ricavate come la somma (grazie all'integrazione, per corpi continui) delle quantità dinamiche elementari delle loro singole parti

\vspace{8pt}
\paragraph{Azioni: forze e momenti.} Le forze e i momenti sono il modello delle azioni agenti sui sistemi e delle interazioni tra di essi.

\vspace{8pt}
\paragraph{Statica.} La statica è caso particolare della dinamica, e si occupa dell'equilibrio dei corpi. Le leggi della statica descrivono le condizioni di equilibrio di un sistema, e si ottengono come caso particolare delle equazioni della dinamica annullando tutte le derivate temporali
\begin{itemize}
    \item corpo puntiforme: % equilibrio delle forze $\mathbf{R}^{ext} = \mathbf{0}$
        \begin{equation}
            \mathbf{0} = \mathbf{R}^{ext} \qquad \qquad \text{(equilibrio delle forze)}
        \end{equation}
    \item corpo esteso: % equilibrio delle forze $\mathbf{R}^{ext} = \mathbf{0}$, e dei momenti $\mathbf{M}_H^{ext} = \mathbf{0}$
        \begin{equation}
        \begin{aligned}
            \mathbf{0} & = \mathbf{R}^{ext}   \qquad \qquad \text{(equilibrio delle forze)} \\
            \mathbf{0} & = \mathbf{M}^{ext}_H \qquad \qquad \text{(equilibrio dei momenti)} \\
        \end{aligned}
        \end{equation}
\end{itemize}

\vspace{8pt}
\paragraph{Dinamica -- Formulazione di Newton.} La dinamica si occupa della descrizione del moto dei corpi e delle sue cause, legando la cinematica e le quantità dinamiche di un sistema alle azioni agenti su di esso: in generale, le azioni agenti su un sistema causano una variazione delle quantità dinamiche.

La dinamica classica può essere formulata a partire dai tre \textbf{principi della dinamica} di Newton:
\begin{enumerate}
    \item principio di inerzia, che riassume la \textbf{relatività di Galileo}: rispetto a un sistema di \textit{riferimento inerziale}, un sistema sul quale non agiscono forze esterne mantiene il suo stato di moto
    \item secondo principio della dinamica: la variazione della quantità di moto di un sistema è uguale all'impulso delle forze esterne agenti sul sistema
        \begin{equation}
            \Delta \mathbf{Q} = \mathbf{I}^{ext} \ ,
        \end{equation}
        o per una variazione in un intervallo di tempo infinitesimo, la derivata nel tempo della quantità di moto è uguale alla risultante dele forze esterne,
        \begin{equation}
            \dot{\mathbf{Q}} = \mathbf{R}^{ext} \ ,
        \end{equation}
    \item principio di \textbf{azione e reazione}: nell'interazione tra due sistemi $i$, $j$, la forza $\mathbf{F}_{ij}$ agente sul sistema $i$ dovuta al sistema $j$ è \textit{uguale e contraria} alla forza agente $\mathbf{F}_{ji}$ sul sistema $j$ dovuta al sistema $i$,
        \begin{equation}
           \mathbf{F}_{ij} = - \mathbf{F}_{ji} \ ,
        \end{equation}
\end{enumerate}
dai quali si possono ricavare le \textbf{equazioni cardinali del moto}, che assumono la stessa forma generale per ogni tipo di sistema meccanico
\begin{enumerate}
    \item l'equazione della quantità di moto coincide con il secondo principio della dinamica
        \begin{equation}
            \dot{\mathbf{Q}} = \mathbf{R}^{ext}
        \end{equation}
    \item l'equazione del momento della quantità di moto mette in relazione, a meno di un termine dipendente dall'eventuale moto del polo $H$, la derivata temporale del momento della quantità di moto con la risultante dei momenti esterni agenti sul sistema
        \begin{equation}
            \dot{\mathbf{\Gamma}}_H = -\dot{\mathbf{x}}_H \times \mathbf{Q} + \mathbf{M}_H^{ext}
        \end{equation}
    \item il teorema dell'energia cinetica, mette in relazione la derivata temporale dell'energia cinetica con la potenza totale delle forze agenti sul sistema
        \begin{equation}
            \dot{K} = P^{tot}
        \end{equation}
\end{enumerate}

\section*{Strumenti matematici}

% ==============================================================================
\chapter{Cinematica}
% ==============================================================================
{\color{red} La cinematica studia il moto dei corpi senza preoccuparsi delle cause del moto \dots}
% ==============================================================================

\begin{definition}[Cinematica] La cinemcatica è quel ramo della meccanica che si occupa di descrivere il moto dei corpi, indipendentemente dalle cause del moto.
\end{definition}
Il moto dei corpi viene descritto usando i concetti di \textbf{tempo} e \textbf{spazio}.

% ------------------------------------------------------------------------------
\section{Cinematica del punto materiale}
\begin{definition}[Posizione]
Una volta definito un sistema di riferimento, si può definire la posizione di un punto $P$ con il raggio vettore $\mathbf{r}_P$ che unisce l'origine del sistema di riferimento con la posizione del punto $P$.
\end{definition}

\begin{definition}[Velocità]
    La velocità di un punto $P$ rispetto a un sistema di riferimento viene definita come la derivata nel tempo del raggio vettore $\mathbf{r}_P$,
    \begin{equation}
        \mathbf{v}_P = \dfrac{d}{dt}\mathbf{r}_P  =: \dot{ \mathbf{r}}_P  \ .
    \end{equation}
\end{definition}

\begin{definition}[Accelerazione]
    L'accelerazione di un punto $P$ rispetto a un sistema di riferimento viene definita come la derivata nel tempo della velocità  $\mathbf{v}_P$ del punto, equivalente alla derivata seconda della posizione
    \begin{equation}
        \mathbf{a}_P = \dot{ \mathbf{v}}_P = \ddot{ \mathbf{r}}_P \ .
    \end{equation}
\end{definition}

% ------------------------------------------------------------------------------
\section{Cinematica di un sistema esteso rigido}
\begin{definition}[Vincolo di moto rigido] Un corpo viene definito rigido se la distanza di ogni coppia di suoi punti rimane costante nel moto,
    \begin{equation}
        d_{PQ} := | \mathbf{r}_P(t) - \mathbf{r}_Q(t) | = \text{cost} \ .
    \end{equation}
\end{definition}
Questa definizione, implica che rimangano costanti le distanze e gli angoli tra punti materiali di un corpo che compie un atto di moto rigido.

\begin{equation}
    \left( \mathbf{v}_P - \mathbf{v}_Q \right) \perp \left( \mathbf{r}_P - \mathbf{r}_Q \right)
\end{equation}
\begin{equation}
\begin{aligned}
    0 & = \dfrac{d}{dt} | \mathbf{r}_P - \mathbf{r}_Q |^2 = \\
    & = \dfrac{d}{dt} \left[ \left( \mathbf{r}_P - \mathbf{r}_Q \right) \cdot \left( \mathbf{r}_P - \mathbf{r}_Q \right) \right] = \\
    & = 2 \dfrac{d}{dt} \left( \mathbf{r}_P - \mathbf{r}_Q \right) \cdot \left( \mathbf{r}_P - \mathbf{r}_Q \right) = 2 \left( \mathbf{v}_P - \mathbf{v}_Q \right) \cdot \left( \mathbf{r}_P - \mathbf{r}_Q \right)
\end{aligned}
\end{equation}

\begin{definition}[Velocità angolare di un corpo rigido] Si può dimostrare che esiste un vettore $\mathbf{\omega}$, definito \textbf{velocità angolare}, tale che vale la relazione 
\begin{equation}
    \left( \mathbf{v}_P - \mathbf{v}_Q \right) = \boldsymbol{\omega} \times \left( \mathbf{r}_P - \mathbf{r}_Q \right) \ ,
\end{equation}
tra la differenze di velocità e di posizione tra ogni coppia di punti $P$, $Q$ del corpo rigido.
\end{definition}

\subsection{Cinematica nel piano}
\subsection{Cinematica nello spazio -- cenni}

% ------------------------------------------------------------------------------
\section{Cinematica di un sistema esteso deformabile -- cenni}

% ------------------------------------------------------------------------------
\section{Vincoli}\label{mechanics:kinematics:constraints}

% ==============================================================================
\chapter{Proprietà inerziali e quantità dinamiche}
% ==============================================================================
La massa può essere definita come la quantità di materia \dots
% ==============================================================================

\section{Proprietà inerziali}

\begin{definition}[Massa] La massa viene definita come la quantità di materia.
\end{definition}
{\color{red} Per la definizione operativa della massa in dinamica si rimanda al capitolo (\ref{mechanics:dynamics}).}

\subsection{Centro di massa}
\begin{equation}
    \mathbf{r}_G = \dfrac{1}{m} \sum_{i=1}^n m_i \mathbf{r}_i \ .
\end{equation}
\subsection{Momento statico}
\subsection{Momento di inerzia}

\section{Quantità dinamiche}
In questa sezione vengono definite alcune quantità dinamiche che risulteranno utili nello studio della dinamica dei corpi nel capitolo \ref{mechanics:dynamics}. In particolare, vengono definiti:
\begin{itemize}
  \item la quantità di moto
  \item il momento della quantità di moto
  \item l'energia cinetica
\end{itemize}
per i sistemi meccanici modellabili come punto materiale, insieme di punti materiali, e corpi rigidi.
Queste quantità sono \textbf{grandezze additive}.

\subsection{Punto materiale}
\subsubsection{Quantità di moto}
\begin{equation}
  \mathbf{Q} = m \mathbf{v}
\end{equation}
\subsubsection{Momento della quantità di moto}
\begin{equation}
    \mathbf{\Gamma}_H = \left(\mathbf{r} - \mathbf{r}_H \right) \times \mathbf{Q}
\end{equation}
\subsubsection{Energia cinetica}
\begin{equation}
    K = \dfrac{1}{2} m | \mathbf{v} |^2
\end{equation}

\subsection{Sistema di punti materiali}
\subsubsection{Quantità di moto}
\begin{equation}
    \mathbf{Q} = \sum_{i} \mathbf{Q}_i = \sum_i m_i \mathbf{v}_i
\end{equation}
\subsubsection{Momento della quantità di moto}
\begin{equation}
    \mathbf{\Gamma}_H = \sum_{i} \mathbf{\Gamma}_{H,i} = \sum_i \left( \mathbf{r}_i - \mathbf{r}_H \right) \times \mathbf{Q}_i
\end{equation}
\subsubsection{Energia cinetica}
\begin{equation}
    K = \sum_i K_i = \sum_i \dfrac{1}{2} m_i | \mathbf{v}_i |^2
\end{equation}

\subsubsection{Corpi rigidi}
\subsubsection{Momento della quantità di moto}
\subsubsection{Energia cinetica}

\subsection{Sistema di punti materiali}

% ==============================================================================
\chapter{Azioni}
% ==============================================================================
% ==============================================================================
\begin{definition}[Forza] Una forza è un'entità fisica vettoriale che è in grado di cambiare lo stato del moto di un sistema.
\end{definition}

% ------------------------------------------------------------------------------
\section{Esempi di forza}
\subsection{Forza peso in prossimità della superficie terrestre}
In prossimità della superficie terrestre, su un corpo di massa $m$ agisce la sua forza peso,
\begin{equation}
    \mathbf{F} = m \mathbf{g} \ ,
\end{equation}
dove $\mathbf{g}$ rappresenta il campo gravitazionale nei pressi della superficie terrestre ed è diretto verso il centro {\color{red} (di massa?)} della Terra. La sua intensità nei pressi della superficie terrestre è circa $9.8 \frac{m}{s^2}$.

In una regione sufficientemente limitata dello spazio nella quale si può approssimare la superficie terrestre come piatta e piana, il campo di gravità e il peso dei corpi hanno quindi direzione verticale verso il basso.

\subsection{Legge di gravitazione universale di Newton}
Secondo la legge di gravitazione universale di Newton, due corpi di massa $m_1$, $m_2$ si attraggono con una forza di intensirà proporzionale al prodotto delle masse, inversamente proporzionale al quadrato della distanza tra i centri di massa dei due corpi e con una direzione lungo la congiunngente tra i due centri di massa.

La forza agente sul corpo di massa $m_2$, dovuta all'attrazione gravitazionale del corpo di massa $m_1$, è quindi
\begin{equation}
    \mathbf{F}_{21} = G \dfrac{m_1 m_2}{|\mathbf{r}_{21}|^2} \mathbf{\hat{r}}_{21} \ ,
\end{equation}
essendo $G$ la \textbf{costante di gravitazione universale},
\begin{equation}
    G = 6.673 \cdot 10^{-11} \dfrac{m^3}{kg \ s^2} \ ,
\end{equation}
il vettore $\mathbf{r}_{21} = \mathbf{r}_1 - \mathbf{r}_2$ il vettore che congiunge il corpo 1 con il corpo 2, $\mathbf{\hat{r}}_{21}$ il versore in quella direzione. La formula può quindi essere riscritta con espressioni equivalenti
\begin{equation}
    \mathbf{F}_{21} = G \dfrac{m_1 m_2}{|\mathbf{r}_{21}|^2} \mathbf{\hat{r}}_{21} = 
    G m_1 m_2 \dfrac{\mathbf{r}_1 - \mathbf{r}_2}{|\mathbf{r}_1 - \mathbf{r}_2|^3} \ .
\end{equation}

\begin{example}[Attrazione gravitazionale terrestre nei pressi della superficie terrestre] Prendendo il raggio e la massa della Terra, 
    \begin{equation}
        R_T = 6.378 \cdot 10^6 \, m \qquad , \qquad  m_T = 5.972 \cdot 10^{24} \, kg
    \end{equation}
    possiamo calcolare l'intensità della forza che percepisce un corpo di massa $m$ nei pressi della superficie terrestre come
    \begin{equation}
        F = \dfrac{G \, m_T}{R_T^2} m = \dfrac{ 6.673 \cdot 10^{-11} \frac{m^3}{kg \ s^2} \cdot  5.972 \cdot 10^{24} \, kg }{\left(6.378 \cdot 10^6 \, m \right)^2} = m \cdot 9.797 \dfrac{m}{s^2} \ .
    \end{equation}
    Abbiamo ritrovato il valore della forza per unità di massa uguale al valore dell'intensità del campo gravitazionale nei pressi della superficie terrestre.
\end{example}

\begin{example}[Attrazione gravitazionale terrestre sulla ISS]
\end{example}

\subsection{Molla lineare e forza elastica}
Una molla ideale lineare senza massa ha una \textbf{legge costitutiva elastica} che rappresenta un legame proporzionale tra la forza impressa agli estremi della molla e l'allungamento della molla stessa,
\begin{equation}
    F = k \, \Delta x \ ,
\end{equation}
essendo $k$ la costante elastica della molla, che ha le dimensioni fisiche $\frac{N}{m} = \frac{kg}{s^2}$, e il cui valore dipende dalla struttura della molla.
.
Le forze applicate ai due estremi sono di intensità uguale e verso contrario, e hanno la direzione dell'asse della molla. In forma vettoriale, assumento una lunghezza a riposo nulla, e indicando con $\mathbf{r}_1$, $\mathbf{r}_2$ le posizioni degli estremi della molla, le forze agenti sugli estremi si possono scrivere come
\begin{equation}
    \mathbf{F}_{12} = - \mathbf{F}_{21} =  k \left( \mathbf{r}_2 - \mathbf{r}_1 \right) \ ,
\end{equation}
essendo la forza $\mathbf{F}_{12}$ la forza agente sul punto 1 a causa della molla che la collega al punto 2.

\subsection{Smorzatore lineare e forza viscosa}
Uno smorzatore ideale lineare senza massa ha una \textbf{legge costitutiva viscosa} che rappresenta un legame proporzionale tra la forza impressa agli estremi dello smorzatore e la velocità di allungamento dello smorzatore,
\begin{equation}
    F = c \, \Delta v = c \, \Delta \dot{x} \ ,
\end{equation}
essendo $c$ la costante viscosa (di smorzamento) dello smorzatore, che ha le dimensioni fisiche $\frac{N}{\frac{m}{s}} = \frac{kg}{s}$, e il cui valore dipende dalla struttura dello smorzatore.
.
Le forze applicate ai due estremi sono di intensità uguale e verso contrario, e hanno la direzione dell'asse dello smorzatore. In forma vettoriale, indicando con $\mathbf{r}_1$, $\mathbf{r}_2$ le posizioni degli estremi della molla, le forze agenti sugli estremi si possono scrivere come
\begin{equation}
    \mathbf{F}_{12} = - \mathbf{F}_{21} = c \left( \mathbf{v}_2 - \mathbf{v}_1 \right)  = c \left( \dot{\mathbf{r}}_2 - \dot{\mathbf{r}}_1 \right) \ ,
\end{equation}
essendo la forza $\mathbf{F}_{12}$ la forza agente sul punto 1 a causa dello smorzatore che lo collega al punto 2.
\subsection{Reazioni vincolari}
{\color{red} Fare riferimento alla sezione \ref{mechanics:kinematics:constraints} e rimandare a una sezione del capitolo della dinamica dove verrà spiegato il calcolo delle reazioni vincolari}

\subsection{Forza di attrito}
\subsubsection{Attrito statico}
La forza di attrito statico agisce tra due corpi tra i quali non c'è moto relativo, in direzione tangenziale alla superficie del vincolo. Il suo valore è determinato dalla condizione di equilibrio 
\begin{equation}
    \mathbf{F}^s_{\mu} = F^s_{\mu} \mathbf{\hat{t}} \ , \qquad \text{con } |\mathbf{F}^s_{\mu}| \le \mu_s N \ ,
\end{equation}
essendo $\mu_s$ il coefficiente di attrito statico, tipico delle superfici (e del loro stato: materiale, rugosità e finiture superficiali, lubrificazione, superficie sporche,\dots) e $N$ il modulo della reazione normale.

\subsubsection{Attrito dinamico}
La forza di attrito dinamico agisce tra due corpi a contatto in moto relativo con velocità $\mathbf{v}^{rel}$, $|\mathbf{v}^{rel}| \ne 0$, in direzione tangenziale alla superficie di contatto. Il modello di attrito di Coulomb prevede l'espressione
\begin{equation}
    \mathbf{F}^d_{\mu} = - \mu_d N \dfrac{\mathbf{v}^{rel}}{|\mathbf{v}^{rel}|}
\end{equation}
essendo $\mu_d$ il coefficiente di attrito dinamico, tipico delle superfici (e del loro stato: materiale, rugosità e finiture superficiali, lubrificazione, superficie sporche,\dots) e $N$ il modulo della reazione normale.

\subsubsection{Attrito volvente}
{\color{red}
\begin{itemize}
    \item Modello dell'azione che fa rallentare il moto di una corpo circolare in rotolamento puro su una superficie piana.Perché no, non è l'attrito statico
    \item Rimandare a esercizio su rotolamento puro di un disco, con e senza attrito volvente
\end{itemize}
}

% ------------------------------------------------------------------------------
\section{Momento di una forza}
\begin{definition}[Momento di una forza] Il momento di una forza rispetto a un \textbf{polo} $H$, $\mathbf{M}_H$, viene definito coem il prodotto vettoriale tra il raggio $\Delta \mathbf{r}_{HP} = \mathbf{r}_P - \mathbf{r}_H$ tra il polo e il punto di applicazione e la forza $\mathbf{F}$,
    \begin{equation} \mathbf{M}_H = \Delta \mathbf{r}_{HP} \times \mathbf{F} \ .
    \end{equation}
\end{definition}

% ------------------------------------------------------------------------------
\section{Impulso di una forza}
\begin{definition}[Impulso elementare di una forza]
    L'impulso elementare $\delta \mathbf{I}$ di una forza $\mathbf{F}$ viene definito come il prodotto della forza e l'intervallo elementare di tempo $dt$ per il quale agisce la forza,
    \begin{equation}
        \delta \mathbf{I} = \mathbf{F} dt \ .
    \end{equation}
\end{definition}
\begin{definition}[Impulso di una forza] Viene definito come la somma di tutti gli impulsi elementari tra due istanti desiderati,  $t \in [t_0, t_1]$, e al limite si riduce all'integrale di Riemann
    \begin{equation}
        \mathbf{I} = \int_{t=t_0}^{t_1} \delta \mathbf{I} = \int_{t=t_0}^{t_1} \mathbf{F} dt \ .
    \end{equation}
\end{definition}

% ------------------------------------------------------------------------------
\section{Lavoro di una forza}
\begin{definition}[Lavoro elementare di una forza] Il lavoro elementare $\delta L$ di una forza $\mathbf{F}$ è il prodotto scalare tra la forza e lo spostamento elementare $d \mathbf{r}$,
    \begin{equation}
        \delta L = \mathbf{F} \cdot d \mathbf{r} \ .
    \end{equation}
\end{definition}
Il lavoro di una forza è la somma dei lavori elementari. Se si suddivide la storia degli spostamenti in una somma di spostamenti incrementali,
\begin{equation}
  \Delta \mathbf{r} = \sum_{i=1}^{n} \Delta \mathbf{r}_i \ ,
\end{equation}
e si considera la forza applicata costante (in intensità e direzione) a tratti su ogni spostamento incrementale $\Delta \mathbf{r}_i$, il lavoro della forza viene definito come la somma dei lavori incrementali, $\Delta L_i = \mathbf{F}_i \cdot \Delta \mathbf{r}_i$
\begin{equation}
  L = \sum_{i=1}^{N} \mathbf{F}_i \cdot \Delta \mathbf{r}_i \ .
\end{equation}
\begin{definition}[Lavoro di una forza]
Se si fa tendere a zero la lunghezza degli spostamenti incrementali, il lavoro di una forza viene definito come l'integrale (di linea) di Riemann
\begin{equation}
    L = \int_{\gamma(\mathbf{r})} \mathbf{F} \cdot d \mathbf{r} =  \int_{\gamma(\mathbf{r})} \mathbf{F} \cdot \mathbf{\hat{t}} \ ,
\end{equation}
avendo indicato con $\gamma(\mathbf{r})$ la curva nello spazio percorsa dal corpo, sulla quale avvengono gli spostamenti elementari.
\end{definition}

% ------------------------------------------------------------------------------
\section{Campo di forze}
\begin{definition}[Campo di forze - definizione operativa]
\end{definition}

\subsection{Campi di forze centrali}
\begin{definition}[Campo di forza centrale] Un campo di forza centrale è un campo di forze che ha direzione radiale rispetto a un punto, il centro.

\noindent
Ponendo l'origine del sistema di riferimento nel centro, è possibile scrivere l'espressione di un campo centrale come
    \begin{equation}
        \mathbf{F}(\mathbf{r}) = F(\mathbf{r}) \mathbf{\hat{r}} \ .
    \end{equation}
\end{definition}

\begin{example}[Legge di gravitazione universale] La forza gravitazionale agente su un corpo puntiforme di massa $m$ con posizione $\mathbf{r}$ dovuta a un corpo di massa $M$ in corrispondenza del quale viene posto l'origine delle coordinate, ha l'espressione
    \begin{equation}
        \mathbf{F}(\mathbf{r}) = - G \dfrac{m \, M}{r^2} \mathbf{\hat{r}} = - G m M \dfrac{\mathbf{r}}{|\mathbf{r}|^3} \ .
    \end{equation}
\end{example}

\begin{example}[Forza elastica da una molla lineare] La forza elastica agente sul punto $\mathbf{r}$ collegato all'origine con una molla lineare di costante elastica $k$ e lunghezza a riposo nulla, ha l'espressione
    \begin{equation}
        \mathbf{F}(\mathbf{r}) = - k \, \mathbf{r}
    \end{equation}
\end{example}


\subsection{Campi di forze conservativi}
\begin{definition}[Campo di forza conservativo] Un campo di forza è conservativo in una regione dello spazio $\Omega$ se l'integrale del lavoro lungo un percorso $\gamma$ qualsiasi in quella regione non dipende dal percorso ma solo dai suoi estremi.
\end{definition}
In questo caso, se il percorso $\gamma$ va dal punto $\mathbf{r}_A$ al punto $\mathbf{r}_B$, si può scrivere
\begin{equation}
    L = \int_{\gamma} \mathbf{F} \cdot \mathbf{\hat{t}} = \int_{\mathbf{r}_A}^{\mathbf{r}_B} \mathbf{F} \cdot \mathbf{\hat{t}} =: L_{A \rightarrow B}
\end{equation}
\begin{definition}[Energia potenziale] Per un campo di forze conservative è possibile definire una funzione scalare dello spazio, $V(\mathbf{r})$, chiamata \textbf{energia potenziale}, che permette di calcolare il lavoro del campo di forze conservative come
\begin{equation}
    L_{A \rightarrow B} = - \Delta V_{A \rightarrow B} = - \left[ V(\mathbf{r}_B) - V(\mathbf{r}_A) \right]
\end{equation}
\end{definition}

{\color{red} EXTRA?
\begin{itemize}
    \item dall'indipendendza del lavoro dal percorso, è possibile scrivere il campo di forze come (meno) il gradiente della funzione energia potenziale, $\mathbf{F}(\mathbf{r}) = - \nabla V(\mathbf{r})$
\begin{equation}
    \int_{\mathbf{r}_A}^{\mathbf{r}_B} \mathbf{F} \cdot \mathbf{r} = L = -\left[  V(\mathbf{r}_B) - V(\mathbf{r}_A) \right] = - \int_{\mathbf{r}_A}^{\mathbf{r}_B} dV = -\int_{\mathbf{r}_A}^{\mathbf{r}_B} \nabla V \cdot d \mathbf{r} 
\end{equation}
\begin{equation}
    \rightarrow \qquad \mathbf{F}(\mathbf{r}) = - \nabla V(\mathbf{r}) \ .
\end{equation}
    \item poiché si può scrivere il campo di forze come un gradiente, per le identità vettoriali ($\nabla \times \nabla f \equiv 0$, $\forall f(\mathbf{r})$), si può dire che i campi di forze conservativi sono irrotazionali, $\nabla \times \mathbf{F} = \mathbf{0}$.
\end{itemize}
}

\begin{example}[Energia potenziale di campi uniformi]
    \begin{equation}
        \mathbf{F}(\mathbf{r}) = \mathbf{F} \qquad \rightarrow \qquad V(\mathbf{r}) = - \mathbf{F} \cdot ( \mathbf{r} - \mathbf{r}_0 ) + V_0
    \end{equation}
\end{example}
\begin{example}[Energia potenziale di campi centrali]
    \begin{equation}
        \mathbf{F}(\mathbf{r}) = F(\mathbf{r}) \mathbf{\hat{r}} = - \nabla V(\mathbf{r}) = - \dfrac{\partial V}{\partial r} \mathbf{\hat{r}}
    \end{equation}
    Se il campo di forze ha simmetria radiale, il campo di forze e l'energia potenziale dipende solo dalla distanza dall'origine $r$, e vale
    \begin{equation}
        v'(r) = - F(r) \qquad \rightarrow \qquad V(r) - V(r_0) = - \int_{r_0}^{r} F(t) dt \ ,
    \end{equation}
    dove {\color{red} $r_0$, $V(r_0)$ si usano per fissare il valore della funzione potenziale; \textbf{la funzione potenziale è definita a meno di costanti additive}}
\end{example}
\begin{example}[Energia potenziale del campo gravitazionale uniforme] Definendo la direzione verticale con il versore $\mathbf{\hat{z}}$ diretto verso l'alto, la forza peso su un corpo di massa $m$ vale $\mathbf{F} = m \mathbf{g} = - m g \mathbf{\hat{z}}$, e quindi l'energia potenziale gravitazionale in prossimità della superficie terrestre vale
    \begin{equation}
        V(\mathbf{r}) - V(\mathbf{r}_0) = m g \mathbf{\hat{z}} \cdot ( \mathbf{r} - \mathbf{r}_0 ) = m g ( z - z_0 ) \ .
    \end{equation}
\end{example}
\begin{example}[Energia potenziale del campo gravitazionale centrale] Per la legge di gravitazione universale, $F(r) = -\frac{G M m}{r^2}$, l'energia potenziale vale
    \begin{equation}
        V(r) - V(r_0) = G M m \int_{r_0}^{r} t^{-2} dt = G M m \left[ -\dfrac{1}{r} + \dfrac{1}{r_0} \right] \ .
    \end{equation}
Si è soliti definire la costante additiva arbitraria nell'energia potenziale gravitazionale con la condizione $\lim_{r \rightarrow \infty} V(r) = 0$, e quindi
    \begin{equation}
        V(r) = - \dfrac{G M m}{r} \ .
    \end{equation}
\end{example}
\begin{example}[Energia potenziale del campo elastico centrale] Per la forza elastica dovuta a una molla lineare con lunghezza a riposo nulla, $F(r) = - k r$, l'energia potenziale vale
    \begin{equation}
        V(r) - V(r_0) = k \int_{r_0}^{r} t \, dt = \dfrac{1}{2} k \left( r^2 - r_0^2 \right) \ .
    \end{equation}
Si è soliti definire la costante additiva arbitraria nell'energia elastica con la condizione $V(0) = 0$, e quindi
    \begin{equation}
        V(r) = \dfrac{1}{2} \, k \, r^2 \ .
    \end{equation}
\end{example}

% ==============================================================================
\chapter{Statica}
% ==============================================================================
% ==============================================================================

\section{Condizioni di equilibrio}
\begin{definition}[Condizioni di equilibrio per un punto materiale]
    \begin{equation}
        \sum_{i=1}^{n} \mathbf{F}_i = \mathbf{0}
    \end{equation}
\end{definition}

\begin{definition}[Condizioni di equilibrio per un corpo esteso]
    \begin{equation}
    \begin{aligned}
        \sum_{i=1}^{n} \mathbf{F}_i & = \mathbf{0} \\
        \sum_{i=1}^{n} \mathbf{M}_{H,i} & = \mathbf{0}
    \end{aligned}
    \end{equation}
\end{definition}

\section{Esempi}


% ==============================================================================
\chapter{Dinamica}\label{mechanics:dynamics}
% ==============================================================================
La dinamica studia il moto dei corpi e le sue cause \dots
% ==============================================================================
\section{Principi della dinamica di Newton}
\begin{enumerate}
  \item principio di inerzia
  \item secondo principio della dinamica
  \item principio di azione e reazione
\end{enumerate}
\subsection{Primo principio della dinamica}
Un corpo imperturbato, sul quale agisce un sistema di forza dalla risultante nulla, rimane nello stato di quiete o in moto rettilineo uniforme rispetto a un sistema di \textbf{riferimento inerziale}.

\noindent
{\color{red}{Cosa intendiamo per sistema di riferimento inerziale?}}

\subsection{Secondo principio della dinamica}
\begin{equation}
    \Delta \mathbf{Q} = \mathbf{I}^{ext} \ ,
\end{equation}
o in forma differenziale
\begin{equation}
    \dfrac{d}{dt}\mathbf{Q} = \mathbf{R}^{ext}
\end{equation}

\subsection{Terzo principio della dinamica}

% ------------------------------------------------------------------------------
\section{Equazioni cardinali della dinamica}
% ------------------------------------------------------------------------------
Dai principi della dinamica è possibile ricavare tre equazioni cardinali della dinamica, valide per ogni sistema meccanico.
\paragraph{Prima equazione cardinale -- quantità di moto}
\begin{equation}
    \dot{\mathbf{Q}} = \mathbf{R}^{ext}
\end{equation}
\paragraph{Seconda equazione cardinale -- momento della quantità di moto}
\begin{equation}
    \dot{\boldsymbol{\Gamma}}_H = - \dot{\mathbf{x}}_H \times \mathbf{Q} + \mathbf{M}_H^{ext}
\end{equation}
\paragraph{Terza equazione cardinale -- energia cinetica}
\begin{equation}
    \dot{K} = P^{tot}
\end{equation}
Nelle sezioni successive, queste equazioni vengono ricavate dai principi della dinamica per i diversi modelli dei sistemi meccanici.

\subsection{Moto di un punto materiale}
\paragraph{Prima equazione cardinale -- quantità di moto}
\begin{equation}
    \dfrac{d}{dt} \mathbf{Q} = \mathbf{R}^{ext}
\end{equation}

\paragraph{Seconda equazione cardinale -- momento della quantità di moto}
\begin{equation} \mathbf{M}_H = \mathbf{r} \times \mathbf{Q} \end{equation}
\begin{equation}
\begin{aligned}
    \dfrac{d}{dt} \mathbf{\Gamma}_H
    & = \dfrac{d}{dt} \left( (\mathbf{r} - \mathbf{r}_H ) \times \mathbf{Q} \right) = \\
    & = \dot{\mathbf{r}} \times \mathbf{Q} - \dot{\mathbf{r}}_H \times \mathbf{Q} + \mathbf{r} \times \dot{\mathbf{Q}} = \\
    & = \underbrace{\dot{\mathbf{r}} \times m \dot{\mathbf{r}}}_{=\mathbf{0}} - \dot{\mathbf{r}}_H \times \mathbf{Q} + \mathbf{r} \times \mathbf{R}^{ext} = - \dot{\mathbf{r}}_H \times \mathbf{Q} + \mathbf{M}_H^{ext} \ .
\end{aligned}
\end{equation}
\paragraph{Terza equazione cardinale -- energia cinetica}

\begin{equation} K = \dfrac{1}{2} m |\mathbf{v}|^2 = \dfrac{1}{2} m \mathbf{v} \cdot \mathbf{v} \ .
\end{equation}
\begin{equation}
\begin{aligned}
    \dfrac{d}{dt} K & = \underbrace{m \dot{\mathbf{v}}}_{=\dot{\mathbf{Q}}} \cdot \mathbf{v} = \\
    & = \dot{\mathbf{Q}} \cdot \mathbf{v} = \\
    & = \mathbf{R}^{ext} \cdot \mathbf{v} = P^{ext} = P^{tot}
\end{aligned}
\end{equation}

% ------------------------------------------------------------------------------
\subsection{Moto di un sistema di punti materiali}
\paragraph{Prima equazione cardinale}
L'equazione cardinale per il punto $i$-esimo,
\begin{equation}
\begin{aligned}
    m_i \ddot{\mathbf{r}}_i & = \mathbf{R}^{ext,i}_i = \\
    & = \mathbf{R}^{ext}_i + \mathbf{R}^{int}_i = \\
    & = \mathbf{R}^{ext}_i + \sum_{j \ne i} \mathbf{F}_{ij} \\
\end{aligned}
\end{equation}
Sommando le equazioni di tutti i punti, si ottiene
\begin{equation}
    \underbrace{\sum_{i} m_i \ddot{\mathbf{r}}_i}_{m \ddot{\mathbf{r}}_G = \dot{\mathbf{Q}} } = \underbrace{\sum_{i} \mathbf{R}^{ext}_i}_{\mathbf{R}^{ext}} + \underbrace{\sum_{i} \sum_{j \ne i} \mathbf{F}_{ij}}_{=\mathbf{0}} \\
\end{equation}
\begin{equation}
    \dot{\mathbf{Q}} = \mathbf{R}^{ext} 
\end{equation}

\paragraph{Seconda equazione cardinale}
\begin{equation}
\begin{aligned}
    & \dfrac{d}{dt} \left( (\mathbf{r}_i-\mathbf{r}_H) \times m_i \dot{\mathbf{r}}_i \right) = (\mathbf{r}_i-\mathbf{r}_H) \times \mathbf{R}^{ext,i}_i \\
    & ( \dot{\mathbf{r}}_i - \dot{\mathbf{r}}_H ) \times m_i \dot{\mathbf{r}}_i + (\mathbf{r}_i-\mathbf{r}_H) \times m_i \ddot{\mathbf{r}}_i =  (\mathbf{r}_i-\mathbf{r}_H) \times \left( \mathbf{R}^{ext}_i + \sum_{j\ne i} \mathbf{F}_{ij} \right) \\
    & \sum_i (\mathbf{r}_i-\mathbf{r}_H) \times m_i \ddot{\mathbf{r}}_i = \dot{\mathbf{r}}_H \times \mathbf{Q} + \mathbf{M}_H^{ext}
\end{aligned}
\end{equation}
%
Per sistemi rigidi, per i quali vale $\mathbf{v}_i - \mathbf{v}_j = \mathbf{\Omega} \times (\mathbf{r}_i - \mathbf{r}_j)$
\begin{equation}
\begin{aligned}
    \mathbf{\Gamma}_H & = \sum_{i} \left( \mathbf{r}_i - \mathbf{r}_H \right) \times m_i \mathbf{v}_i = \\
    & = \sum_i \left( \mathbf{r}_i - \mathbf{r}_G + \mathbf{r}_G - \mathbf{r}_H \right) \times m_i \left[ \mathbf{v}_G + \mathbf{\Omega} \times (\mathbf{r}_i - \mathbf{r}_G) \right] = \\
    & = \underbrace{\sum_i m_i (\mathbf{r}_i - \mathbf{r}_G)}_{m \mathbf{r}_G - m \mathbf{r}_G=\mathbf{0}} \times \mathbf{v}_G +
        \underbrace{\sum_i m_i}_{=m} (\mathbf{r}_G - \mathbf{r}_H) \times \mathbf{v}_G + \\
    & \quad + (\mathbf{r}_G - \mathbf{r}_0) \times \bigg( \mathbf{\Omega} \times \underbrace{\sum_i m_i (\mathbf{r}_i - \mathbf{r}_G)}_{m \mathbf{r}_G - m \mathbf{r}_G=\mathbf{0}} \bigg)
    \underbrace{- \sum_i m_i (\mathbf{r}_i - \mathbf{r}_G ) \times [ (\mathbf{r}_i - \mathbf{r}_G ) \times }_{=\mathbb{I}_G} \mathbf{\Omega} ]
\end{aligned}
\end{equation}
\begin{equation}
    \mathbf{\Gamma}_H = (\mathbf{r}_G - \mathbf{r}_H) \times \mathbf{Q} + \mathbb{I}_G \cdot \mathbf{\Omega}
\end{equation}
\begin{equation}
\begin{aligned}
    \dfrac{d}{dt} \left( (\mathbf{r}_G - \mathbf{r}_H) \times \mathbf{Q}  \right) & = ( \mathbf{v}_G - \dot{\mathbf{r}}_H ) \times \mathbf{Q} + (\mathbf{r}_G - \mathbf{r}_H) \times \dot{\mathbf{Q}} = \\
    & = - \dot{\mathbf{r}}_H \times \mathbf{Q} + (\mathbf{r}_G - \mathbf{r}_H) \times \dot{\mathbf{Q}}
\end{aligned}
\end{equation}
\begin{equation}
\begin{aligned}
    \dfrac{d}{dt} \left( \mathbb{I}_G \cdot \mathbf{\Omega} \right) & =
    \dfrac{d}{dt} \left( - \sum_i m_i (\mathbf{r}_i - \mathbf{r}_G ) \times \left[ (\mathbf{r}_i - \mathbf{r}_G ) \times \mathbf{\Omega} \right]  \right) = \\
    & = - \sum_i  m_i (\mathbf{v}_i - \mathbf{v}_G ) \times \left[ (\mathbf{r}_i - \mathbf{r}_G ) \times \mathbf{\Omega} \right] + \\
    & \quad - \sum_i  m_i (\mathbf{r}_i - \mathbf{r}_G ) \times \left[ (\mathbf{v}_i - \mathbf{v}_G ) \times \mathbf{\Omega} \right] + \\
    & \quad - \sum_i  m_i (\mathbf{r}_i - \mathbf{r}_G ) \times \left[ (\mathbf{r}_i - \mathbf{r}_G ) \times \dot{\mathbf{\Omega}} \right] = \\
    & = - \sum_i  m_i \underbrace{\left[ (\mathbf{r}_i - \mathbf{r}_G ) \times \mathbf{\Omega} \right]  \times \left[ (\mathbf{r}_i - \mathbf{r}_G ) \times \mathbf{\Omega} \right]}_{= \mathbf{0}} + \\
    & \quad - \sum_i  m_i (\mathbf{r}_i - \mathbf{r}_G ) \times \left[ (\mathbf{v}_i - \mathbf{v}_G ) \times \mathbf{\Omega} \right] + \\
    & \quad + \mathbb{I}_G \cdot \dot{\mathbf{\Omega}} \\
    & = + \sum_i  m_i \mathbf{\Omega} \times [ (\mathbf{r}_i - \mathbf{r}_G) \times \underbrace{(\mathbf{v}_i - \mathbf{v}_G)}_{\mathbf{\Omega} \times (\mathbf{r}_i - \mathbf{r}_G) } ] + \\
    & \quad + \sum_i  m_i \underbrace{(\mathbf{v}_i - \mathbf{v}_G) \times [ \mathbf{\Omega} \times (\mathbf{r}_i - \mathbf{r}_G ) ]}_{= \mathbf{0}}  + \\
    & \quad + \mathbb{I}_G \cdot \dot{\mathbf{\Omega}} \\
    & = \mathbf{\Omega} \times \left( - \sum_i (\mathbf{r}_i - \mathbf{r}_G) \times [ (\mathbf{r}_i - \mathbf{r}_G) \times \mathbf{\Omega} ] \right) + \mathbb{I}_G \cdot \dot{\mathbf{\Omega}} = \\
    & = \mathbb{I}_G \cdot \dot{\mathbf{\Omega}} + \mathbf{\Omega} \times \left( \mathbb{I}_G \cdot \mathbf{\Omega} \right)
\end{aligned}
\end{equation}

\paragraph{Terza equazione cardinale}
\begin{equation}
\begin{aligned}
    \mathbf{v}_i \cdot m_i \dot{\mathbf{v}_i} & = \mathbf{v}_i \cdot \mathbf{R}_i^{ext,i} \\
    \dfrac{d}{d t} \left( \dfrac{1}{2} m_i |\mathbf{v}_i|^2  \right) & = \mathbf{v}_i \cdot \mathbf{R}_i^{ext} + \mathbf{v}_i \cdot \mathbf{R}_i^{int} =  \mathbf{v}_i \cdot \mathbf{R}_i^{ext} + \mathbf{v}_i \cdot \sum_{j \ne i} \mathbf{F}_{ij} \\
    \dfrac{d}{d t} K_i & = P_i^{ext} + P_i^{int}
\end{aligned}
\end{equation}
e sommando su tutti i punti,
\begin{equation}
\begin{aligned}
    \dfrac{d}{dt} K & = P^{ext} + P^{int} \\
\end{aligned}
\end{equation}
dove la potenza delle forze interne in generale non è nulla
\begin{equation}
\begin{aligned}
    P^{int} & =  \sum_i  \mathbf{v}_i \cdot  \sum_{j \ne i} \mathbf{F}_{ij} =  \sum_{\{i,j\}} ( \mathbf{v}_i - \mathbf{v}_j ) \cdot \mathbf{F}_{ij} \ .
\end{aligned}
\end{equation}
{\color{red} Nel caso di corpo rigido, si ottiene \dots}

% ------------------------------------------------------------------------------
\subsection{Moto di un corpo rigido}
\paragraph{Prima equazione cardinale}
\paragraph{Seconda equazione cardinale}
\paragraph{Terza equazione cardinale}



% ==============================================================================
\chapter{Moti notevoli ed esempi}\label{mechanics:motions}
% ==============================================================================
In questo capitolo vengono indagati alcuni moti notevoli, espicitando le espressioni della quantità di moto, del momento angolare, le azioni esterne, e le condizioni iniziali proprie di ogni moto.
Ogni moto verrà indagato dopo aver introdotto il \textbf{sistema di coordinate} più \textbf{adeguato} allo studio del sistema.
\section{Moti di punti materiali}
\subsection{Moto rettilineo uniforme}
Se la risultante delle forze esterne è nulla, la derivata nel tempo della quantità di moto è nulla. Il secondo principio della dinamica si riduce a
\begin{equation}
m \ddot{\mathbf{r}} = \mathbf{0} \qquad
\begin{cases}
    \mathbf{r}(0) = \mathbf{r}_0 \\
    \dot{\mathbf{r}}(0) = \mathbf{v}_0 
\end{cases}
\end{equation}
Introducendo un sistema di coordinate cartesiane $Oxy$, si può scrivere
\begin{equation}
    \begin{cases}
        m \ddot{x} = 0 \\
        m \ddot{y} = 0 \\
    \end{cases} \qquad
    \begin{cases}
        x(0) = x_0 \quad , \qquad  \dot{x}(0) = v_{x,0} \\
        y(0) = y_0 \quad , \qquad  \dot{y}(0) = v_{y,0}
    \end{cases}
\end{equation}
Legge del moto
\begin{equation}
    \begin{cases}
        x(t) = v_{x,0} t + x_0 \\
        y(t) = v_{y,0} t + y_0
    \end{cases}
\end{equation}

\subsection{Moto di un proiettile in prossimità della superficie terrestre}
Introducendo un sistema di coordinate cartesiane $Oxy$, con l'asse $x$ diretto in direzione orizzontale, l'asse $y$ in direzione verticale verso l'alto, si può scrivere
\begin{itemize}
    \item la posizione del punto come $\mathbf{r} = x \mathbf{\hat{x}} + y \mathbf{\hat{y}}$
    \item il campo di gravità come $\mathbf{g} = -g \mathbf{\hat{y}}$
\end{itemize}
\begin{equation}
m \ddot{\mathbf{r}} = m \mathbf{g} \qquad
\begin{cases}
    \mathbf{r}(0) = \mathbf{r}_0 \\
    \dot{\mathbf{r}}(0) = \mathbf{v}_0 
\end{cases}
\end{equation}
Utilizzando questo sistema di coordinate, si possono scrivere le coordinate dell'equazione della quantità di moto e delle condizioni iniziali,
\begin{equation}
    \begin{cases}
        m \ddot{x} = 0 \\
        m \ddot{y} = -m g \\
    \end{cases} \qquad
    \begin{cases}
        x(0) = x_0 \quad , \qquad  \dot{x}(0) = v_{x,0} \\
        y(0) = y_0 \quad , \qquad  \dot{y}(0) = v_{y,0}
    \end{cases}
\end{equation}
Legge del moto
\begin{equation}
    \begin{cases}
        x(t) = v_{x,0} t + x_0 \\
        y(t) = -\dfrac{1}{2} g t^2 + v_{y,0} t + y_0
    \end{cases}
\end{equation}

\subsection{Moto sul piano inclinato}
\begin{example}[Statica sul piano inclinato]
    {\color{red} Disegno e diagramma delle forze}
\begin{equation}
    \begin{cases}
        0 & = m g \sin \alpha - \mu_s N  \\
        0 & = -m g \cos \alpha + N
    \end{cases} \qquad \rightarrow \qquad
    N = m g \cos \alpha
\end{equation}
La condizione di equilibrio può esistere solo se 
    $\mu_{s, max} \ge \mu_s = \tan \alpha$
\end{example}

\begin{example}[Moto sul piano inclinato scabro]
    {\color{red} Disegno e diagramma delle forze}
\begin{equation}
    \dot{\mathbf{Q}} = \mathbf{R}^{ext}
\end{equation}
\begin{equation}
    \begin{cases}
        m \ddot{x} & = m g \sin \alpha R - \mu_d N \frac{\dot{x}}{|\dot{x}|} \\
    0 = m \ddot{y} & = -m g \cos \alpha + N
    \end{cases} \qquad \rightarrow \qquad
    N = m g \cos \alpha
\end{equation}
\begin{equation}
    m \ddot{x} = mg \left( \sin \alpha - \mu_d \frac{\dot{x}}{|\dot{x}|} \cos \alpha  \right)
\end{equation}
\end{example}

\begin{example}[Moto sul piano inclinato con carrucola] 
    {\color{red} Disegno e diagramma delle forze}
\end{example}

\begin{example}[Moto sul piano inclinato libero di muoversi] 
    {\color{red} Disegno e diagramma delle forze}
\end{example}

\subsection{Sistema massa-molla-smorzatore}
Usando il secondo principio della dinamica
\begin{equation}
\begin{aligned}
    &  m \ddot{x}(t) + c \dot{x}(t) + k x(t) = F(t) \\
    & x(0) = x_0 \\
    & \dot{x}(0) = v_0 \\
\end{aligned}
\end{equation}
\subsection{Moto circolare}

\subsection{Moto del pendolo}
Usando la seconda equazione cardinale della dinamica
\begin{equation}
    m L^2 \ddot{\theta}(t) = - m g L \sin \theta(t) \ .
\end{equation}

\subsection{Moti impulsivi}
{\color{red}
\subsubsection{Collisioni tra punti materiali}
\subsubsection{Collisioni tra punti materiali e superfici fisse}
}

\section{Moto di corpi estesi}

\subsection{Rotolamento di un disco}
\begin{example}[Rotolamento puro su superficie orizzontale, in assenza di attrito volvente]
    {\color{red} Disegno e diagramma delle forze}

    \noindent
    \textbf{Gradi di libertà e di vincolo.} Studiando un corpo esteso nel piano, il problema ha al massimo 3 gradi di libertà. I vincoli in questo problema introducono 2 gradi di vincolo: la superficie orizzontale impone che il punto inferiore del disco sia sempre a contatto con la superficie, mentre l'ipotesi di puro rotolamento puro (senza strisciamento) impone che non ci sia velocità relativa tra il punto di contatto e la superficie.

    \noindent
    \textbf{Vincolli cinematici.}
    \begin{equation}
        x_G(t) = R \theta(t)
    \end{equation}

    \noindent
    \textbf{Equazioni del moto -- approccio con equazione cardinale del momento angolare.}
    \begin{equation}
        \dot{\mathbf{\Gamma}}_H = - \dot{\mathbf{x}}_H \times \mathbf{Q} + \mathbf{M}_H^{ext}
    \end{equation}
    \begin{equation}
        \begin{aligned}
            \mathbf{\Gamma}_H & = I_H \dot{\theta} \, \mathbf{\hat{z}} = \left( I_G + m R^2 \right) \dot{\theta} \, \mathbf{\hat{z}} \\
            \mathbf{x}_H & = R \theta \, \mathbf{\hat{x}} \\
            \mathbf{Q}   & = m \mathbf{v}_G = m R \, \dot{\theta} \, \mathbf{\hat{x}} \\
            \mathbf{M}_H^{ext} & = \mathbf{0} \\
        \end{aligned}
    \end{equation}
    %
    \begin{equation}
        \begin{cases}
            I_H \ddot{\theta} = 0 \\
            \theta(0) = 0 \\
            \dot{\theta}(0) = \Omega \\
        \end{cases}
        \qquad \rightarrow \qquad
        \begin{cases}
            \ddot{\theta}(t) = 0 \\
            \dot{ \theta}(t) = \Omega \\
                  \theta (t) = \Omega t \\
        \end{cases}
        \qquad \rightarrow \qquad
        \begin{cases}
            \ddot{x}_{G}(t) = 0 \\
            \dot{ x}_{G}(t) = R \Omega \\
                 {x}_{G}(t) = R \Omega t \\
        \end{cases}
    \end{equation}

\end{example}
\begin{example}[Rotolamento puro su superficie orizzontale, con attrito volvente]
    {\color{red} Disegno e diagramma delle forze}

\end{example}
\begin{example}[Rotolamento puro su suferficie inclinata]
    {\color{red} Disegno e diagramma delle forze}
    \begin{equation}
        \dot{\mathbf{\Gamma}}_H = - \dot{\mathbf{x}}_H \times \mathbf{Q} + \mathbf{M}_H^{ext}
    \end{equation}
    \begin{equation}
        \mathbf{M}_H^{ext} = (\mathbf{r}_G - \mathbf{r}_H) \times m \mathbf{g} = R \, \mathbf{\hat{y}} \times (-m g \sin(\alpha) \mathbf{\hat{x}} - m g \cos(\alpha) \mathbf{\hat{y}}) = m g R \sin (\alpha) \, \mathbf{\hat{z}}
    \end{equation}
    %
    \begin{equation}
        \begin{cases}
            I_H \ddot{\theta} = m g R \sin \alpha \\
            \theta(0) = 0 \\
            \dot{\theta}(0) = \Omega \\
        \end{cases}
        \qquad \rightarrow \qquad
        \begin{cases}
            \ddot{\theta}(t) = \dfrac{ m g R}{I_H} \sin (\alpha) \\
            \dot{ \theta}(t) = \dfrac{ m g R}{I_H} \sin (\alpha) \, t + \Omega \\
                  \theta (t) = \dfrac{1}{2} \dfrac{ m g R}{I_H} \sin(\alpha) \, t^2 + \Omega \, t \\
        \end{cases}
    \end{equation}
\end{example}

\begin{example}[Attrito statico e attrito dinamico in frenata -- ossia, perché non bloccare le ruote quando si frena]
\end{example}

\section{Moto dei corpi celesti}
Per ragioni di importanza storica, e per la rilevanza che ha avuto la \textit{dinamica celeste} negli sviluppi della dinamica, viene riservata un'intera sezione al moto dei corpi celesti.

\subsection{Problema dei due corpi}
La forza agente su un pianeta di massa $m$ è
\begin{equation}
    \mathbf{F} = - \dfrac{G \, M \, m}{r^2} \mathbf{\hat{r}}
\end{equation}
e quindi l'equazione del moto del pianeta è
\begin{equation}
    m \ddot{\mathbf{r}} = - \dfrac{G \, M \, m}{r^2} \mathbf{\hat{r}}
    \qquad \rightarrow \qquad
    \ddot{\mathbf{r}} = - \mu \dfrac{\mathbf{r}}{|\mathbf{r}|^3}
\end{equation}
L'orbita descritta dal pianeta è un'orbita piana. \'E possibile dimostrare questa affermazione, dimostrando che il vettore $\mathbf{n} = \mathbf{r} \times \dot{\mathbf{r}}$ è costante.
\begin{equation}
\begin{aligned}
    \dfrac{d}{dt} \mathbf{n}
    & = \dfrac{d}{dt} \left( \mathbf{r} \times \dot{\mathbf{r}} \right) = \\
    & = \underbrace{\dot{\mathbf{r}} \times \dot{\mathbf{r}}}_{=\mathbf{0}} + \mathbf{r} \times \ddot{\mathbf{r}} = \\
    & = - \mu \mathbf{r} \times \dfrac{\mathbf{r}}{|\mathbf{r}|^3} = \mathbf{0}
\end{aligned}
\end{equation}
\'E quindi possibile usare un sistema di coordinate polari.

\noindent
Usando le coordinate polari,
\begin{equation}
\begin{aligned}
    & \mathbf{r} = r \, \mathbf{\hat{r}} \\
    & \dot{\mathbf{r}} = \dot{r} \, \mathbf{\hat{r}} + r \dot{\theta} \boldsymbol{\hat{\theta}} \\
    & \ddot{\mathbf{r}} = \left( \ddot{r} - r \dot{\theta}^2 \right) \, \mathbf{\hat{r}} + \left( 2 \dot{r} \dot{\theta} + r \ddot{\theta} \right) \, \boldsymbol{\hat{\theta}} \\
\end{aligned}
\end{equation}
si può scrivere l'equazione del moto in componenti
\begin{equation}
\begin{aligned}
    & r:      \quad \ddot{r} - r \dot{\theta}^2 + \frac{\mu}{r^2} = 0 \\
    & \theta: \quad 2 \dot{r} \dot{\theta} + r \ddot{\theta} = 0 \\
\end{aligned}
\end{equation}
La seconda, moltiplicata per $r$, si può scrivere
\begin{equation}
    0 = r \left( 2 \dot{r} \dot{\theta} + r \ddot{\theta}\right) = \dfrac{d}{dt} \left( r^2 \dot{\theta} \right ) \ ,
\end{equation}
che equivale alla costanza del momento angolare
\begin{equation}
    \mathbf{L} = \mu \mathbf{r} \times \dot{\mathbf{r}} = \mu r^2 \dot{\theta} \mathbf{\hat{n}} \ .
\end{equation}
\paragraph{Traiettoria, $r(\theta)$.} Si definisce $z := \frac{1}{r}$, e calcolando le derivate di $r$
\begin{equation}
    \dot{r} = \dfrac{d r}{d \theta} \dot{\theta} = - z^{-2} z'(\theta) \dot{\theta} = - \ell z'(\theta) 
\end{equation}
poiché
\begin{equation}
    \dot{\theta} = \ell r^{-2} \qquad \rightarrow \qquad \ddot{\theta} = -2 \ell r^{-3} \dot{r} \ .
\end{equation}
Derivando un'altra volta
\begin{equation}
    \ddot{r} = - \ell z''(\theta) \dot{\theta} = - \ell^2 z(\theta) z''(\theta)\ .
\end{equation}
Sostituendo nella componente radiale dell'equazione di moto
\begin{equation}
    - \ell^2 z^2(\theta) z''(\theta) - \ell^2 z^3(\theta) + \mu z^2(\theta) = 0 \ ,
\end{equation}
e dividendo per $z^2(\theta)$
\begin{equation}
    z''(\theta) + z(\theta) = \frac{\mu}{\ell^2} \ ,
\end{equation}
la cui soluzione è
\begin{equation}
    \dfrac{1}{r(\theta)} = z(\theta) = A \, \cos \theta + B \, \sin \theta + \frac{\mu}{\ell^2} \ .
\end{equation}

\subsubsection{Leggi di Keplero}
\begin{theorem}[Prima legge di Keplero] L'orbita di un pianeta è un'ellisse e il Sole si trova in uno dei due fuochi.
\end{theorem}
\begin{theorem}[Seconda legge di Keplero] La velocità areolare di un pianeta è costante lungo la sua orbita.
\end{theorem}
L'area elementare $dA$ descritta da uno spostamento $d \mathbf{r}$ vale
\begin{equation}
    d A \mathbf{\hat{n}} = \frac{1}{2} \mathbf{r} \times d \mathbf{r}\ .
\end{equation}
La velocità areolare è quindi
\begin{equation}
    \dot{A} \mathbf{\hat{n}} = \dfrac{d A}{d t} \mathbf{\hat{n}} = \dfrac{1}{2} \mathbf{r} \times \dot{\mathbf{r}} \ .
\end{equation}
Questo vettore è esattamente il vettore $\mathbf{n}$ che è stato dimostrato essere costante in precedenza.


\begin{theorem}[Terza legge di Keplero] I quadrati dei periodi di due orbite sono proporzionali al cubo dei loro assi maggiori.
\end{theorem}

\subsection{Problema dei tre corpi -- cenni}


