

% ==============================================================================
\chapter{Cinematica}
\begin{definition}[Cinematica] La cinemcatica è quel ramo della meccanica che si occupa di descrivere il moto dei corpi, indipendentemente dalle cause del moto.
\end{definition}
Il moto dei corpi viene descritto usando i concetti di \textbf{tempo} e \textbf{spazio}.

\section{Cinematica del punto materiale}
\begin{definition}[Posizione]
Una volta definito un sistema di riferimento, si può definire la posizione di un punto $P$ con il raggio vettore $\mathbf{r}_P$ che unisce l'origine del sistema di riferimento con la posizione del punto $P$.
\end{definition}

\begin{definition}[Velocità]
    La velocità di un punto $P$ rispetto a un sistema di riferimento viene definita come la derivata nel tempo del raggio vettore $\mathbf{r}_P$,
    \begin{equation}
        \mathbf{v}_P = \dfrac{d}{dt}\mathbf{r}_P  =: \dot{ \mathbf{r}}_P  \ .
    \end{equation}
\end{definition}

\begin{definition}[Accelerazione]
    L'accelerazione di un punto $P$ rispetto a un sistema di riferimento viene definita come la derivata nel tempo della velocità  $\mathbf{v}_P$ del punto, equivalente alla derivata seconda della posizione
    \begin{equation}
        \mathbf{a}_P = \dot{ \mathbf{v}}_P = \ddot{ \mathbf{r}}_P \ .
    \end{equation}
\end{definition}

\section{Cinematica di un insieme di punti materiali}

\section{Cinematica del corpo rigido}

\section{Vincoli}\label{mechanics:kinematics:constraints}

% ==============================================================================
\chapter{Massa, proprietà inerziali e dinamiche}

\section{Proprietà inerziali}

\begin{definition}[Massa] La massa viene definita come la quantità di materia.
\end{definition}
{\color{red} Per la definizione operativa della massa in dinamica si rimanda al capitolo (\ref{mechanics:dynamics}).}

\subsection{Centro di massa}
\begin{equation}
    \mathbf{r}_G = \dfrac{1}{m} \sum_{i=1}^n m_i \mathbf{r}_i \ .
\end{equation}
\subsection{Momento statico}
\subsection{Momento di inerzia}

\section{Quantità dinamiche}
In questa sezione vengono definite alcune quantità dinamiche che risulteranno utili nello studio della dinamica dei corpi nel capitolo \ref{mechanics:dynamics}. In particolare, vengono definiti:
\begin{itemize}
  \item la quantità di moto
  \item il momento della quantità di moto
  \item l'energia cinetica
\end{itemize}
per i sistemi meccanici modellabili come punto materiale, insieme di punti materiali, e corpi rigidi.
Queste quantità sono \textbf{grandezze additive}.

\subsection{Punto materiale}
\subsubsection{Quantità di moto}
\begin{equation}
  \mathbf{Q} = m \mathbf{v}
\end{equation}
\subsubsection{Momento della quantità di moto}
\begin{equation}
    \mathbf{\Gamma}_0 = \left(\mathbf{r} - \mathbf{r}_0 \right) \times \mathbf{Q}
\end{equation}
\subsubsection{Energia cinetica}
\begin{equation}
    K = \dfrac{1}{2} m | \mathbf{v} |^2
\end{equation}

\subsection{Sistema di punti materiali}
\subsubsection{Quantità di moto}
\begin{equation}
    \mathbf{Q} = \sum_{i} \mathbf{Q}_i = \sum_i m_i \mathbf{v}_i
\end{equation}
\subsubsection{Momento della quantità di moto}
\begin{equation}
    \mathbf{\Gamma}_O = \sum_{i} \mathbf{\Gamma}_{O,i} = \sum_i \left( \mathbf{r}_i - \mathbf{r}_0 \right) \times \mathbf{Q}_i
\end{equation}
\subsubsection{Energia cinetica}
\begin{equation}
    K = \sum_i K_i = \sum_i \dfrac{1}{2} m_i | \mathbf{v}_i |^2
\end{equation}

\subsubsection{Corpi rigidi}
\subsubsection{Momento della quantità di moto}
\subsubsection{Energia cinetica}

\subsection{Sistema di punti materiali}
% ==============================================================================
\chapter{Azioni}
\begin{definition}[Forza] Una forza è un'entità fisica vettoriale che è in grado di cambiare lo stato del moto di un sistema.
\end{definition}

\section{Momento di una forza}
\begin{definition}[Momento di una forza] Il momento di una forza rispetto a un \textbf{polo} $O$, $\mathbf{M}_O$, viene definito coem il prodotto vettoriale tra il raggio $\Delta \mathbf{r}_{OP} = \mathbf{r}_P - \mathbf{r}_O$ tra il polo e il punto di applicazione e la forza $\mathbf{F}$,
    \begin{equation} \mathbf{M}_O = \Delta \mathbf{r}_{OP} \times \mathbf{F} \ .
    \end{equation}
\end{definition}

\section{Lavoro di una forza}
\begin{definition}[Lavoro elementare di una forza] Il lavoro elementare $\delta L$ di una forza $\mathbf{F}$ è il prodotto scalare tra la forza e lo spostamento elementare $d \mathbf{r}$,
    \begin{equation}
        \delta L = \mathbf{F} \cdot d \mathbf{r} \ .
    \end{equation}
\end{definition}
Il lavoro di una forza è la somma dei lavori elementari. Se si suddivide la storia degli spostamenti in una somma di spostamenti incrementali,
\begin{equation}
  \Delta \mathbf{r} = \sum_{i=1}^{n} \Delta \mathbf{r}_i \ ,
\end{equation}
e si considera la forza applicata costante (in intensità e direzione) a tratti su ogni spostamento incrementale $\Delta \mathbf{r}_i$, il lavoro della forza viene definito come la somma dei lavori incrementali, $\Delta L_i = \mathbf{F}_i \cdot \Delta \mathbf{r}_i$
\begin{equation}
  L = \sum_{i=1}^{N} \mathbf{F}_i \cdot \Delta \mathbf{r}_i \ .
\end{equation}
\begin{definition}[Lavoro di una forza]
Se si fa tendere a zero la lunghezza degli spostamenti incrementali, il lavoro di una forza viene definito come l'integrale (di linea) di Riemann
\begin{equation}
    L = \int_{\gamma(\mathbf{r})} \mathbf{F} \cdot d \mathbf{r} \ ,
\end{equation}
avendo indicato con $\gamma(\mathbf{r})$ la curva nello spazio percorsa dal corpo, sulla quale avvengono gli spostamenti elementari.
\end{definition}

\section{Impulso di una forza}
\begin{definition}[Impulso elementare di una forza]
    L'impulso elementare $d \mathbf{I}$ di una forza $\mathbf{F}$ viene definito come il prodotto della forza e l'intervallo elementare di tempo $dt$ per il quale agisce la forza,
    \begin{equation}
        \delta \mathbf{I} = \mathbf{F} dt \ .
    \end{equation}
\end{definition}
\begin{definition}[Impulso di una forza] Viene definito come la somma di tutti gli impulsi elementari tra due istanti desiderati,  $t \in [t_0, t_1]$, e al limite si riduce all'integrale di Riemann
    \begin{equation}
        \mathbf{I} = \int_{t=t_0}^{t_1} \delta \mathbf{I} = \int_{t=t_0}^{t_1} \mathbf{F} dt \ .
    \end{equation}
\end{definition}

\section{Esempi di forza}
\subsection{Reazioni vincolari}
{\color{red} Fare riferimento alla sezione \ref{mechanics:kinematics:constraints}}

\subsection{Forza peso in prossimità della superficie terrestre}
\subsection{Forza elastica}
\subsection{Forza di attrito}
\subsubsection{Attrito statico}
\subsubsection{Attrito dinamico}



% ==============================================================================
\chapter{Statica}

\section{Condizioni di equilibrio}
\begin{definition}[Condizioni di equilibrio per un punto materiale]
    \begin{equation}
        \sum_{i=1}^{n} \mathbf{F}_i = \mathbf{0}
    \end{equation}
\end{definition}

\begin{definition}[Condizioni di equilibrio per un corpo rigido]
    \begin{equation}
    \begin{aligned}
        \sum_{i=1}^{n} \mathbf{F}_i & = \mathbf{0} \\
        \sum_{i=1}^{n} \mathbf{M}_{O,i} & = \mathbf{0}
    \end{aligned}
    \end{equation}
\end{definition}

\section{Esempi}



% ==============================================================================
\chapter{Dinamica}\label{mechanics:dynamics}
\section{Principi della dinamica di Newton}
\begin{enumerate}
  \item principio di inerzia
  \item secondo principio della dinamica
  \item principio di azione e reazione
\end{enumerate}
\subsection{Primo principio della dinamica}
Un corpo imperturbato, sul quale agisce un sistema di forza dalla risultante nulla, rimane nello stato di quiete o in moto rettilineo uniforme rispetto a un sistema di \textbf{riferimento inerziale}.

\noindent
{\color{red}{Cosa intendiamo per sistema di riferimento inerziale?}}

\subsection{Secondo principio della dinamica}
\begin{equation}
    \Delta \mathbf{Q} = \mathbf{I}^{ext} \ ,
\end{equation}
o in forma differenziale
\begin{equation}
    \dfrac{d}{dt}\mathbf{Q} = \mathbf{R}^{ext}
\end{equation}

\subsection{Terzo principio della dinamica}

% ------------------------------------------------------------------------------
\section{Equazioni cardinali della dinamica}
% ------------------------------------------------------------------------------
\subsection{Moto di un punto materiale}
\paragraph{Prima equazione cardinale -- quantità di moto}
\begin{equation}
    \dfrac{d}{dt} \mathbf{Q} = \mathbf{R}^{ext}
\end{equation}

\paragraph{Seconda equazione cardinale -- momento della quantità di moto}
\begin{equation} \mathbf{M}_O = \mathbf{r} \times \mathbf{Q} \end{equation}
\begin{equation}
\begin{aligned}
    \dfrac{d}{dt} \mathbf{\Gamma}_O
    & = \dfrac{d}{dt} \left( (\mathbf{r} - \mathbf{r}_O ) \times \mathbf{Q} \right) = \\
    & = \dot{\mathbf{r}} \times \mathbf{Q} - \dot{\mathbf{r}}_O \times \mathbf{Q} + \mathbf{r} \times \dot{\mathbf{Q}} = \\
    & = \underbrace{\dot{\mathbf{r}} \times m \dot{\mathbf{r}}}_{=\mathbf{0}} - \dot{\mathbf{r}}_O \times \mathbf{Q} + \mathbf{r} \times \mathbf{R}^{ext} = - \dot{\mathbf{r}}_O \times \mathbf{Q} + \mathbf{M}_O^{ext} \ .
\end{aligned}
\end{equation}
\paragraph{Terza equazione cardinale -- energia cinetica}

\begin{equation} K = \dfrac{1}{2} m |\mathbf{v}|^2 = \dfrac{1}{2} m \mathbf{v} \cdot \mathbf{v} \ .
\end{equation}
\begin{equation}
\begin{aligned}
    \dfrac{d}{dt} K & = \underbrace{m \dot{\mathbf{v}}}_{=\dot{\mathbf{Q}}} \cdot \mathbf{v} = \\
    & = \dot{\mathbf{Q}} \cdot \mathbf{v} = \\
    & = \mathbf{R}^{ext} \cdot \mathbf{v} = P^{ext} = P^{tot}
\end{aligned}
\end{equation}

% ------------------------------------------------------------------------------
\subsection{Moto di un sistema di punti materiali}
\paragraph{Prima equazione cardinale}
L'equazione cardinale per il punto $i$-esimo,
\begin{equation}
\begin{aligned}
    m_i \ddot{\mathbf{r}}_i & = \mathbf{R}^{ext,i}_i = \\
    & = \mathbf{R}^{ext}_i + \mathbf{R}^{int}_i = \\
    & = \mathbf{R}^{ext}_i + \sum_{j \ne i} \mathbf{F}_{ij} \\
\end{aligned}
\end{equation}
Sommando le equazioni di tutti i punti, si ottiene
\begin{equation}
    \underbrace{\sum_{i} m_i \ddot{\mathbf{r}}_i}_{m \ddot{\mathbf{r}}_G = \dot{\mathbf{Q}} } = \underbrace{\sum_{i} \mathbf{R}^{ext}_i}_{\mathbf{R}^{ext}} + \underbrace{\sum_{i} \sum_{j \ne i} \mathbf{F}_{ij}}_{=\mathbf{0}} \\
\end{equation}
\begin{equation}
    \dot{\mathbf{Q}} = \mathbf{R}^{ext} 
\end{equation}

\paragraph{Seconda equazione cardinale}
\begin{equation}
\begin{aligned}
    & \dfrac{d}{dt} \left( (\mathbf{r}_i-\mathbf{r}_O) \times m_i \dot{\mathbf{r}}_i \right) = (\mathbf{r}_i-\mathbf{r}_O) \times \mathbf{R}^{ext,i}_i \\
    & ( \dot{\mathbf{r}}_i - \dot{\mathbf{r}}_O ) \times m_i \dot{\mathbf{r}}_i + (\mathbf{r}_i-\mathbf{r}_O) \times m_i \ddot{\mathbf{r}}_i =  (\mathbf{r}_i-\mathbf{r}_O) \times \left( \mathbf{R}^{ext}_i + \sum_{j\ne i} \mathbf{F}_{ij} \right) \\
    & \sum_i (\mathbf{r}_i-\mathbf{r}_O) \times m_i \ddot{\mathbf{r}}_i = \dot{\mathbf{r}}_O \times \mathbf{Q} + \mathbf{M}_O^{ext}
\end{aligned}
\end{equation}
%
Per sistemi rigidi, per i quali vale $\mathbf{v}_i - \mathbf{v}_j = \mathbf{\Omega} \times (\mathbf{r}_i - \mathbf{r}_j)$
\begin{equation}
\begin{aligned}
    \mathbf{\Gamma}_O & = \sum_{i} \left( \mathbf{r}_i - \mathbf{r}_O \right) \times m_i \mathbf{v}_i = \\
    & = \sum_i \left( \mathbf{r}_i - \mathbf{r}_G + \mathbf{r}_G - \mathbf{r}_O \right) \times m_i \left[ \mathbf{v}_G + \mathbf{\Omega} \times (\mathbf{r}_i - \mathbf{r}_G) \right] = \\
    & = \underbrace{\sum_i m_i (\mathbf{r}_i - \mathbf{r}_G)}_{m \mathbf{r}_G - m \mathbf{r}_G=\mathbf{0}} \times \mathbf{v}_G +
        \underbrace{\sum_i m_i}_{=m} (\mathbf{r}_G - \mathbf{r}_O) \times \mathbf{v}_G + \\
    & \quad + (\mathbf{r}_G - \mathbf{r}_0) \times \bigg( \mathbf{\Omega} \times \underbrace{\sum_i m_i (\mathbf{r}_i - \mathbf{r}_G)}_{m \mathbf{r}_G - m \mathbf{r}_G=\mathbf{0}} \bigg)
    \underbrace{- \sum_i m_i (\mathbf{r}_i - \mathbf{r}_G ) \times [ (\mathbf{r}_i - \mathbf{r}_G ) \times }_{=\mathbb{I}_G} \mathbf{\Omega} ]
\end{aligned}
\end{equation}
\begin{equation}
    \mathbf{\Gamma}_O = (\mathbf{r}_G - \mathbf{r}_O) \times \mathbf{Q} + \mathbb{I}_G \cdot \mathbf{\Omega}
\end{equation}
\begin{equation}
\begin{aligned}
    \dfrac{d}{dt} \left( (\mathbf{r}_G - \mathbf{r}_O) \times \mathbf{Q}  \right) & = ( \mathbf{v}_G - \dot{\mathbf{r}}_O ) \times \mathbf{Q} + (\mathbf{r}_G - \mathbf{r}_O) \times \dot{\mathbf{Q}} = \\
    & = - \dot{\mathbf{r}}_O \times \mathbf{Q} + (\mathbf{r}_G - \mathbf{r}_O) \times \dot{\mathbf{Q}}
\end{aligned}
\end{equation}
\begin{equation}
\begin{aligned}
    \dfrac{d}{dt} \left( \mathbb{I}_G \cdot \mathbf{\Omega} \right) & =
    \dfrac{d}{dt} \left( - \sum_i m_i (\mathbf{r}_i - \mathbf{r}_G ) \times \left[ (\mathbf{r}_i - \mathbf{r}_G ) \times \mathbf{\Omega} \right]  \right) = \\
    & = - \sum_i  m_i (\mathbf{v}_i - \mathbf{v}_G ) \times \left[ (\mathbf{r}_i - \mathbf{r}_G ) \times \mathbf{\Omega} \right] + \\
    & \quad - \sum_i  m_i (\mathbf{r}_i - \mathbf{r}_G ) \times \left[ (\mathbf{v}_i - \mathbf{v}_G ) \times \mathbf{\Omega} \right] + \\
    & \quad - \sum_i  m_i (\mathbf{r}_i - \mathbf{r}_G ) \times \left[ (\mathbf{r}_i - \mathbf{r}_G ) \times \dot{\mathbf{\Omega}} \right] = \\
    & = - \sum_i  m_i \underbrace{\left[ (\mathbf{r}_i - \mathbf{r}_G ) \times \mathbf{\Omega} \right]  \times \left[ (\mathbf{r}_i - \mathbf{r}_G ) \times \mathbf{\Omega} \right]}_{= \mathbf{0}} + \\
    & \quad - \sum_i  m_i (\mathbf{r}_i - \mathbf{r}_G ) \times \left[ (\mathbf{v}_i - \mathbf{v}_G ) \times \mathbf{\Omega} \right] + \\
    & \quad + \mathbb{I}_G \cdot \dot{\mathbf{\Omega}} \\
    & = + \sum_i  m_i \mathbf{\Omega} \times [ (\mathbf{r}_i - \mathbf{r}_G) \times \underbrace{(\mathbf{v}_i - \mathbf{v}_G)}_{\mathbf{\Omega} \times (\mathbf{r}_i - \mathbf{r}_G) } ] + \\
    & \quad + \sum_i  m_i \underbrace{(\mathbf{v}_i - \mathbf{v}_G) \times [ \mathbf{\Omega} \times (\mathbf{r}_i - \mathbf{r}_G ) ]}_{= \mathbf{0}}  + \\
    & \quad + \mathbb{I}_G \cdot \dot{\mathbf{\Omega}} \\
    & = \mathbf{\Omega} \times \left( - \sum_i (\mathbf{r}_i - \mathbf{r}_G) \times [ (\mathbf{r}_i - \mathbf{r}_G) \times \mathbf{\Omega} ] \right) + \mathbb{I}_G \cdot \dot{\mathbf{\Omega}} = \\
    & = \mathbb{I}_G \cdot \dot{\mathbf{\Omega}} + \mathbf{\Omega} \times \left( \mathbb{I}_G \cdot \mathbf{\Omega} \right)
\end{aligned}
\end{equation}

\paragraph{Terza equazione cardinale}
\begin{equation}
\begin{aligned}
    \mathbf{v}_i \cdot m_i \dot{\mathbf{v}_i} & = \mathbf{v}_i \cdot \mathbf{R}_i^{ext,i} \\
    \dfrac{d}{d t} \left( \dfrac{1}{2} m_i |\mathbf{v}_i|^2  \right) & = \mathbf{v}_i \cdot \mathbf{R}_i^{ext} + \mathbf{v}_i \cdot \mathbf{R}_i^{int} =  \mathbf{v}_i \cdot \mathbf{R}_i^{ext} + \mathbf{v}_i \cdot \sum_{j \ne i} \mathbf{F}_{ij} \\
    \dfrac{d}{d t} K_i & = P_i^{ext} + P_i^{int}
\end{aligned}
\end{equation}
e sommando su tutti i punti,
\begin{equation}
\begin{aligned}
    \dfrac{d}{dt} K & = P^{ext} + P^{int} \\
\end{aligned}
\end{equation}
dove la potenza delle forze interne in generale non è nulla
\begin{equation}
\begin{aligned}
    P^{int} & =  \sum_i  \mathbf{v}_i \cdot  \sum_{j \ne i} \mathbf{F}_{ij} =  \sum_{\{i,j\}} ( \mathbf{v}_i - \mathbf{v}_j ) \cdot \mathbf{F}_{ij} \ .
\end{aligned}
\end{equation}
{\color{red} Nel caso di corpo rigido, si ottiene \dots}

% ------------------------------------------------------------------------------
\subsection{Moto di un corpo rigido}
\paragraph{Prima equazione cardinale}
\paragraph{Seconda equazione cardinale}
\paragraph{Terza equazione cardinale}



% ==============================================================================
\chapter{Moti notevoli}\label{mechanics:motions}
In questo capitolo vengono indagati alcuni moti notevoli, espicitando le espressioni della quantità di moto, del momento angolare, le azioni esterne, e le condizioni iniziali proprie di ogni moto.
Ogni moto verrà indagato dopo aver introdotto il \textbf{sistema di coordinate} più \textbf{adeguato} allo studio del sistema.
\section{Moti di punti materiali}
\subsection{Moto rettilineo uniforme}
Se la risultante delle forze esterne è nulla, la derivata nel tempo della quantità di moto è nulla. Il secondo principio della dinamica si riduce a
\begin{equation}
m \ddot{\mathbf{r}} = \mathbf{0} \qquad
\begin{cases}
    \mathbf{r}(0) = \mathbf{r}_0 \\
    \dot{\mathbf{r}}(0) = \mathbf{v}_0 
\end{cases}
\end{equation}
Introducendo un sistema di coordinate cartesiane $Oxy$, si può scrivere
\begin{equation}
    \begin{cases}
        m \ddot{x} = 0 \\
        m \ddot{y} = 0 \\
    \end{cases} \qquad
    \begin{cases}
        x(0) = x_0 \quad , \qquad  \dot{x}(0) = v_{x,0} \\
        y(0) = y_0 \quad , \qquad  \dot{y}(0) = v_{y,0}
    \end{cases}
\end{equation}
Legge del moto
\begin{equation}
    \begin{cases}
        x(t) = v_{x,0} t + x_0 \\
        y(t) = v_{y,0} t + y_0
    \end{cases}
\end{equation}

\subsection{Moto di un proiettile in prossimità della superficie terrestre}
Introducendo un sistema di coordinate cartesiane $Oxy$, con l'asse $x$ diretto in direzione orizzontale, l'asse $y$ in direzione verticale verso l'alto, si può scrivere
\begin{itemize}
    \item la posizione del punto come $\mathbf{r} = x \mathbf{\hat{x}} + y \mathbf{\hat{y}}$
    \item il campo di gravità come $\mathbf{g} = -g \mathbf{\hat{y}}$
\end{itemize}
\begin{equation}
m \ddot{\mathbf{r}} = m \mathbf{g} \qquad
\begin{cases}
    \mathbf{r}(0) = \mathbf{r}_0 \\
    \dot{\mathbf{r}}(0) = \mathbf{v}_0 
\end{cases}
\end{equation}
Utilizzando questo sistema di coordinate, si possono scrivere le coordinate dell'equazione della quantità di moto e delle condizioni iniziali,
\begin{equation}
    \begin{cases}
        m \ddot{x} = 0 \\
        m \ddot{y} = -m g \\
    \end{cases} \qquad
    \begin{cases}
        x(0) = x_0 \quad , \qquad  \dot{x}(0) = v_{x,0} \\
        y(0) = y_0 \quad , \qquad  \dot{y}(0) = v_{y,0}
    \end{cases}
\end{equation}
Legge del moto
\begin{equation}
    \begin{cases}
        x(t) = v_{x,0} t + x_0 \\
        y(t) = -\dfrac{1}{2} g t^2 + v_{y,0} t + y_0
    \end{cases}
\end{equation}
\subsection{Sistema massa-molla-smorzatore}
Usando il secondo principio della dinamica
\begin{equation}
\begin{aligned}
    &  m \ddot{x}(t) + c \dot{x}(t) + k x(t) = F(t) \\
    & x(0) = x_0 \\
    & \dot{x}(0) = v_0 \\
\end{aligned}
\end{equation}
\subsection{Moto circolare}

\subsection{Moto del pendolo}
Usando la seconda equazione cardinale della dinamica
\begin{equation}
    m L^2 \ddot{\theta}(t) = - m g L \sin \theta(t) \ .
\end{equation}

\subsection{Moto di sistemi di punti}
\subsection{Moti dei corpi celesti}

\section{Moti di corpi estesi}


